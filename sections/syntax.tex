% SPDX-FileCopyrightText: Copyright (c) 2016-2025 Objectionary.com
% SPDX-License-Identifier: MIT

The entire syntax of the \eolang{} language in BNF is available on the first page of the
\texttt{objectionary/eo} GitHub repository\footnote{\url{https://github.com/objectionary/eo}}.
Here, we only explain the basics of the language, not its entire syntax.

Similar to Python~\citep{lutz2013learning}, indentation in \eolang{} is part of the syntax:
the scope of a code block is determined by its horizontal position
in relation to other blocks, which is also known as ``off-side rule''~\citep{landin1966next}.

There are no keywords in \eolang{} but only a few special symbols
denoting grammar constructs, such as \ff{>} for attribute naming,
\ff{.} for dot notation, and \ff{[]} for specifying parameters of formations.

Attributes, which are the only identifiers that exist in \eolang{}, may have
any Unicode symbols in their names, as long as they start with a small English letter
and do not contain spaces, line breaks, or the symbols that are part of the syntax:
\ff{test-File} and
\begin{CJK}{UTF8}{gbsn}
\ff{i文件}
\end{CJK}
 are valid identifiers.
Identifiers are case-sensitive: \ff{acar} and \ff{aCar} are two different identifiers.
Java notation is used for numbers and strings.

\subsection{Objects}

This is a \emph{formation} of a new object \ff{book} that has a single \ff{isbn} attribute:

\begin{ffcode}
# A book in the library.
[isbn] > book (*@\label{ln:book}@*)
\end{ffcode}

To make another object with a specific ISBN, the \ff{book}
has to be \emph{copied}, with the \emph{data} as an argument:

\begin{ffcode}
book "978-1519166913" > b1
\end{ffcode}

Here, \ff{b1} is a new object created.
Its only attribute is accessible as \ff{b1.isbn}.

\subsection{Indentation}

This is an example of an \emph{abstract} object \ff{vector}, where
spaces are replaced with the ``\textvisiblespace'' symbol to demonstrate
the importance of their presence in specific quantities
(for example, there must be exactly one space after the closing square bracket on the
second line and the \ff{>} symbol, while two spaces will break the syntax):

{\lstset{showspaces=true}\begin{ffcode}
# This is a vector in 2D space.(*@\label{ln:comment}@*)
[dx dy] > vector(*@\label{ln:vector}@*)
  sqrt. > length(*@\label{ln:length}@*)
    plus.
      dx.times dx
      dy.times dy(*@\label{ln:length-end}@*)(*@\label{ln:vector-end}@*)
\end{ffcode}
}

The code at \lref{comment} is a \emph{comment}.
Two \emph{void attributes}, \ff{dx} and \ff{dy},
are listed in square brackets at \lref{vector}.
The \emph{name} of the object goes after the \ff{>} symbol.
The code at \lref{length} defines
an \emph{attached attribute} \ff{length}. Wherever an object
needs to get a name, the \ff{>} symbol can be added after the object.

The declaration of the attribute \ff{length} at \lrefs{length}{length-end}
can be written in one line, using \emph{dot notation}:

\begin{ffcode}
((dx.times dx).plus (dy.times dy)).sqrt > length
\end{ffcode}

An \emph{inverse} dot notation is used to simplify
the syntax. The identifier that goes after the dot is written
first, the dot follows, and the next line contains the part
that is supposed to come before the dot. It is also possible to rewrite
this expression in multiple lines without the usage of
inverse notation, but it will look subjectively less readable:

\begin{ffcode}
dx.times dx (*@\label{ln:dx-pow}@*)
.plus
  dy.times dy (*@\label{ln:dx-pow-2}@*)
.sqrt > length (*@\label{ln:dx-pow-3}@*)
\end{ffcode}

Here, \lref{dx-pow} is the application of the object \ff{dx.times} with
a new argument \ff{dx}. Then, the next line is the object \ff{plus} taken
from the object created at the first line, using dot notation. Then,
\lref{dx-pow-2} is the argument passed to the object \ff{plus}.
The code at \lref{dx-pow-3} takes the object \ff{sqrt} from the object constructed
at the previous line, and gives it the name \ff{length}.

Indentation is used for two purposes: either to define attributes
of an abstract object or to specify arguments for object application, also
known as making a \emph{copy}.

A definition of an abstract object starts with a list of void attributes
in square brackets on one line, followed by a list of attached attributes
each in its own line. For example, this is an abstract \emph{anonymous} object
(it does not have a name)
with one void attribute \ff{x} and two attached attributes \ff{succ} and \ff{prev}:

\begin{ffcode}
# A counter.
[x]
  x.plus 1 > succ
  x.minus 1 > prev
\end{ffcode}

The arguments of \ff{plus} and \ff{minus} are provided in \emph{horizontal}
mode, without the use of indentation. It is possible to rewrite this code
in a \emph{vertical} mode, where indentation will be required:

\begin{ffcode}
# A counter.
[x] (*@\label{ln:succ}@*)
  x.plus > succ
    1
  x.minus > prev
    1 (*@\label{ln:succ-end}@*)
\end{ffcode}

This abstract object can also be written in a horizontal mode,
because it is anonymous:

\begin{ffcode}
[x] (x.plus 1 > succ) (x.minus 1 > prev)
\end{ffcode}

\subsection{Data}

There is only one abstract object that can encapsulate data: \ff{bytes}.
Copies of the \ff{bytes} object are created by the compiler when it meets
a special syntax for data, for example:

\begin{ffcode}
3.14 > radius
\end{ffcode}

This code is compiled into the following:

\begin{ffcode}
number > radius
  bytes
    40-09-1E-B8-51-EB-85-1F
\end{ffcode}

Similar to the floating-point number compilation into a copy
of \ff{bytes}, there are a few other syntax shortcuts, listed in~\cref{tab:types}.

\begin{table}
\begin{tabularx}{\columnwidth}{l|X|r}
\toprule
Data & Example & Size \\
\midrule
\ff{bytes} & \ff{1F-E5-77-A6} & 4 \\
\ff{string} & \ff{"Hello, \foreignlanguage{russian}{друг}!"} & 16 \\
  & \ff{"\textbackslash{}u5BB6"} or \begin{CJK}{UTF8}{gbsn}\ff{"家"}\end{CJK} & 2 \\
\ff{number} & \ff{1024}, \ff{0x1A7E}, or \ff{-42.133e14} & 8 \\
\bottomrule
\end{tabularx}
\figcap{The syntax of all data with examples. The ``Size'' column
denotes the number of bytes in the \ff{as-bytes} attribute.
UTF-8 is the encoding used in \ff{string} object.}
\label{tab:types}
\end{table}

\subsection{Tuples}

The \ff{tuple} is yet another syntax sugar, for arrays:

\begin{ffcode}
* "Lucy" "Jeff" 3.14 (*@\label{ln:tuple-1}@*)
* > a (*@\label{ln:tuple-2a}@*)
  (* "a")
  TRUE (*@\label{ln:tuple-2b}@*)
* > b (*@\label{ln:tuple-3}@*)
\end{ffcode}

The code at \lref{tuple-1} makes a tuple of three elements: two strings
and one float. The code at \lrefs{tuple-2a}{tuple-2b} makes a tuple named \ff{a} with another
tuple as its first element and \ff{TRUE} as the second item.
The code at \lref{tuple-3} is an empty tuple with the name \ff{b}.

\subsection{Scope Brackets}

Brackets can be used to group object arguments in horizontal mode:

\begin{ffcode}
sum (div 45 5) 10  (*@\label{ln:sum}@*)
\end{ffcode}

The \ff{(div 45 5)} is a copy of the abstract object \ff{div}
with two arguments \ff{45} and \ff{5}. This object itself is
the first argument of the copy of the object \ff{sum}. Its second
argument is \ff{10}. Without brackets the syntax would read differently:

\begin{ffcode}
sum div 45 5 10
\end{ffcode}

This expression denotes a copy of \ff{sum} with four arguments.

\subsection{Inner Objects}

An object may have other abstract objects as its attributes, for example:

\begin{ffcode}
# A point on a 2D canvas.
[x y] > point
  [to] > distance
    length. > len (*@\label{ln:vector-length}@*)
      vector
        to.x.minus (^.x)
        to.y.minus (^.y)
\end{ffcode}

The object \ff{point} has two void attributes \ff{x} and \ff{y}
and the attribute \ff{distance}, which is attached to an abstract
object with one void attribute \ff{to} and one attached attribute \ff{len}.
The \emph{inner} abstract object \ff{distance} may only be copied
with a reference to its \emph{parent} object \ff{point}, via
a special attribute denoted by the \ff{\^{}} symbol:

\begin{ffcode}
distance. (*@\label{ln:point-copy}@*)
  point
    5:x
    -3:y
  point:to
    13:x
    3.9:y
\end{ffcode}

The parent object is \ff{(point 5 -3)}, while the object \ff{(point 13 3.9)}
is the argument for the void attribute \ff{to} of the object \ff{distance}.
Suffixes \ff{:x}, \ff{:y}, and \ff{:to} are optional and may be used
to denote the exact name of the void attribute to be attached to the
provided argument.

\subsection{Decorators}

An object may extend another object by \emph{decorating} it:

\begin{ffcode}
# A circle with a center and radius.
[center radius] > circle (*@\label{ln:circle}@*)
  center > @ (*@\label{ln:circle-phi}@*)
  [p] > is-inside
    lte. > @
      ^.@.distance $.p (*@\label{ln:circle-phi-2}@*)
      ^.radius (*@\label{ln:circle-end}@*)
\end{ffcode}

The object \ff{circle} has a special attribute \ff{@}
at \lref{circle-phi}, which denotes
the \emph{decoratee}: an object to be extended,
also referred to as ``component'' by~\citet{gamma1994design}.

The \emph{decorator} \ff{circle}
has the same attributes as its decoratee \ff{center}, but also
its own attribute \ff{is-inside}. The attribute \ff{@} may be used
the same way as other attributes, including in dot notation, as it is done
at \lref{circle-phi-2}. However, this line
may be re-written in a more compact way, omitting the explicit
reference to the \ff{@} attribute, because all attributes
of the \ff{center} are present in the \ff{circle};
and omitting the reference to \ff{\$} because the default scope of visibility of
\ff{p} is the object \ff{is-inside}:

\begin{ffcode}
^.distance p
\end{ffcode}

The inner object \ff{is-inside} also has the \ff{@} attribute: it
decorates the object \ff{lte} (stands for ``less than equal'').
The expression at \lref{circle-phi-2} means:
take the parent object of \ff{is-inside},
take the attribute \ff{@} from it, then take the inner object \ff{distance}
from there, and then make a copy of it with the attribute \ff{p}
taken from the current object (denoted by the \ff{\$} symbol).

The object \ff{circle} may be used like this, to understand whether
the \((0,0)\) point is inside the circle at \((-3,9)\) with the radius \(40\):

\begin{ffcode}
circle (point -3 9) 40 > c  (*@\label{ln:circle-c}@*)
c.is-inside (point 0 0) > i
\end{ffcode}

Here, \ff{i} will be a copy of \ff{bool} behaving like \ff{TRUE}
because \ff{lte} decorates \ff{bool}.

It is also possible to make decoratee void, similar to other void
attributes, specifying it in the list of void attributes in
square brackets.

\subsection{Anonymous Formations}

A formation may be used as an argument of another object when
making a copy of it, for example:

\begin{ffcode}
(dir "/tmp").walk
  *
    [f]
      f.is-dir > @
\end{ffcode}

Here the object \ff{walk} is copied with a single argument:
the one-item tuple, which is a formation with a single void attribute \ff{f}. The \ff{walk}
will use this formation, which does not have a name,
to filter out files while traversing the directory tree. It will
make a copy of the formation and then treat it as a boolean
value to make a decision about each file.

The syntax may be simplified and the formation may be inlined
(the tuple is also inlined):

\begin{ffcode}
(dir "/tmp").walk
  * ([f] (f.is-dir > @))
\end{ffcode}

An anonymous formation may have multiple attributes:

\begin{ffcode}
[x] (x.plus 1 > succ) (x.minus 1 > prev)
\end{ffcode}

This object has two attributes \ff{succ} and \ff{prev}, and does not
have a name.

The parent of each copy of the abstract object will be set by
the object \ff{walk} and will point to the \ff{walk} object itself.

\subsection{Constants}

\eolang{} is a declarative language with lazy evaluations. This means
that this code would read the input stream two times:

\begin{ffcode}
# Just say hello.
[] > hello
  QQ.io.stdout > say
    QQ.txt.sprintf
      "The length of %s is %d"
      QQ.io.stdin.next-line > x!
      x.length
\end{ffcode}

The \ff{sprintf} object will go to the \ff{x} two times. The first time,
to use it as a substitute for \ff{\%s}, and the second time for
\ff{\%d}. There will be two round-trips to the standard input stream, which
is obviously not correct. The exclamation mark at the \ff{x!} solves the
problem, making the object by the name \ff{x} a \emph{constant}. This means
that on the first trip to \ff{x}, it turns into bytes.

Here, \ff{x} is an attribute of the object \ff{hello}, even though
it is not defined as explicitly as \ff{say}. Wherever a new
name appears after the \ff{>} symbol, it is a declaration of a new
attribute in the nearest object abstraction.

\subsection{Metas}

A program may have an optional list of \emph{meta} statements,
which are passed to the compiler as is. The meaning of them depends
on the compiler and may vary
between target platforms. This program instructs the compiler
to put all objects from the file into the package \ff{org.example}
and helps it resolve the name \ff{stdout}, which is external
to the file:

\begin{ffcode}
+package org.example
+alias org.eolang.io.stdout

# A simple app.
[args] > app
  stdout > @
    "Hello, world!\n"
\end{ffcode}

\subsection{Atoms}

Some objects in \eolang{} programs may need to be platform-specific
and cannot be composed from other existing objects---they are called
\emph{atoms}.
For example, the object \ff{app} uses the object \ff{stdout},
which is an atom. Its implementation would be provided by the
runtime. This is how the object may be defined:

\begin{ffcode}
+rt jvm org.eolang:eo-runtime:0.7.0
+rt ruby eolang:0.1.0

[text] > stdout ? (*@\label{ln:stdout}@*)
\end{ffcode}

The \ff{?} suffix informs the compiler that this object must
not be compiled from \eolang{} to the target language. The object
with this suffix already exists in the target language and most
probably could be found in the library specified by the \ff{rt}
meta. The exact library to import has to be selected by the compiler.
In the example above, there are two libraries specified: for JVM and
for Ruby.

Atoms in \eolang{} are similar to ``native'' methods in Java
and ``\nospell{extern}'' methods in C\#: this mechanism is also
known as foreign function interface (FFI).
