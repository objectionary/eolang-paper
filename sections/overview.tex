% SPDX-FileCopyrightText: Copyright (c) 2016-2025 Objectionary.com
% SPDX-License-Identifier: MIT

\eolang{}\footnote{\url{https://www.eolang.org}} was created to eliminate complexity in OOP code.
Complexity reduction is guaranteed by
\begin{inparaenum}[1)]
  \item a formal calculus behind the language
  and
  \item a reduced feature set.
\end{inparaenum}
The proposed \phic{} represents an object model through objects, data, and functions.
Operations on them are possible through formation, application, decoration, dispatch, morphing, and dataization.

\eolang{} is a dynamically and weakly typed language.
Key qualities are the following:

\begin{itemize}
  \item No classes
  \item No constructors or destructors
  \item No data primitives aside from byte sequences
  \item No imperative directives, only FFI functions
  \item No types or type annotations
  \item No reflection on objects at runtime
  \item No returning or passing NULL as an argument
  \item No changing of object structure at runtime
  \item No attribute mutability
  \item No implementation inheritance
  \item No operators
  \item No traits or mixins
  \item No varargs
  \item No arrays
  \item No method overloading
\end{itemize}

An \eolang{} compiler must be strictly and statically typed.
The inability to infer types for all objects must lead to compilation failure.
