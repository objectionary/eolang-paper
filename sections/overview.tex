% SPDX-FileCopyrightText: Copyright (c) 2016-2025 Objectionary.com
% SPDX-License-Identifier: MIT

\eolang{}\footnote{\url{https://www.eolang.org}}
was created in order to eliminate the problem of complexity of
OOP code, providing
\begin{inparaenum}[1)]
  \item a formal object calculus and
  \item a programming language with a reduced set of features.
\end{inparaenum}
The proposed \phic{} represents an object model through
data and objects, while operations with them are possible
through formation, application, decoration, dispatch, and dataization. The calculus
introduces a formal apparatus for manipulations with objects.

\eolang{}, the proposed programming language, fully implements
all elements of the calculus and enables implementation of
an object model on any computational platform.
Being an OO programming language, \eolang{} enables four key principles of OOP:
abstraction, inheritance, polymorphism, and encapsulation.
