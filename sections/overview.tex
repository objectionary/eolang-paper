% SPDX-FileCopyrightText: Copyright (c) 2016-2025 Objectionary.com
% SPDX-License-Identifier: MIT

\section{Informal Overview}\label{sec:overview}

An object \emph{formation} is a collection of \emph{attributes}, which are uniquely named pairs, for example:
\begin{phiquation}
\label{eq:price-color}
[[ price -> ?, color -> [[ D> FF-C0-CB ]] ]] {.}
\end{phiquation}
This formation has two attributes \ff{price} and \ff{color}.
The formation is an \emph{abstract} because the \ff{price} attribute is \emph{void}, i.e. there is nothing \emph{attached} to it.
The \ff{color} attribute is attached to another formation with one $D$-\emph{asset},
  which is attached to \emph{data} (three bytes).

A \emph{full application} of a pair---an attribute and an \emph{expression}---to an abstract
  \emph{results} in a new \emph{closed} formation, for example:
\begin{phiquation}
\label{eq:simple-application}
[[ |x| -> ? ]]( |y| -> b ) \trans [[ |x| -> b ]] {.}
\end{phiquation}
A \emph{partial} application results in a new abstract, for example:
\begin{phiquation*}
[[ |x| -> ?, |y| -> ? ]]( |x| -> b ) \trans [[ |x| -> b, |y| -> ? ]] {.}
\end{phiquation*}

A formation may be \emph{dispatched} from another formation with the help of \emph{dot notation}
  where the right side \emph{accesses} the left side, for example:
\begin{phiquation}
\label{eq:dot-notation}
[[ |x| -> \phiOver{$.|p|(|t| -> b)}.|y|, |p| -> [[ |t| -> ?, |y| -> $.|t| ]] ]].|x| \trans b {.}
\end{phiquation}
The leftmost symbol \phiTerminal{\(\xi\)} denotes the \emph{scope} of the formation,
  which is the formation itself;
  the following $.|p|$ part retrieves the formation attached to the attribute \(\ff{p}\);
  then, the application $(|t|->b)$ makes a copy of the formation;
  finally, the $.|y|$ part retrieves formation \(b\).

Attributes are \emph{immutable}, i.e. an application to a formation,
  where the attribute is already attached,
  results in a \emph{terminator} denoted as $T$ (runtime error), for example:
\begin{phiquation*}
[[ |x| -> b_1 ]]( |x| -> b_2 ) \trans T {.}
\end{phiquation*}

An expression may be textually reduced, for example:
\begin{phiquation*}
[[ |x| -> ?, |y| -> ? ]]( |x| -> b_1 )( |y| -> b_2 ) \trans
  \trans [[ |x| -> b_1, |y| -> ? ]] ( |y| -> b_2 ) \trans
  \trans [[ |x| -> b_1, |y| -> b_2 ]] {.}
\end{phiquation*}
The formation on the left is reduced to the formation on the right in two reduction steps.
Some expressions may be in \emph{normal form}, which means no further applicable reductions.

A formation is called a \emph{decorator} if it has the $@$-attribute with an expression attached to it,
  known as a \emph{decoratee}.
An attribute dispatched from a decorator reduces to the same attribute dispatched from the decoratee,
  if the attribute is not present in the decorator, for example:
\begin{phiquation*}
[[ @ -> [[ |x| -> b ]] ]].|x| \trans b {.}
\end{phiquation*}

A formation may have data attached only to its $D$-asset, for example:
\begin{phiquation*}
[[ |x| -> [[ D> 00-2A ]] ]] {.}
\end{phiquation*}

A formation may have a \emph{function} attached to its $L$-asset.
Such a formation is referred to as an \emph{atom}.
An atom may be \emph{morphed} to another formation by evaluating its function,
for example:
\begin{phiquation}
\label{eq:Sqrt}
\frac \
  { \vdash |Sqrt| {(} b {,}\; u {,}\; s {)} \to << [[ D> \sqrt{\mathbb{D} {(} b.|x| {,}\; u {,}\; s {)} } ]], s >> } \
  {[[ L> |Sqrt|, |x| -> 256 ]] \to 16} {.}
\end{phiquation}
There is also a \emph{dataization} function \(\mathbb{D}(b, u, s)\)
  that normalizes its first argument and then either returns the data
  attached to the $D$-asset of the normal form or morphs the atom into a formation and dataizes it (recursively),
  for example:
\begin{phiquation*}
\mathbb{D} {(} [[ |x| -> 42 ]].|x| \trans 42 {,}\; u {,}\; s {)} \to 42.
\end{phiquation*}

A program is a formation, called \emph{Universe}, that is attached to the $Q$ attribute of nowhere:
\begin{phiquation*}
Q -> [[ D> CA-FE ]].
\end{phiquation*}
The dataization of universe is the evaluation of the program.
