% SPDX-FileCopyrightText: Copyright (c) 2016-2025 Objectionary.com
% SPDX-License-Identifier: MIT

\section{Informal Overview}

An \emph{object} is a collection of \emph{attributes}, which are uniquely named pairs, for example:
\begin{phiquation}
\label{eq:price-color}
[[ price -> ?, color -> [[ D> FF-C0-CB ]] ]].
\end{phiquation}
This is a \emph{formation} of an object with two attributes \ff{price} and \ff{color}.
The object is \emph{abstract} because the \ff{price} attribute is \emph{void}, i.e. there is no object \emph{attached} to it.
The \ff{color} attribute is attached to another object formation with one \phiTerminal{\(\Delta\)}-\emph{asset},
  which is attached to \emph{data} (three bytes).
Objects do not have names; only attributes, to which objects are attached, do.

A \emph{full application} of an abstract object to a pair of an attribute and an object \emph{results} in a new \emph{closed} object, for example:
\begin{phiquation}
\label{eq:simple-application}
[[ |x| -> ? ]]( |y| -> b_2 ) \trans [[ |x| -> b_2 ]].
\end{phiquation}
A \emph{partial} application results in a new abstract object, for example:
\begin{phiquation*}
[[ |x| -> ?, |y| -> ? ]]( |x| -> b ) \trans [[ |x| -> b, |y| -> ? ]].
\end{phiquation*}

An object may be \emph{dispatched} from another object with the help of \emph{dot notation}
  where the right side \emph{accesses} the left side (underlined), for example:
\begin{phiquation}
\label{eq:dot-notation}
[[ |x| -> \underline{\xi.|p|(|t| -> b)}.|y|, |p| -> [[ |t| -> ?, |y| -> \xi.|t| ]] ]].|x| \trans b.
\end{phiquation}
The leftmost symbol \phiTerminal{\(\xi\)} denotes the \emph{scope} of the object, which is the object formation itself;
  the following \(.\ff{p}\) part retrieves the object attached to the attribute \(\ff{p}\);
  then, the application $(|t|->b)$ makes a copy of the object;
  finally, the \(.\ff{y}\) part retrieves object \(b\).

Each object has a \emph{parent} attribute $^$, which is attached to the
formation from which the object was dispatched.

Attributes are \emph{immutable}, i.e. an application of a binding to an object,
  where the attribute is already attached,
  results in a \emph{terminator} denoted as $T$ (runtime error), for example:
\begin{phiquation*}
[[ |x| -> b_1 ]]( |x| -> b_2 ) \trans T.
\end{phiquation*}

An object may be textually reduced, for example:
\begin{phiquation*}
[[ |x| -> ?, |y| -> ? ]]( |x| -> b_1 )( |y| -> b_2 ) \trans
  \trans [[ |x| -> b_1, |y| -> ? ]] ( |y| -> b_2 ) \trans
  \trans [[ |x| -> b_1, |y| -> b_2 ]].
\end{phiquation*}
The object on the left is reduced to the object on the right in two reduction steps.
Every object has a \emph{normal form}, which is a form that
has no further applicable reductions.

An object is called a \emph{decorator} if it has the $@$-attribute
with an object attached to it, known as a \emph{decoratee}. An attribute
dispatched from a decorator reduces to the same attribute dispatched from the decoratee,
if the attribute is not present in the decorator, for example:
\begin{phiquation*}
[[ @ -> [[ |x| -> b ]] ]].|x| \trans b.
\end{phiquation*}

An object may have data attached only to its \phiTerminal{\(\Delta\)}-asset, for example:
\begin{phiquation*}
[[ |x| -> [[ D> 00-2A ]] ]].
\end{phiquation*}

An object may have a \emph{function} attached to its \phiTerminal{\(\lambda\)}-asset.
Such an object is referred to as an \emph{atom}.
An atom may be \emph{morphed} to another object by evaluating
its function with \begin{inparaenum}[1)]
    \item the object and
    \item the state of evaluation
\end{inparaenum}
as arguments, for example:
\begin{phiquation}
\label{eq:Sqrt}
\frac \
  { \vdash |Sqrt| (b, s) \to \langle [[ D> \sqrt{\mathbb{D}(b.|x|, u)} ]], s \rangle } \
  {[[ L> |Sqrt|, |x| -> 256 ]] \to 16}.
\end{phiquation}
There is also a \emph{dataization} function \(\mathbb{D}(b, u, s)\)
  that normalizes its first argument and then either returns the data
  attached to the \phiTerminal{\(\Delta\)}-asset of the normal form or morphs the atom into an object and dataizes it (recursively),
  for example:
\begin{phiquation*}
\mathbb{D}([[ |x| -> 42 ]].|x| \trans 42, u, s) \to 42.
\end{phiquation*}

A program is an object that is attached to the $Q$ attribute of the \emph{Universe}, for example:
\begin{phiquation*}
Q -> [[ D> CA-FE ]].
\end{phiquation*}
The dataization of $Q$ is the evaluation of the program.
