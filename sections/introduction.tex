% SPDX-FileCopyrightText: Copyright (c) 2016-2025 Objectionary.com
% SPDX-License-Identifier: MIT

\section{Introduction}

Object-oriented programming has become the dominant paradigm for software development,
  with languages such as Java, C++, Python, and C\# being among the most widely used in industry.
Despite this prevalence, the theoretical foundations of OOP remain less developed than those of functional programming,
  which has long benefited from the \(\lambda\)-calculus~\citep{church1936some,barendregt2012} as a formal basis.
The absence of a comparable formalism for OOP hinders rigorous reasoning about object-oriented programs,
  complicates the development of verified compilers and static analyzers,
  and limits the potential for principled language design.

Several attempts have been made to formalize object-oriented concepts.
\citet{abadi1995imperative} proposed an imperative object calculus,
  while \citet{igarashi2001featherweight} introduced Featherweight Java as a minimal core calculus.
However, these formalisms either focus on specific language features
  or omit constructs commonly found in modern OO languages.
As \citet{nierstrasz1991towards} observed, the development of object-based programming languages
  has suffered from the lack of any generally accepted formal foundations.

This paper presents \phic{}, a calculus designed to serve as a formal foundation for object-oriented programming.
The calculus is built around a single primitive---the object formation---which
  encapsulates attributes, data, and functions.
Objects interact through application, which attaches expressions to attributes,
  and dispatch, which retrieves attached expressions.

The contributions of this paper are as follows:
\begin{itemize}
  \item A formal syntax for \phic{} defined by a context-free grammar (\cref{sec:syntax}).
  \item Rigorous definitions of the core constructs (\cref{sec:foundations}).
  \item A family of semantic operators (\cref{sec:operators}).
\end{itemize}

The remainder of this paper is organized as follows.
\Cref{sec:overview} provides an informal overview of the calculus.
\Cref{sec:syntax} presents the syntax as a context-free grammar.
\Cref{sec:foundations} defines the foundational concepts.
\Cref{sec:operators} introduces the semantic operators.
\Cref{sec:future} suggests directions of future studies.
Finally, \cref{sec:related} discusses related work.
