% SPDX-FileCopyrightText: Copyright (c) 2016-2025 Objectionary.com
% SPDX-License-Identifier: MIT

The proposed \phic{} is based on set theory~\citep{jech2013set} and lambda calculus,
representing objects as sets of pairs and their internals as \(\lambda\)-terms.
\Cref{fig:ebnf} presents formal syntax of \phic{}.

\begin{figure*}
\begin{mdframed}
\raggedright
\begin{ebnf}[8em]
<Program> := "\(\Phi\)" "\(\mapsto\)" <Object> \\
<Object> := <Formation> | <Application> | <Dispatch> | "\(\Phi\)" | "\(\xi\)" | "\(\perp\)" \\
<Formation> := "\(\llbracket\)" <Bindings> "\(\rrbracket\)" \\
<Application> := <Object> "\(\lparen\)" <Bindings> "\(\rparen\)" \\
<Dispatch> := <Object> "." <Attribute> \\
<Bindings> := <Bindings> "," <Binding> | \(\epsilon\) \\
<Binding> := <\(\tau\)-Binding> | <\(\Delta\)-Binding> | <\(\lambda\)-Binding> \\
<\(\tau\)-Binding> := <Attribute> "\(\mapsto\)" ( <Object> | "\(\varnothing\)" ) \\
<\(\Delta\)-Binding> := "\(\Delta\)" "\(\phiDotted\)" ( <Data> | "\(\varnothing\)" ) \\
<\(\lambda\)-Binding> := "\(\lambda\)" "\(\phiDotted\)" <Function> \\
\end{ebnf}
\end{mdframed}
\caption{Syntax as a context-free grammar, in EBNF}
\label{fig:ebnf}
\end{figure*}

Besides the literals mentioned in the grammar in blue color, the alphabet includes three non-terminals that rewrite to terminals as such:
\begin{itemize}
\item \EbnfNonTerminal{Attribute}: either \begin{inparaenum}[1)]
    \item one of two Greek letters \(\varphi\) and \(\rho\),
    \item a Greek letter \(\alpha\) with an optional index,
    or
    \item a string of English letters starting with a lowercase letter, e.g. ``\ff{price}'' or ``\ff{color}'';
\end{inparaenum}
\item \EbnfNonTerminal{Data}: a sequence of bytes in hexadecimal format, e.g. ``\texttt{EF-41-5C}'' is a sequence of three bytes, ``\texttt{42-}'' is a one-byte sequence (with a tailing dash in order to avoid confusion with integers), and ``\texttt{-{}-}'' (double dash) is an empty sequence of bytes;
\item \EbnfNonTerminal{Function}: a string of English letters where the first letter is uppercase, e.g. ``\ff{Sqrt}''.
\end{itemize}

\subsection{Bindings}

\begin{definition}[Binding]
\textbf{Binding}, ranging by \(B\), is a possibly empty map of key-value pairs, where all keys are unique, denoted as $k_1 -> v_1, k_2 -> v_2, \dots, k_n -> v_n$.
\end{definition}

The following operations can be performed on a binding:
\begin{itemize}
\item The statement \(k \in B\) evaluates to true if the key \(k\) is present within the map;
% \item The operation \(B-k\) results in a new binding that excludes the key-value pair with the key \(k\) (if such a pair exists);
\item The expression \(B[k]\) returns the value associated with the key \(k\).
\end{itemize}

\subsection{Attributes}

\begin{definition}[Attribute]
\textbf{Attribute}, ranged over \(\mathcal{T}\) by \(\tau_i\), is an identifier that is either
\begin{inparaenum}[1)]
    \item a text string that starts with a lower-case English letter,
    \item \(\alpha\) with an index,
    or
    \item one of the following two Greek letters:
    \(\varphi\),
    and
    \(\rho\).
\end{inparaenum}
\end{definition}

\subsection{Asset}

\begin{definition}[Asset]
\textbf{Asset} is either \(\Delta\) or \(\lambda\) Greek letter.
\end{definition}

\subsection{Data}

\begin{definition}[Data]
\textbf{Data} is a possibly empty sequence of 8-bit bytes, ranged over \(\mathcal{D}\) by \(\delta_i\).
\end{definition}

\subsection{Objects}

\begin{definition}[Object]
\textbf{Object} is either an terminator, denoted as \(\perp\), or a binding where keys are attributes or assets and the following conditions hold:
\begin{inparaenum}[1)]
\item if a key is \(\Delta\), then the value is a data or \(\varnothing\),
\item if a key is \(\lambda\), then the value is a function,
\item otherwise, the value is either an object or \(\varnothing\).
\end{inparaenum}
Objects range over \(\mathcal{B}\) by \(b_i\).
\end{definition}

The following expression is an example of an object:
\begin{phiquation}
\label{eq:object-example}
b_1 = [[ D> 42, 0-> b_2(|foo| -> b_3).|bar|, 1-> [[ L> Sqrt ]], 5-> \perp ]]
\end{phiquation}

The arrow ``\(\mapsto\)'' denotes an attachment of an object (right-hand side) to the attributes (left-hand side). The arrow ``\(\phiDotted\)'' denotes an attachment of a data or a function to either \(\Delta\) or \(\lambda\) assets respectively.

\begin{definition}[Domain]
\textbf{Domain} of an object is a set of all attributes of the object.
\end{definition}

The domain of the object of \cref{eq:object-example} is the following set:
\(\{ \alpha_0, \alpha_1, \alpha_5 \}\). The asset \(\Delta\) does not belong to the domain.

\begin{definition}[Abstract Object]
An object is \textbf{abstract} if at least one of its attribute is attached to \(\varnothing\), otherwise the object is \textbf{closed}.
\end{definition}

The object of \cref{eq:object-example} is abstract, because its attribute \ff{price} is attached to \(\varnothing\). The object of \cref{eq:simple-application} was abstract before the application, but the object that it was reduced to by the application, is a closed object since its attribute \(\alpha_1\) is attached to \(b_2\).

Even though \(\varnothing\) may be attached to an attribute of an object, it is not an object by itself. Instead, \(\varnothing\) is a ``placeholder'' for an object, which is attached to an attribute until an object is attached to it.

\subsection{Formation}

\begin{definition}[Formation]
Object \textbf{formation}, denoted as $[[B]]$, is a construction of a new object.
\end{definition}

We introduce the term ``object formation'' rather than using a more convenient
term ``construction'' because the latter generally implies a presence of a
class from which an object is being constructed or instantiated. Instead,
object formation is closer to the creation of a prototype, which may either be
used ``as is'' or copied.

\Cref{eq:price-color} is an example of object formation, where the binding of
the abstract object being formed consists of two pairs: $price -> ?$ and
$color -> [[D>"red"]]$. The object attached to the \ff{color} attribute is a
formation of a closed object.

\subsection{Application}

\begin{definition}[Application]
Object \textbf{application}, denoted as $b(B)$, is a copy of an existing abstract object \(b\), with new values attached to some of its attributes and/or assets.
\end{definition}

\Cref{eq:simple-application} demonstrates object application, where the binding $( 1-> b_2 )$ is applied to the formation of an abstract object $[[ 1-> ?]]$. The application creates a new object $[[ 1-> b_2 ]]$, while the existing abstract object remains intact.

\subsection{Forma}

\begin{definition}[Forma]
A \textbf{forma} of an object \(b\), denoted as \(\mathbb{F}(b)\), is the abstract object that was copied in order to create \(b\). A forma of a formation is the formation itself.
\end{definition}

In \cref{eq:simple-application}, the forma of $[[ 1-> b_2 ]]$ is the abstract object $[[ 1->? ]]$, while the forma of $[[ 1->? ]]$ is $[[ 1->? ]]$.

\subsection{Programs}

\begin{definition}[Program]
\textbf{Program} is an object attached to \(\Phi\) attribute of \emph{Universe}.
\end{definition}

The Universe resembles an object with the only attribute \(\Phi\), but it is not an object since it may not be attached to any attribute of any other object.

\subsection{Scope}

\begin{definition}[Scope]
Object \textbf{scope} is either the formation where the object is attached to an attribute, or the scope of the application where the object is used in a binding.
\end{definition}

In simpler terms, scope is the formation that is the ``closest'' to the object, moving to the left. In the following object formation, the scope of \(b_1\) is the formation where the \ff{source} attribute stays, while the scope of \(b_2\) is the formation where the \ff{ref} attribute stays:
\begin{phiquation}
    [[ ref -> [[ source -> Q.book( author -> b_1 ) ]], cite -> b_2 ]]
\end{phiquation}

\subsection{Dispatch}

\begin{definition}[Dispatch]
Object \textbf{dispatch} (also known as ``dot notation''), denoted as $x.\tau_1$, is a retrieval of an object attached to attribute \(\tau_1\) of \(x\), where \(x\) (the ``subject'') may be an object, \(\Phi\), or \(\xi\); and then attaching its \(\rho\) attribute to the subject, if it is not attached yet.
\end{definition}

The subject \(\Phi\) means the program.

The subject \(\xi\) means the scope of the dispatch.

\Cref{eq:dot-notation} demonstrates how dot notation works in a combination with object application. First, the scope \(\xi\) denotes the object formation, where \(x\) stays. Then, the \(.\ff{p}\) retrieves the object attached to \(\ff{p}\) attribute---this object is the formation where \(|t|\) stays. Then, the application of the $|t|->b$ binding makes a copy of the formation and attaches \(b\) to \(|t|\). Then, \(.|y|\) retrieves the object attached to \(|y|\) attribute of the copy: it is \(\xi.|t|\), where \(\xi\) means the nearest object formation, which is the copy itself. Finally, the dispatch retrieves what is the attached to \(b\) of the copy, which is \(b\).

\subsection{Terminator}

\begin{definition}[Terminator]
The \textbf{terminator} object, denoted as \(\perp\), equals to itself when being a subject of a dispatch or an application:
\begin{equation*}
\forall \tau : \perp.\tau \leadsto \perp \qquad \forall B : \perp(B) \leadsto \perp.
\end{equation*}
\end{definition}

\subsection{Immutability}

\begin{definition}[Immutability]
Every object is \textbf{immutable}, meaning that an application of its already attached attribute or asset equals to \(\perp\):
\begin{equation*}
\forall \tau, b_1, b_2 : \llbracket \tau \mapsto b_1 \rrbracket ( \tau \mapsto b_2 ) \leadsto \perp
\end{equation*}
\end{definition}

\subsection{Atoms}

\begin{definition}[Atom]
\textbf{Atom} is an object formation with a total function \(\langle \mathcal{B}, \mathcal{S} \rangle \to \langle \mathcal{B}, \mathcal{S} \rangle\) attached to its \(\lambda\)-asset, that maps objects to objects, possibly modifying the \emph{state} of evaluation \(s\), which ranges over \(\mathcal{S}\).
\end{definition}

\Cref{eq:Sqrt} demonstrates an atom with a function that calculates a square root of a number, which it retrieves from the \(\Delta\)-asset of \(b.\alpha_0\), with the help of dataization function. The implementation of functions is outside of the scope of \phic{}: they may be implemented, for example, in \(\lambda\)-calculus or some programming language, for example, Java or C++.

\subsection{Decoration}

\begin{definition}[Decoration]
Object \textbf{decoration} is a mechanism of extending an object (``decoratee'') by attaching it to the attribute \(\varphi\) of another object (``decorator''), which makes attributes of the decoratee retrievable from the decorator, unless the decorator has its own attributes with the same names.
\end{definition}

\subsection{Parent}

\begin{definition}[Parent]
Attaching object \(b_1\) to the \(\rho\) attribute of object \(b_2\) means setting the \textbf{parent} of \(b_2\) to \(b_1\).
\end{definition}
