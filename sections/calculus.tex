% SPDX-FileCopyrightText: Copyright (c) 2016-2025 Objectionary.com
% SPDX-License-Identifier: MIT

In this section we introduce \phic{}, a formalism that we use later to optimize object-oriented code.
We give an informal introduction to the calculus, helping the reader to grasp the idea without diving into details.
In \cref{app:syntax}, we formulate the syntax using EBNF and introduce individual concepts of the calculus, explaining their semantics and illustrating them with examples.

An \emph{object} is a collection of \emph{attributes}, which are uniquely named pairs, for example:
\begin{phiquation}
\label{eq:price-color}
[[ price -> ?, color -> [[ D> FF-C0-CB ]] ]].
\end{phiquation}
This is a \emph{formation} of an object with two attributes |price| and |color|.
The object is \emph{abstract} because the |price| attribute is \emph{void}, i.e. there is no object \emph{attached} to it.
The |color| attribute is attached to another object formation with one \(\Delta\)-\emph{asset}, which is attached to \emph{data} (three bytes).
Objects do not have \emph{names}, while only the attributes, which objects are attached to, do.

A \emph{full application} of an abstract object to a pair of attribute and object \emph{results} in a new \emph{closed} object, for example:
\begin{phiquation}
\label{eq:simple-application}
[[ |x| -> ? ]]( |y| -> b_2 ) \trans [[ |x| -> b_2 ]].
\end{phiquation}
A \emph{partial} application results in a new abstract object, for example:
\begin{phiquation*}
[[ |x| -> ?, |y| -> ? ]]( |x| -> b ) \trans [[ |x| -> b, |y| -> ? ]].
\end{phiquation*}

An object may be \emph{dispatched} from another object with the help of \emph{dot notation} where the right side \emph{accesses} the left side (underlined), for example:
\begin{phiquation}
\label{eq:dot-notation}
[[ |x| -> \underline{\xi.|p|(|t| -> b)}.|y|, |p| -> [[ |t| -> ?, |y| -> \xi.|t| ]] ]].|x| \trans b.
\end{phiquation}
The leftmost symbol \(\xi\) denotes the \emph{scope} of the object,
which is the object formation itself; the following \(.|p|\) part
retrieves the object attached to the attribute \(|p|\);
then, the application $(|t|->b)$ makes a copy of the object;
finally, the \(.|y|\) part retrieves object \(b\).

Each object has a \emph{parent} attribute \(\rho\), which is attached to the
formation from where the object was dispatched.

Attributes are \emph{immutable}, i.e. an application of a binding to an object, where the attribute is already attached, results in a \emph{terminator} denoted as \(\dead\) (runtime error), for example:
\begin{phiquation*}
[[ |x| -> b_1 ]]( |x| -> b_2 ) \trans \dead{}.
\end{phiquation*}

An object may be textually reduced, for example:
\begin{phiquation*}
[[ |x| -> ?, |y| -> ? ]]( |x| -> b_1 )( |y| -> b_2 ) \trans
  \trans [[ |x| -> b_1, |y| -> ? ]] ( |y| -> b_2 ) \trans
  \trans [[ |x| -> b_1, |y| -> b_2 ]].
\end{phiquation*}
The object on the left is reduced to the object on the right in two reduction steps.
Every object has a \emph{normal form}, which is a form that
has no more possible applicable reductions.

An object is called a \emph{decorator} if it has the \(\varphi\) attribute
with an object attached to it, known as a \emph{decoratee}. An attribute
dispatched from a decorator reduces to the same attribute dispatched from the decoratee,
if the attribute is not present in the decorator, for example:
\begin{phiquation*}
[[ @ -> [[ |x| -> b ]] ]].|x| \trans b.
\end{phiquation*}

An object may have data attached only to its \(\Delta\)-asset, for example:
\begin{phiquation*}
[[ |x| -> [[ D> 00-2A ]] ]].
\end{phiquation*}

An object may have a \emph{function} attached to its \(\lambda\)-asset.
Such an object is referred to as \emph{atom}.
An atom may be \emph{morphed} to another object by evaluating
its function with \begin{inparaenum}[1)]
    \item the object and
    \item the state of evaluation
\end{inparaenum}
as arguments, for example:
\begin{phiquation}
\label{eq:Sqrt}
\frac \
  { \vdash Sqrt (b, s) \to \langle [[ D> \sqrt{\mathbb{D}(b.|x|)} ]], s \rangle } \
  {[[ L> Sqrt, |x| -> 256 ]] \to 16}.
\end{phiquation}
There is also a \emph{dataization} function \(\mathbb{D}(b, s)\) that normalizes its first argument
and then either returns the data attached to the \(\Delta\)-asset of
the normal form or morphs the atom into an object and dataizes it (recursively),
for example:
\begin{phiquation*}
\mathbb{D}([[ @ -> 42 ]].|x| \trans [[ |x| -> 42 ]].|x| \trans 42, s) \to 42.
\end{phiquation*}

A program is an object that is attached to the \(\Phi\) attribute
of the \emph{Universe}, for example:
\begin{phiquation*}
Q -> [[ D> CA-FE ]].
\end{phiquation*}
The dataization of \(\Phi\) is the evaluation of the program.

\subsection{Syntax}\label{sec:syntax}

The syntax of a program is defined by EBNF in \cref{fig:ebnf} (the starting symbol is \EbnfNonTerminal{Program}).

\begin{figure*}
\begin{mdframed}
\raggedright
\begin{ebnf}[8em]
<Program> := "\(\Phi\)" "\(\mapsto\)" <Expression> \\
<Expression> := <Formation> | <Application> | <Dispatch> | "\(\dead\)" \\
<Formation> := "\(\llbracket\)" <Binding> "\(\rrbracket\)" \\
<Application> := <Expression> "\(\lparen\)" <A-Pair> "\(\rparen\)" \\
<A-Pair> := <\(\tau\)-Pair> | <\(\alpha\)-Pair> \\
<Dispatch> := <Subject> "." <Attribute> \\
<Subject> := <Expression> | <Locator> \\
<Locator> := "\(\Phi\)" | "\(\xi\)" \\
<Binding> := <Pair> <Bindings> | \(\epsilon\) \\
<Bindings> := "," <Pair> <Bindings> | \(\epsilon\) \\
<Pair> := <\(\varnothing\)-Pair> | <\(\tau\)-Pair> | <\(\Delta\)-Pair> | <\(\lambda\)-Pair> \\
<\(\varnothing\)-Pair> := <Attribute> "\(\mapsto\)" "\(\varnothing\)" \\
<\(\tau\)-Pair> := <Attribute> "\(\mapsto\)" <Subject> \\
<\(\alpha\)-Pair> := <Alpha> "\(\mapsto\)" <Subject> \\
<\(\Delta\)-Pair> := "\(\Delta\)" "\(\phiDotted\)" <Data> \\
<\(\lambda\)-Pair> := "\(\lambda\)" "\(\phiDotted\)" <Function> \\
\end{ebnf}
\end{mdframed}
\caption{Syntax as a context-free grammar, in EBNF.}
\label{fig:ebnf}
\end{figure*}

Besides the literals mentioned in the grammar in blue color, the
alphabet includes three non-terminals that rewrite to terminals as follows:
\begin{itemize}
  \item \EbnfNonTerminal{Attribute}: either \begin{inparaenum}[1)]
      \item Greek letter \(\varphi\),
      \item Greek letter \(\rho\),
      or
      \item a string of lowercase English letters possibly with dashes inside, e.g. ``|price|'' or ``|a-car|'';
  \end{inparaenum}
  \item \EbnfNonTerminal{Data}: a sequence of bytes in hexadecimal format, e.g. ``\texttt{EF-41-5C}'' is a sequence of three bytes, ``\texttt{42-}'' is a one-byte sequence (with a trailing dash in order to avoid confusion with integers), and ``\texttt{-{}-}'' (double dash) is an empty sequence of bytes;
  \item \EbnfNonTerminal{Function}: a string of English letters where the first letter is in uppercase, e.g. ``|Sqrt|'';
  \item \EbnfNonTerminal{Alpha}: a Greek letter \(\alpha\) with a non-negative whole-number index, e.g. \(\alpha_2\).
\end{itemize}

\subsection{Expression}\label{sec:expression}

\begin{definition}[Expression]
\textbf{Expression}, ranged over \(\mathcal{E}\) by \(e_i\),
is a grammatical construct that may result in an object after
certain transformations.
\end{definition}

\subsection{Attribute}\label{sec:attribute}

\begin{definition}[Attribute]
\textbf{Attribute}, ranged over \(\mathcal{T}\) by \(\tau_i\),
is an identifier to which either ``\stx{\varnothing}'' or an expression
may be attached.
\end{definition}

\begin{definition}[Void vs. Attached]
An attribute is \textbf{void} if it is attached to ``\stx{\varnothing},''
otherwise it is an \textbf{attached} attribute.
\end{definition}

\subsection{Data}\label{sec:data}

\begin{definition}[Data]
\textbf{Data}, ranged over \(\mathcal{D}\) by \(\delta_i\), is a possibly
empty sequence of 8-bit bytes.
\end{definition}

\subsection{Binding}\label{sec:binding}

\begin{definition}[Binding]
\textbf{Binding}, ranged over \(\mathcal{G}\) by \(B\), is a possibly empty set of key-value pairs,
denoted as \( k_1 \to v_1, k_2 \to v_2, \dots, k_n \to v_n \), where all keys are unique.
\end{definition}

The predicate \(k \in B\) holds if \(k\) is present in any pair of \(B\).

\subsection{Function}\label{sec:function}

\begin{definition}[Function]
\textbf{Function} is a total mapping
\(\langle \mathcal{G}, \mathcal{S} \rangle \to \langle \mathcal{G}, \mathcal{S} \rangle\)
that maps binding to binding, possibly modifying the \emph{state} of evaluation \(s_i\),
which itself ranges over \(\mathcal{S}\).
\end{definition}

\subsection{Asset}\label{sec:asset}

\begin{definition}[Asset]
\textbf{Asset} is an identifier to which either data
(denoted as ``\stx{\Delta}''-asset) or function
(denoted as ``\stx{\lambda}''-asset) is attached.
\end{definition}

\subsection{Object}\label{sec:object}

\begin{definition}[Object]
\textbf{Object}, ranged over \(\mathcal{B}\) by \(b_i\), is a binding
where keys are either attributes or assets and values are either
``\stx{\varnothing},'' expressions, data, or functions.
\end{definition}

The following is an example of an object with four pairs, where the first one
is an asset attached to data, while the other three are attributes attached to
expressions:
\begin{phiquation}
\label{eq:object-example}
[[ D> 00-2A, |a| -> b_2(0-> b_3).|bar|, |b| -> [[ L> Sqrt ]], foo -> \dead ]]
\end{phiquation}

The arrow ``\stx{\mapsto}'' denotes an attachment of an expression (right-hand side)
to an attribute (left-hand side). The arrow ``\stx{\phiDotted}'' denotes
an attachment of data or function to an asset.

\subsection{Domain}\label{sec:domain}

\begin{definition}[Domain]
\textbf{Domain} of object \(b\), denoted as \(\bar{b}\), is a set
that includes all attributes of \(b\).
\end{definition}

The domain of the object in \cref{eq:object-example} is \(\{ |a|, |b|, |foo| \}\).
Assets ``\stx{\Delta}'' and ``\stx{\lambda}'' do not belong to object domain.

\subsection{Formation}\label{sec:formation}

\begin{definition}[Formation]
Object \textbf{formation}, denoted as ``\(\stx{\llbracket} B \stx{\rrbracket}\)'',
is a construction of a new object.
\end{definition}

We introduce the term ``object formation'' rather than using a more traditional
``construction'' term because the latter generally implies a presence of a
class from which an object is being constructed or instantiated. Instead,
object formation is closer to the creation of a prototype, which may either be
used ``as is'' or copied.

\subsection{Program}\label{sec:program}

\begin{definition}[Program]
\textbf{Program} is a pair $ Q -> [[ B ]] $ that exists only
in the \textit{Universe} binding.
\end{definition}

The Universe resembles an object with only the attribute \(\Phi\), but it
is not an object since it cannot be attached to any attribute of any other object.

\subsection{Abstract Object}\label{sec:abstract}

\begin{definition}[Abstract Object]
An object is \textbf{abstract} if at least one of its attributes is void,
otherwise the object is \textbf{closed}.
\end{definition}

\Cref{eq:price-color} is an example of an object formation, where the binding of
the abstract object being formed consists of two pairs: $price -> ?$ and
$color -> [[ D> FF-C0-CB ]]$. The object attached to the |color| attribute is a
formation of a closed object.

\subsection{Ordinal}\label{sec:ordinal}

\newcommand\ordinal[2]{#1 \circ #2}
\begin{definition}[Ordinal]
An attribute's \textbf{ordinal}, denoted as \(\ordinal{\tau}{b}\),
is a non-negative whole number that is equal to the position of \(\tau\)
in \(\mathbb{F}(b)\), starting from zero, not counting assets.
\end{definition}

\begin{example}
\Cref{tab:ordinals} shows a few examples of attributes and their ordinals.

\begin{table*}
\caption{A few examples of attributes' ordinals in their objects.}
\label{tab:ordinals}
\begin{tabular}{lr}
\toprule
\(b\) & \(\ordinal{|x|}{b}\) \\
\midrule
$[[ |x| -> \xi.|k| ]]$
  & 0 \\
$[[ |foo| -> \xi.|k|, |x| -> \Phi.|t| ]]$
  & 1 \\
$[[ L> Fn, |x| -> \xi.|k| ]]$
  & 0 \\
$[[ D> CA-FE, |foo| -> \xi, |x| -> \xi.|foo| ]]$
  & 2 \\
\bottomrule
\end{tabular}
\end{table*}
\end{example}

\subsection{Application}\label{sec:application}

\begin{definition}[Application]
Object \textbf{application}, denoted as \( b \stx{(} \tau \;\stx{\mapsto}\; e \stx{)} \), is a copy of an existing abstract object \(b\) (the ``subject''), with \(e\) attached to its \(\tau\) attribute.
\end{definition}

\Cref{eq:simple-application} demonstrates object application, where $ |a| -> b_2 $ is
applied to the formation of an abstract object $[[ |a| -> ? ]]$. The application creates
a new object $[[ |a| -> b_2 ]]$, while the existing abstract object remains intact.

The object in \cref{eq:price-color} is abstract, because its attribute |price| is void.
The object in \cref{eq:simple-application} was abstract before the application, but the object created by the application is closed since its attribute |a| is not void (attached to \(b_2\)).

Even though \stx{\varnothing} may be attached to an attribute of an object,
it is not an object by itself. Instead, \(\varnothing\) is a ``placeholder''
for an object, which remains attached to an attribute until an object is attached to it.
Even though this mechanism resembles NULL references, there is a significant
difference: in \phic{}, void attributes may be attached to objects only once,
while any further reattachments are prohibited.

In $b(\alpha_i -> e)$ application, \(e\) must be attached to the attribute \(\tau\) of \(b\) for which \(\ordinal{\tau}{\mathbb{F}(b)}\) equals \(i\).

\subsection{Forma}\label{sec:forma}

\begin{definition}[Forma]
A \textbf{forma} of an object \(b\), denoted as \(\mathbb{F}(b)\), is an abstract
object that was copied in order to create \(b\).
A forma of a formation is the formation itself.
\end{definition}

In \cref{eq:simple-application}, the forma of $[[ |a| -> b_2 ]]$ is the abstract
object $[[ |a| -> ? ]]$, while the forma of $[[ |a| -> ? ]]$ is itself.

\subsection{Dispatch}\label{sec:dispatch}

\begin{definition}[Dispatch]
Object \textbf{dispatch} (also known as ``dot notation''), denoted as
\( b \stx{.} \tau \) where \(b\) is the ``subject'', means retrieval of the object
attached to the \(\tau\) attribute of \(b\).
\end{definition}

\subsection{Scope}\label{sec:scope}

\begin{definition}[Scope]
The \textbf{scope} of expression \(e\), denoted as \(e^\varsigma\) is either
\begin{inparaenum}[a)]
\item the formation where the expression is attached to an attribute,
\item or the scope of the expression where \(e\) is used.
\end{inparaenum}
\end{definition}

In simpler terms, the scope is the formation that is the ``closest'' to the pair,
moving to the left in the expression. In~\cref{eq:simple-scope}, the scope of |author| is the
formation where the |source| attribute stays, while the scope of |cite|
is the formation where the |ref| attribute stays.
\begin{phiquation}
\label{eq:simple-scope}
  [[ ref -> \uplace{}{ [[ source -> Q.book( author -> b_1 ) ]] }, cite -> b_2 ]]
\end{phiquation}

\Cref{fig:scopes} illustrates the concept of scope of a pair.

\begin{figure*}
\begin{mdframed}
\begin{phiquation*}
[[ |a| -> \uplace{}{ [[ |y| -> b_2.|t|( |f| -> [[ |z| -> b_3 ]] ( |x| -> e ) ) ]] } ]]

[[ |f| -> \uplace{}{ [[ |k| -> b_1( |x| -> e ) ]] } ]]

[[ |k| -> [[ |y| -> b( |f| -> \uplace{}{ [[ |x| -> e ]] } ) ]] ]]

[[ |k| -> \uplace{}{ [[ |f| -> [[ |x| -> ? ]] ( |x| -> b_2 ) ( |x| -> b_3 ) ( |x| -> e ) ]] } ]]
\end{phiquation*}
\end{mdframed}
\label{fig:scopes}
\caption{Illustrative examples of scopes: the bars over the terms highlight \(e^\varsigma\), the scope of \(e\).}
\end{figure*}

\subsection{Locators}\label{sec:locators}

In an expression, the locator ``\stx{\Phi}'' means the program,
while the locator ``\stx{\xi}'' means the scope of the expression.

\subsection{Head and Tail}

\begin{definition}
Since any expression may recursively be defined as either \begin{inparaenum}[1)]
    \item \(\dead\),
    \item formation,
    \item application,
    or
    \item dispatch,
\end{inparaenum}
it consists of a \textbf{head} and a possibly empty \textbf{tail}, denoted together as \(h\bullet{}t\).
\end{definition}

\begin{example}
\Cref{tab:head-and-tail} shows a few examples that demonstrate the separation
between a head and a possibly empty tail of an expression.

\begin{table*}
\caption{A few examples of objects that demonstrate the separation between a head and a possibly empty tail of an expression.}
\label{tab:head-and-tail}
\begin{tabular}{lll}
\toprule
Expression & Head & Tail \\
\midrule
$b_1( foo -> b_2 )$
  & $b_1( foo -> b_2 )$
  & --- \\
$b_1.\alpha_1.\alpha_2( 0-> b_2 )$
  & $b_1$
  & $\alpha_1.\alpha_2( 0-> b_2 )$ \\
$[[ |a| -> b_2 ]].|a|.|test|.|b|$
  & $[[ |a| -> b_2 ]]$
  & $\alpha_0.|test|.\alpha_2$ \\
$b_1( |foo| -> b_2)( 0-> 42 ).|print|( 1-> 7 )$
  & $b_1( |foo| -> b_2)( 0-> 42 )$
  & $|print|( 1-> 7 )$ \\
\bottomrule
\end{tabular}
\end{table*}
\end{example}

\subsection{Terminator}\label{sec:terminator}

\begin{definition}[Terminator]
The \textbf{terminator}, denoted as \stx{\dead}, is an object that equals itself when it is
the subject of a dispatch or an application:
\begin{equation*}
\forall \tau : \dead.\tau \trans \dead \qquad \forall \tau, e : \dead( \tau \mapsto e) \trans \dead.
\end{equation*}
\end{definition}

\subsection{Immutability}\label{sec:immutability}

\begin{definition}[Immutability]
Every object is \textbf{immutable}, meaning that an application of
its already attached attribute equals to \(\dead\):
\begin{equation*}
\forall \tau, e_1, e_2 : \llbracket \tau \mapsto e_1 \rrbracket ( \tau \mapsto e_2 ) \trans \dead
\end{equation*}
\end{definition}

\subsection{Atoms}\label{sec:atoms}

\begin{definition}[Atom]
\textbf{Atom} is an object with a function attached to its \(\lambda\)-asset.
\end{definition}

\Cref{eq:Sqrt} demonstrates an atom with a function that calculates
the square root of a number, which it retrieves from the \(\Delta\)-asset
of \(b.\alpha_0\) with the help of the morphing function (\cref{sec:morphing}).
The implementation of functions is outside the scope of \phic{}: they may be implemented,
for example, in \(\lambda\)-calculus or a programming language
such as Java or C++.

\subsection{Decoration}\label{sec:decoration}

\begin{definition}[Decoration]
Object \textbf{decoration} is a mechanism of extending an object (``decoratee'')
by attaching it to the ``\stx{\varphi}''-attribute of another object (``decorator''),
which makes attributes of the decoratee retrievable from the decorator,
unless the decorator has its own attributes with the same names.
\end{definition}

\subsection{Syntactic Sugar}

\Cref{tab:sugar} shows all possible syntax sugar.

\begin{table*}
\caption{Syntax sugar.}
\label{tab:sugar}
\newcommand\sugar[2]{$ #1 $ & $ #2 $ \\}
\newcommand\subs[1]{& \textcolor{gray}{(#1)} \\}
\newcommand\tto{\;\stx{\mapsto}\;}
\begin{tabular}{ll}
\toprule
Syntax sugar & Its more verbose equivalent \\
\midrule
\sugar
  {\stx{ QQ }}
  {\stx{ Q } \stx{.} |org| \stx{.} |eolang|}
\sugar
  {e \stx{(} \tau_1 \tto e_1, \tau_2 \tto e_2, \dots \stx{)}}
  {e \stx{(} \tau_1 \tto e_1 \stx{)}\stx{(} \tau_2 \tto e_2 \stx{)} \dots}
\sugar
  {e \stx{(} e_0 \stx{,}\; e_1 \stx{,}\; \dots \stx{)}}
  {e \stx{(} \alpha_0 \tto e_0 \stx{,}\; \alpha_1 \tto e_1 \stx{,}\; \dots \stx{)}}
\sugar
  {\tau_1 \stx{(} \tau_2 \stx{,}\; \tau_3 \stx{,}\; \dots \stx{)} \tto \stx{[[} B \stx{]]}}
  {\tau_1 \tto \stx{[[} \tau_2 \tto \stx{?} \stx{,}\; \tau_3 \tto \stx{?} \stx{,}\; \dots \stx{,}\; B \stx{]]}}
\sugar
  {\tau_1 \tto \tau_2}
  {\tau_1 \tto \stx{\xi}\stx{.}\tau_2}
\sugar
  {\stx{[[} B \stx{]]}}
  {\stx{[[} B \stx{,}\; \rho \tto \stx{?} \stx{]]} \quad\text{if}\; \rho \notin B}
\sugar
  {\tau \;\stx{\mapsto}\; \texttt{\begin{CJK}{UTF8}{gbsn}"你好"\end{CJK}}}
  {\tau \tto \stx{ QQ } \stx{.} |string| \stx{(} \stx{ QQ } \stx{.} |bytes| \stx{(} \stx{[[} \stx{ D> }\; |E4-BD-A0-E5-A5-BD| \stx{]]} \stx{)} \stx{)}}
  \subs{UTF-8 string}
\sugar
  {\tau \tto 42}
  {\tau \tto \stx{ QQ } \stx{.} |number| \stx{(} \stx{ QQ } \stx{.} |bytes| \stx{(} \stx{[[} \stx{ D> }\; |40-45-00-00-00-00-00-00| \stx{]]} \stx{)} \stx{)}}
  \subs{eight bytes per integer}
\sugar
  {\tau \tto 3.14}
  {\tau \tto \stx{ QQ } \stx{.} |number| \stx{(} \stx{ QQ } \stx{.} |bytes| \stx{(} \stx{[[} \stx{ D> }\; |40-09-1E-B8-51-EB-85-1F| \stx{]]} \stx{)} \stx{)}}
  \subs{eight bytes per number with a floating point}
\sugar
  {\stx{\Big\{} e \stx{\Big\}}}
  {\stx{ Q } \tto \stx{[[} e \stx{]]}}
\bottomrule
\end{tabular}
\end{table*}

\newpage
\subsection{Contextualization}\label{sec:contextualization}
\newcommand\ctx[2]{\lceil #1 \;\textcolor{gray}{\shortmid}\; #2 \rfloor}

\begin{definition}[Contextualization]
A contextualization function \(\mathbb{C} : \mathcal{E} \times \mathcal{B} \to \mathcal{E}\) denoted as \( \ctx{e}{b} \), which replaces locators with objects, is defined by induction:
\begin{enumerate}[label=\(\mathbb{C}_\arabic*:\),ref=\ensuremath{\mathbb{C}.\arabic*}]
  \item\label{C:Phi} $ \ctx{\Phi}{b} \trans e $ \quad if $ \Phi -> e $,
  \item\label{C:xi} $ \ctx{\xi}{b} \trans b $,
  \item\label{C:forma} $ \ctx{[[ B ]]}{b} \trans [[ B ]] $,
  \item\label{C:dead} $ \ctx{\dead}{b} \trans \dead $,
  \item\label{C:dot} $ \ctx{e.\tau}{b} \trans \ctx{e}{b}.\tau $,
  \item\label{C:app} $ \ctx{e_1( \tau -> e_2 )}{b} \trans \ctx{e_1}{b}( \tau -> \ctx{e_2}{b} ) $.
\end{enumerate}
\end{definition}

\Cref{app:contextualization-examples} demonstrates how contextualization function works, by examples.

\subsection{Normalization}\label{sec:normalization}

An expression that may be rewritten by the \emph{rules} (or \emph{reductions})
listed in \cref{fig:reduction} is a \emph{reducible} expression.
The notation \(e_1 \trans e_2\), optionally followed by a condition,
denotes a reduction of \(e_1\) to \(e_2\), if the condition holds.

These rules may be applied in any order.

A specific reduction may be denoted, for example, as \(\trans_{\nameref{r:dot}}\),
or just \(\trans\) if a reduction is meant with no specificity.
The notation \(e_1 \strans e_2\) denotes a reflexive transitive
closure of all reductions, so that there is a possibly empty finite
sequence of reductions between \(e_1\) and \(e_2\).

An expression that has no more possible applications of reductions
is \emph{irreducible} or a \emph{normal form}, denoted as \(\nf{}_i\)
ranging over \(\mathcal{N} \in \mathcal{E}\). Thus, \(\nf\) is a normal
form of \(e\) if \(e \strans \nf\) and there is no expression \(e_1\)
such that \(\nf \trans e_1\).

An expression may not have a normal form.

\begin{figure*}
\newcommand\trrule[5][]{%
  \newrule[#1]{#2}:
  &
  { $ #3 $ }
  \(\trans\)
  { $ #4 $ }
  \quad #5
  \\%
}
\begin{mdframed}
\renewcommand{\arraystretch}{1.2}
\begin{tabular}{rl}
\trrule{copy}
  { [[ B_1, \tau -> ?, B_2 ]] ( \tau -> e ) }
  { [[ B_1, \tau -> n, B_2 ]] }
  { if \( \ctx{e}{e\textsuperscript{\(\varsigma\)}} \strans n \) }
\trrule[\alpha]{alpha}
  { [[ B_1, \tau -> ?, B_2 ]] ( \alpha_i -> e ) }
  { [[ B_1, \tau -> ?, B_2 ]] ( \tau -> e) }
  { if $ \ordinal{\tau}{[[B]]} = i $ }
\trrule{dot}
  { [[ B_1, \tau -> \nf, B_2 ]].\tau }
  { \ctx{\nf}{n\textsuperscript{\(\varsigma\)}} ( \rho -> n\textsuperscript{\(\varsigma\)} ) }
  { }
\trrule[\varphi]{phi}
  { [[ B ]].\tau }
  { [[ B ]].\varphi.\tau }
  { if \( \tau \notin B\) and \( \varphi \in B\)  }
\trrule{stay}
  { [[ B_1, \rho -> e_1, B_2 ]]( \rho -> e_2 ) }
  { [[ B_1, \rho -> e_1, B_2 ]] }
  { }
\trrule{over}
  { [[ B_1, \tau -> e_1, B_2 ]]( \tau -> e_2) }
  { \dead }
  { if \( \tau \not= \rho \) }
\trrule{stop}
  { [[ B ]].\tau }
  { \dead }
  { if \( [ \tau, \varphi, \lambda ] \cap B = \emptyset \) }
\trrule{null}
  { [[ B_1, \tau -> ?, B_2 ]].\tau }
  { \dead }
  { }
\trrule{miss}
  { [[ B ]] ( \tau -> e ) }
  { \dead }
  { if \( \tau \notin B \) and \( \tau \notin [ \alpha_0, \alpha_1, \dots ] \) }
\trrule{dd}
  { \dead.\tau }
  { \dead }
  { }
\trrule{dc}
  { \dead ( \tau -> e ) }
  { \dead }
  { }
\end{tabular}
\end{mdframed}
\caption{Reduction rules.}
\label{fig:reduction}
\end{figure*}

\Cref{app:normalization-examples} demonstrates how normalization works, by examples.

\subsection{Primitives}\label{sec:primitives}

\begin{definition}[Primitive]
\textbf{Primitive} denoted by \(p\) and ranging over \(\mathcal{P} \in \mathcal{N}\)
is either \(\dead\) or object formation without \(\lambda\) asset.
\end{definition}

\begin{theorem}\label{th:norm-head}
A normal form is either a primitive or a dispatch where the subject is an atom.
\end{theorem}

\begin{proof}
\tbd{The proof is yet to be written.}
\end{proof}

\subsection{Confluence}\label{sec:confluence}

The reduction strategy in \cref{fig:reduction} is \emph{confluent},
i.e., possesses the Church–Rosser property~\citep{church1936some}. This means
that if there are reduction sequences from any expression to two different expressions,
then there exist reduction sequences from those two expressions to some common expression.
Moreover, the reduction strategy possesses the unique normal form property:

\begin{theorem}[Confluence]
Any expression may have only one normal form.
\end{theorem}

\begin{proof}
\tbd{The proof is going to be written soon.}
\end{proof}

\subsection{Equivalence}\label{sec:equivalence}

Two expressions are said to be \emph{syntactically equivalent} or just
\emph{equivalent} (denoted by \(\equiv\)) if their normal forms are
syntactically identical.

\Cref{fig:equivalence} shows some properties that hold.

\begin{figure*}
\begin{mdframed}
\begin{phiquation*}
e(\tau_1 -> e_1)(\tau_2 -> e_2) \equiv e(\tau_2 -> e_2)(\tau_1 -> e_1) \
\quad\text{(by \nameref{r:copy})}
  \text{Commutativity of Application}
\end{phiquation*}
\end{mdframed}
\caption{Equivalence properties that hold.}
\label{fig:equivalence}
\end{figure*}

\subsection{Morphing}\label{sec:morphing}

The \emph{morphing} function \(\mathbb{M} : \langle \mathcal{E}, \mathcal{S} \rangle \to \langle \mathcal{P}, \mathcal{S} \rangle\)
maps expressions to primitives, possibly modifying the \emph{state} of evaluation.
The inference rules at \cref{fig:morphing} inductively describe the algorithm of morphing.

\begin{figure*}
\begin{mdframed}
\begin{phiquation*}
\newrule{prim} \
\frac \
{  } \
{ \langle e, s \rangle \stepto \langle p, s \rangle }  \
\;\text{if}\; e \in \mathcal{P} \
\quad\quad \
\newrule{nmz} \
\frac \
{ \langle \nf, s_1 \rangle \stepto \langle p, s_2 \rangle } \
{ \langle e, s_1 \rangle \stepto \langle p, s_2 \rangle }  \
\;\text{if}\; e \strans \nf \;\text{and}\; e \not\equiv n

\newrule[\lambda]{lambda} \
\frac \
{ \langle e \bullet t, s_1 \rangle \stepto \langle p, s_2 \rangle} \
{ \langle [[ B_1, L> f, B_2 ]] \bullet{} t, s_1 \rangle \stepto \langle p, s_2 \rangle }  \
\;\text{if}\; f( [[ B_1, B_2 ]], s_1 ) \to \langle e, s_2 \rangle

\newrule[\Phi]{Phi} \
\frac \
{ \langle e \bullet{} t, s_1 \rangle \stepto \langle p, s_2 \rangle} \
{ \langle \Phi.\tau \bullet{} t, s_1 \rangle \stepto \langle p, s_2 \rangle }  \
\;\text{if}\; \Phi \mapsto \llbracket B_1, \tau \mapsto e, B_2 \rrbracket
\end{phiquation*}
\end{mdframed}
\caption{Morphing rules.}
\label{fig:morphing}
\end{figure*}

The notation \(\langle e, s_1 \rangle \stepto \langle p, s_2 \rangle\)
means that \(\mathbb{M}(e, s_1)\) evaluates to \(p\), thus \emph{morphing} \(e\)
with a side-effect of changing the state of evaluation \(s_1\) to \(s_2\).

In \nameref{r:lambda}, the function \(f\) is called by value.

\begin{lemma}
The morphing function is not total because not every expression has a normal form.
\end{lemma}

\subsection{Dataization}\label{sec:dataization}

The \emph{dataization} function
\(\mathbb{D} : \langle \mathcal{E}, \mathcal{S} \rangle \rightharpoonup \langle \mathcal{D}, \mathcal{S} \rangle\)
maps expressions to data, possibly modifying the \emph{state} of evaluation.
The inference rules in \cref{fig:dataization} inductively describe the algorithm of dataization.

\begin{figure*}
\begin{mdframed}
\begin{phiquation*}
\newrule[\Delta]{delta} \
\frac \
{ } \
{ \langle [[ B_1, D> \delta, B_2 ]], s \rangle \stepto \langle \delta, s \rangle } \
\quad\quad \
\newrule{norm} \
\frac{ \langle p, s_1 \rangle \stepto \langle \delta, s_2 \rangle }{ \langle e, s_1 \rangle  \stepto \langle \delta, s_2 \rangle } \
\;\text{if}\; \mathbb{M}(e, s_1) \to \langle p, s_2 \rangle

\newrule{box} \
\frac \
{ \langle \ctx{e}{e\textsuperscript{\(\varsigma\)}}, s_1 \rangle \stepto \langle \delta, s_2 \rangle } \
{ \langle [[ B_1, @ -> e, B_2 ]], s_1 \rangle  \stepto \langle \delta, s_2 \rangle } \
\;\text{if}\; [\Delta, \lambda] \cap ( B_1 \cup B_2 ) = \emptyset
\end{phiquation*}
\end{mdframed}
\caption{Dataization rules.}
\label{fig:dataization}
\end{figure*}

The rules \nameref{r:box} and \nameref{r:phi} coexist because
\(\Delta \notin \mathcal{T}\), thus making the expression $ [[ @ -> 42 ]].\Delta $ invalid.

The notation \(\langle e, s_1 \rangle \stepto \langle \delta, s_2 \rangle\)
means that \(\mathbb{D}(e, s_1)\) evaluates to \(\delta\),
thus \emph{dataizing} \(e\) with a side-effect of changing the state of evaluation \(s_1\) to \(s_2\).
The notation \(\mathbb{D}(e)\) is a shortened form of \(\mathbb{D}(e, \varnothing)_{(1)}\),
which is a \emph{pure} dataization --- inputting an empty state of evaluation and ignoring
the output state of evaluation. Here and later,
the notation \(x_{(i)}\) denotes the \(i\)-th element of the tuple \(x\).

\begin{lemma}
The dataization function is partial, because not every primitive contains \(\Delta\)-asset.
\end{lemma}

\Cref{app:dataization-examples} demonstrates how dataization function works, by examples.

\subsection{Parent}

\begin{definition}[Parent]
Attaching expression \(e\) to the \(\rho\) attribute of object \(b\)
means setting the \textbf{parent} of \(b\) to \(e\).
\end{definition}

The presence of ``parent'' in each object is essential for the coordination of inner
objects after dynamic dispatch. Consider the following formation of an abstract object
with an inner object:
\begin{phiquation*}
\Big\{ |x| -> [[ |a| -> ?, next -> [[ @ -> \xi.^.\alpha_0.plus( 1 ), ^ -> Q.|x| ]] ]], |k| -> \xi.|x|( |42| ) \Big\}.
\end{phiquation*}
Here, if the parent of \(\Phi.|x|.|next|\) were attached in the formation, the result of \(\mathbb{D}(\Phi.|k|.|next|)\)
would not be equal to |43|. Instead, it would be equal to \(\dead\), because \(\Phi.|k|.|next|.\rho\)
would still be attached to \(\Phi.|x|\) after the dispatch of \(\Phi.|k|.|next|\).
The parent attribute may be compared with the |this| pointer in Java or C++, which
does not point anywhere until a method of a class is called. Then, when the method
is called, the |this| pointer refers to the object that owns the method.

\subsection{Congruence}

Two expressions \(e_1\) and \(e_2\) are said to be \emph{behaviorally equivalent}
or \emph{congruent} (denoted by \(\cong\)) if for any state \(s\): \(\mathbb{D}(e_1, s) = \mathbb{D}(e_2, s)\).

The following properties hold:
\begin{phiquation*}
\nf \cong [[ @ -> \nf ]] \quad \text{(by \nameref{r:phi})}
  \text{Transparency of Decoration}

[[ \tau_1 -> ?, B]] \cong [[ B ]] \quad \text{(by \nameref{r:stop})}
  \text{Redundancy of Void Attributes}

[[ B ]] \cong [[ B ]].@ \quad\text{if}\; \stx{@} \in B \;\text{and}\; \stx{\Delta} \not\in B \quad \text{(by \nameref{r:phi} and \nameref{r:dot})}
  \text{Implicitness of Decoration}
\end{phiquation*}

Two congruent expressions may be non-equivalent, for example:
\begin{phiquation*}
\Big\{ \tau_1 -> [[ foo -> ?, D> 01-02 ]], \tau_2 -> [[ bar -> ?, D> 01-02 ]] \Big\}
Q.\tau_1 \cong Q.\tau_2 \;\not\to\; Q.\tau_1 \equiv Q.\tau_2.
\end{phiquation*}

% \subsection{Dataless Objects}

% \begin{definition}[\(\Delta\)-depth]
% \(\Delta\)-depth of an object describes how deep data is in the object when recursively traversing values attached to the object attributes.
% More specifically:
% \begin{enumerate}
%   \item the \(\Delta\)-depth of an object with \(\Delta\)-asset is \(1\);
%   \item the \(\Delta\)-depth of an empty object is \(\infty\);
%   \item otherwise, the \(\Delta\)-depth of an object is:
%     \begin{inparaenum}
%       \item \(1 + M\), where \(M\) is the minimal \(\Delta\)-depth among objects attached to its attributes;
%       \item \(\infty\) if no objects are attached to its attributes;
%     \end{inparaenum}.
% \end{enumerate}
% \end{definition}

% \begin{definition}[Dataless Object]
% An object is \emph{dataless} if its \(\Delta\)-depth is greater than 2.
% \end{definition}

% \begin{example}
% \Cref{tab:depths} demonstrates objects with their \(\Delta\)-depths.
% \begin{table*}
% \caption{A few examples of objects and their \(\Delta\)-depths.}
% \label{tab:depths}
% \begin{tabular}{lll}
% \toprule
% Formation                                               & \(\Delta\)-Depth & Dataless \\
% \midrule
% $[[  ]]$                                                & \(\infty\)       & Yes      \\
% $[[ |a| -> ? ]]$                                           & \(\infty\)       & Yes      \\
% $[[ D> |42-43| ]]$                                      & 1              & No       \\
% $[[ bar -> [[ D> ? ]] ]]$                               & \(\infty\)       & Yes      \\
% $[[ foo -> [[ D> 01-02-03 ]] ]]$                        & 2              & No       \\
% $[[ D> |42-|, L> Fn ]]$                                 & 1              & No       \\
% $[[ L> Fn ]]$                                           & \(\infty\)       & Yes      \\
% $[[ |a| -> ?, x -> [[ y -> ?, z -> [[ D> |42-01| ]] ]] ]]$ & 3              & Yes      \\
% \bottomrule
% \end{tabular}
% \end{table*}
% \end{example}
