% SPDX-FileCopyrightText: Copyright (c) 2016-2025 Objectionary.com
% SPDX-License-Identifier: MIT

\section{Future Studies}\label{sec:future}

This paper intentionally focuses on defining \phic{} (syntax, foundations, and semantic operators) and does not yet demonstrate practical use.
At the same time, it motivates the calculus as a step toward more rigorous reasoning,
  verified compilers and static analyzers, and better language design.
Future work may build on \phic{} by creating the following artifacts:

\begin{itemize}
  \item A short list of core laws about reductions (what always works, what may fail).
  \item A clear way to prove that two expressions behave the same in any context.
  \item A type system that catches mistakes early (missing attributes, wrong attachments, etc.).
  \item A way to mark which expressions are pure and which may have side effects.
  \item A catalog of safe rewrites (how to change code without changing meaning).
  \item A translation from a small subset of an OO language (e.g., Java/C++) into \(\varphi\)-expressions.
  \item A machine-checked version of the calculus in a proof assistant.
  \item Basic tools on top of phino: find unreachable parts, simplify expressions, suggest safe refactorings.
  \item A standard library of built-in atoms (numbers, strings, collections) with precise rules.
  \item Larger language features built as add-ons (overriding, interfaces, constructors).
  \item Add-ons for mutation, concurrency, and distribution, with simple safety guarantees.
\end{itemize}

We invite readers to extend, formalize, and apply \phic{} to real object-oriented patterns and languages,
  and to help turn these foundations into a shared body of results and artifacts.
