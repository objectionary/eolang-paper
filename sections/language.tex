% SPDX-FileCopyrightText: Copyright (c) 2016-2025 Objectionary.com
% SPDX-License-Identifier: MIT

% synthetic attributes
% const
% idempotence
% lints

% true, false, if
% try/catch
% seq, goto, while,
% tuples
% malloc
% number
% const

The entire syntax of the \eolang{} language in BNF is available on the first page of the \texttt{objectionary/eo} GitHub repository%
  \footnote{\url{https://github.com/objectionary/eo}}.
In this section, we only explain key concepts of the language.

\subsection{Informal Overview}

At runtime, every \eolang{} object consists of a few \emph{cells}.
A cell responds to a few requests, where ``reference'' resembles a pointer:
\begin{enumerate}
  \item \ff{take(attr)} --- returns a reference, by provided attribute;
  \item \ff{put(attr, cell)} --- stores a reference, by the attribute;
  \item \ff{delta()} --- turns itself to data;
  \item \ff{copy()} --- makes a copy of itself.
\end{enumerate}

\begin{figure*}
\begin{mdframed}
\ffinput{order/order.eo}
\end{mdframed}
\capt{Order object}
\label{fig:order}
\end{figure*}


Consider EO code at \cref{fig:order}.
This code is equivalent to the following \phic{} expression:
\iexec[quiet]{eoc parse --sources=order}
\iexec{phino rewrite --sweet --flat --nonumber --output=latex --input=xmir --hide=Q.order.json.xi🌵 --hide=Q.order.xi🌵 < .eoc/1-parse/order.xmir}

\cref{fig:cells} shows eight cells that represent the \ff{order} object at runtime.
When being asked to \ff{take()}, cells behave differently.
Their behavior constitutes the semantics of the language.
Cells belong to one of six types, denoted by different colors: application, dispatch, formation, decoration, atom, and payload.

\begin{figure*}
\begin{mdframed}
\begin{tikzpicture}
[every node/.style={draw},
up/.style={draw=none, anchor=south west, label position=north west, align=left},
cell/.style={rotate=90, font={\scriptsize\sffamily}, draw=none, anchor=south west, label position=south west}]
\node [label={[up, text width=10em]$c_1$}, label={[cell]decoration}, fill=orange!5] (c1)
  {\parbox{9em}{
  \begin{tabularx}{\linewidth}{Xl}
  \ff{decorator}: & \(c_2\) \\
  \ff{decoratee}: & \(c_5\) \\
  \ff{memo}: & \(\langle\varnothing, \varnothing\rangle\) \\
  \end{tabularx}}};
\node [label={[up, text width=10em]$c_2=[[ qty -> ?, \newline price -> c_7, \rho -> ? ]]$}, label={[cell]formation}, below=1.2 of c1, fill=blue!5] (c2)
  {\parbox{7em}{
  \begin{tabularx}{\linewidth}{Xl}
  \ff{qty}: & \(\varnothing\) \\
  \ff{price}: & \(c_7\) \\
  \(\rho\): & \(\varnothing\) \\
  \end{tabularx}}};
\node [label={[up]$c_3 = c_0.qty$}, label={[cell]dispatch}, right=of c1, fill=green!5] (c3)
  {\parbox{7em}{
  \begin{tabularx}{\linewidth}{Xl}
  \ff{base}: & \(c_2\) \\
  \ff{attr}: & \ff{qty} \\
  \ff{memo}: & \(\varnothing\) \\
  \end{tabularx}}};
\node [label={[up]$c_4 = c_3.mul$}, label={[cell]dispatch}, fill=green!5, right=of c3] (c4)
  {\parbox{7em}{
  \begin{tabularx}{\linewidth}{Xl}
  \ff{base}: & \(c_3\) \\
  \ff{attr}: & \ff{mul} \\
  \ff{memo}: & \(\varnothing\) \\
  \end{tabularx}}};
\node [label={[up]$c_6 = c_0.price$}, label={[cell]dispatch}, fill=green!5, right=of c2] (c6)
  {\parbox{8em}{
  \begin{tabularx}{\linewidth}{Xl}
  \ff{base}: & \(c_2\) \\
  \ff{attr}: & \ff{price} \\
  \ff{memo}: & \(\varnothing\) \\
  \end{tabularx}}};
\node [label={[up]$c_5 = c_4(0-> c_6)$}, label={[cell]application}, fill=red!5, right=of c6] (c5)
  {\parbox{8em}{
  \begin{tabularx}{\linewidth}{Xl}
  \ff{prototype}: & \(c_4\) \\
  \ff{attribute}: & \(\alpha_0\) \\
  \ff{argument}: & \(c_6\) \\
  \ff{memo}: & \(\varnothing\) \\
  \end{tabularx}}};
\node [label={[up]$c_7 = \ff{19.99}$}, label={[cell]payload}, fill=cyan!5, below=of c6] (c7)
  {\parbox{7em}{
  \begin{tabularx}{\linewidth}{Xl}
  \(\Delta\): & \ff{19.99} \\
  \ff{base}: & \(\dead\) \\
  \end{tabularx}}};
\node [label={[up]\(c_8\)}, label={[cell]atom}, fill=purple!5, right=of c7] (c8)
  {\parbox{5em}{
  \begin{tabularx}{\linewidth}{Xl}
  \(\lambda\): & \ff{J} \\
  \ff{base}: & \(\dead\) \\
  \end{tabularx}}};
\end{tikzpicture}
\end{mdframed}
\capt{Cells representing the object \ff{order} of \cref{fig:order}.}
\label{fig:cells}
\end{figure*}

\subsection{Application}

The \emph{application} cell contains
\begin{inparaenum}[1)]
\item a ``prototype'' reference,
\item an attribute,
\item an ``argument'' reference,
and
\item a mutable initially blank ``memo'' reference.
\end{inparaenum}

On any request, if the memo is blank, the cell
\begin{inparaenum}[1)]
\item asks prototype to \ff{copy()} expecting a ``copy'' reference to be returned,
\item asks the copy to \ff{put(attribute, argument)},
and
\item stores the copy to the memo.
\end{inparaenum}
Then, the cell passes the request to the memo and returns the result.

\subsection{Dispatch}

The \emph{dispatch} cell contains
\begin{inparaenum}[1)]
\item a ``base'' reference,
\item an attribute,
and
\item a mutable initially blank ``memo'' reference.
\end{inparaenum}

On any request, if memo is blank, the cell
\begin{inparaenum}[1)]
\item asks base to \ff{take(attribute)} expecting a ``target'' reference to be returned,
\item asks target to \ff{put(\(\rho\),base)},
and
\item stores target to the memo.
\end{inparaenum}
Then, it passes the request to the memo and returns the result.

\subsection{Formation}

The \emph{formation} cell contains
\begin{inparaenum}[1)]
\item an associative array of attributes and references,
\item a possibly blank \(\lambda\)-asset,
and
\item a possibly blank \(\Delta\)-asset.
\end{inparaenum}

When being asked to \ff{put()}, the cell stores the reference to the array if the given attribute is not yet attached.
If the attribute is already attached, the cell returns \(\dead\).

When being asked to \ff{take()}, the cell returns the cell that is referenced by the corresponding attribute.
If the attribute is attached to a blank, the cell returns \(\dead\).

\subsection{Decoration}

The \emph{decoration} cell contains
\begin{inparaenum}[1)]
\item a ``decorator'' reference,
\item a ``decoratee'' reference,
and
\item a mutable initially blank ``memo'' associative array of attributes and references.
\end{inparaenum}

When being asked to \ff{take()}, the cell returns the cell stored in the memo.
If the memo is empty, the cell asks the decorator to \ff{take()}.
If \(\dead\) is returned, the cell asks the decoratee to \ff{take()}.
The cell stores the result into the memo and returns it.

\subsection{Atom}

The \emph{atom} cell contains
\begin{inparaenum}[1)]
\item a \(\lambda\)-asset attached to a function
and
\item a ``base'' reference.
\end{inparaenum}

On any request, the cell passes the request to the base and returns the result.
If the result is \(\dead\), the cell calls the function and then passes the request to the result of the function.
The execution of a function attached may be implemented via foreign function interface (FFI).

The function accepts a reference to the cell.

A function may create cells, but it man not delete them.
Instead, they get deleted automatically except the cell that the function returns and the cells directly or indirectly referenced by it.

\subsection{Payload}

The \emph{payload} cell contains
\begin{inparaenum}[1)]
\item a \(\Delta\)-asset attached to data
and
\item a ``base'' reference.
\end{inparaenum}

On the \ff{delta()} request, the cell returns the data attached to the \(\Delta\)-asset.

The cell passes all other requests to the ``base'' reference.
