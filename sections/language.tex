% SPDX-FileCopyrightText: Copyright (c) 2016-2025 Objectionary.com
% SPDX-License-Identifier: MIT

% synthetic attributes
% const
% idempotence
% lints

% true, false, if
% try/catch
% seq, goto, while,
% tuples
% malloc
% number
% const

The entire syntax of the \eolang{} language in BNF is available on the first page of the \texttt{objectionary/eo} GitHub repository%
  \footnote{\url{https://github.com/objectionary/eo}}.
In this section, we only explain key concepts of the language.

\subsection{Informal Overview}

At runtime, every \eolang{} object resembles an associative array.
It responds to a few requests, where ``reference'' resembles a pointer:
\begin{enumerate}
  \item \ff{take(attr)} returns a reference, by provided attribute;
  \item \ff{put(attr, object)} saves a reference, by the attribute;
  \item \ff{delta()} turns itself to data;
  \item \ff{copy()} makes a copy of itself.
\end{enumerate}

Consider the following code:

\begin{ffcode}
[qty price] > order (*@\label{ln:order-start}@*)
  qty.mul > cost
    price (*@\label{ln:order-stop}@*)
\end{ffcode}

The code is equivalent to the following \phic{} expression:
\begin{phiquation*}
\Big\{ order -> [[ qty -> ?, price -> ?, cost -> \xi.qty.mul(\xi.price) ]] \Big\}
\end{phiquation*}

\cref{fig:order} shows five objects that represent the code in runtime.
When being asked to \ff{take()}, objects behave differently.
Their behavior constitutes the semantics of the language.
Objects belong to one of three types: application, dispatch, and formation.

\begin{figure*}
\begin{mdframed}
\begin{tikzpicture}
[every node/.style={draw},
up/.style={draw=none, anchor=south west, label position=north west, align=left},
object/.style={rotate=90, font={\scriptsize\sffamily}, draw=none, anchor=south west, label position=south west}]
\node [label={[up, text width=10em]$b_1=[[ qty -> ?, price -> ?, cost -> b_4, \rho -> ? ]]$}, label={[object]formation}, fill=blue!5] (b1)
  {\parbox{7em}{
  \begin{tabularx}{\linewidth}{Xl}
  \ff{qty}: & \(\varnothing\) \\
  \ff{price}: & \(\varnothing\) \\
  \ff{cost}: & \(b_4\) \\
  \(\rho\): & \(\varnothing\) \\
  \end{tabularx}}};
\node [label={[up]$b_2 = b_1.qty$}, label={[object]dispatch}, above right=0 and 2.5 of b1, fill=green!5] (b2)
  {\parbox{7em}{
  \begin{tabularx}{\linewidth}{Xl}
  \ff{base}: & \(b_1\) \\
  \ff{attr}: & \ff{qty} \\
  \ff{memo}: & \(\varnothing\) \\
  \end{tabularx}}};
\node [label={[up]$b_3 = b_2.mul$}, label={[object]dispatch}, fill=green!5, right=of b2] (b3)
  {\parbox{7em}{
  \begin{tabularx}{\linewidth}{Xl}
  \ff{base}: & \(b_2\) \\
  \ff{attr}: & \ff{mul} \\
  \ff{memo}: & \(\varnothing\) \\
  \end{tabularx}}};
\node [label={[up]$b_4 = b_3(0-> b_5)$}, label={[object]application}, fill=red!5, below=of b3] (b4)
  {\parbox{8em}{
  \begin{tabularx}{\linewidth}{Xl}
  \ff{prototype}: & \(b_3\) \\
  \ff{attr}: & \(\alpha_0\) \\
  \ff{argument}: & \(b_5\) \\
  \end{tabularx}}};
\node [label={[up]$b_5 = b_1.price$}, label={[object]dispatch}, fill=green!5, left=of b4] (b5)
  {\parbox{8em}{
  \begin{tabularx}{\linewidth}{Xl}
  \ff{base}: & \(b_1\) \\
  \ff{attr}: & \ff{price} \\
  \ff{memo}: & \(\varnothing\) \\
  \end{tabularx}}};
\end{tikzpicture}
\end{mdframed}
\capt{Actors representing the program at the lines \ref{ln:order-start}--\ref{ln:order-stop}.}
\label{fig:order}
\end{figure*}

\subsection{Application}

The \emph{application} object contains
\begin{inparaenum}[1)]
\item a ``prototype'' reference,
\item an attribute,
and
\item an ``argument'' reference.
\end{inparaenum}

On any request, the object
\begin{inparaenum}[1)]
\item asks prototype to \ff{copy()} expecting a ``copy'' reference to be returned,
\item asks the copy to \ff{put(attribute, argument)},
and
\item passes the request to the copy.
\end{inparaenum}

\subsection{Dispatch}

The \emph{dispatch} object contains
\begin{inparaenum}[1)]
\item a ``base'' reference,
\item an attribute,
and
\item a mutable initially blank ``memo'' reference.
\end{inparaenum}

On any request, the object does two things.
First, if memo is blank, it
\begin{inparaenum}[1)]
\item asks base to \ff{take(attribute)} expecting a ``target'' reference to be returned,
\item asks target to \ff{put(\(\rho\),base)},
and
\item saves target to the memo.
\end{inparaenum}
Then, it passes the request to the memo.

\subsection{Formation}

The \emph{formation} object contains
\begin{inparaenum}[1)]
\item an associative array of attributes and references,
\item a possibly blank \(\lambda\)-asset,
and
\item a possibly blank \(\Delta\)-asset.
\end{inparaenum}

When being asked to \ff{put()}, the object saves the reference to the array if the given attribute is not yet attached.
If the attribute is already attached, the object returns an error.

When being asked to \ff{take()}, the object returns the reference that corresponds to the given attribute.
If the attribute is attached to a blank, the object returns an error.
If the attribute is absent and the \(\varphi\)-attribute is present, the object asks the object referenced by the attribute to \ff{take()}.
If the \(\varphi\)-attribute is absent but the \(\lambda\)-asset is present, the object calls the function attached to the asset, which returns a reference, and the object uses it to \ff{take()}.
Otherwise, the object returns an error.

When being asked to \ff{delta()}, the object returns the \(\Delta\)-asset if it has one.
If it doesn't have it but the \(\lambda\)-asset is present, the object calls the function attached to the \(\lambda\)-asset, gets a reference, and uses it to \ff{delta()}.
If the \(\lambda\)-asset is also absent but the \(\varphi\)-attribute is present, the object asks the object referenced by the attribute to \ff{delta()}.
Otherwise, the object returns an error.

\subsection{Atoms}

A formation object with a non-blank \(\lambda\)-asset is an \emph{atom}.
The execution of a function attached to the \(\lambda\)-asset may be implemented via foreign function interface (FFI).

The function accepts a reference to the owner of the \(\lambda\)-asset and returns a reference.
