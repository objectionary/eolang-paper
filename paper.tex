% SPDX-FileCopyrightText: Copyright (c) 2016-2025 Objectionary.com
% SPDX-License-Identifier: MIT

\documentclass[acmsmall,nonacm,11pt]{acmart}
\settopmatter{printfolios=false,printccs=false,printacmref=false}
\usepackage[utf8]{inputenc}
\usepackage[T2A,T1]{fontenc}
\usepackage{natbib} % for \citep and \citet
\usepackage{amsmath}
\usepackage[russian,english]{babel}
  \renewcommand\ttdefault{cmtt}
\usepackage{csquotes}
\usepackage{mdframed} % for drawing frames around figures
\usepackage[novert]{ffcode} % for fixed-fonts
\usepackage{phigure} % local, in this directory
\usepackage{CJKutf8} % for chinese font
\usepackage{paralist} % for inlined lists
\usepackage{cancel} % to enable \cancel command
\usepackage{anyfontsize} % To get rid of font not found warnings
\usepackage{eolang} % for EO sources and formulas
\usepackage{tabularx} % for special tables
\usepackage{naive-ebnf} % for EBNF
\usepackage[hide]{to-be-determined} % for \tbd command
\usepackage{href-ul} % for nicely underscored links
\usepackage{multicol} % for two cols in BNF
\usepackage{mathtools} % for matrix* environment
\usepackage[shortlabels]{enumitem} % for changing labes in "enumerate"
\usepackage[capitalize,nameinlink]{cleveref} % must stay last!
\renewcommand\emph[1]{\textit{#1}} % to prevent a weird bug coming from some of the packages above

\usepackage{silence}
  \WarningFilter{acmart}{\vspace should only be used to provide space above/below}

\tolerance=1500
\raggedbottom
\setlength\headheight{21pt}

\newcommand\nospell[1]{#1}
\newcommand\br{\\[-4pt]}
\newcommand\figcap[1]{\caption{#1}\Description{#1}}
\newcommand\lref[1]{the line no.~\ref{ln:#1}}
\newcommand\lrefs[2]{the lines~\ref{ln:#1}--\ref{ln:#2}}

\newenvironment{twocols}{}{}
\newcommand\fn[1]{\text{\scshape\sffamily #1}}
\def\trans{\mathrel{\leadsto}}
\def\strans{\mathrel{\setbox0\hbox{\(\trans\)}\rlap{\hbox to \wd0{\hss*\hss}}\box0}}
\def\stepto{\mathrel{\Downarrow}}
\def\dtz{\mathrel{\downdownarrows}}
\def\nf{n}
\def\dead{\bot}
\newcommand\nv{v_{+}}
\newcommand\rj{\mathop{\circ}\nolimits}
\definecolor{rgbTerminals}{RGB}{0,0,128}
\renewcommand\EbnfTerminal[1]{\textcolor{rgbTerminals}{#1}}

\AtEndPreamble{\newtheorem{axiom}[theorem]{Axiom}}

\newcommand\stx[1]{%
  \EbnfTerminal{%
    \relax\ifmmode%
      #1%
    \else%
      \(#1\)%
    \fi%
  }%
}

\newcounter{rule}
\makeatletter
\newcommand\rulemnemo[1]{\text{\sffamily{}R\(_\text{\scshape\sffamily{}#1}\)}}
\newcommand\newrule[2][]{%
  \def\param{#1}%
  \ifx\param\empty%
    \def\sub{#2}%
  \else
    \def\sub{\(#1\)}%
  \fi%
  \refstepcounter{rule}%
  \protected@edef\@currentlabelname{\rulemnemo{\sub}}%
  \label@noarg{r:#2}%
  \rulemnemo{\sub}%
}
\makeatother

\newcounter{uplace}
\makeatletter
\newcommand\uplace[2]{%
  \ifmeasuring@\else\refstepcounter{uplace}\cref@label{#1}\fi%
  \def\param{#1}%
  {\color{gray}\overbracket[0.4pt][1pt]{{\color{black}#2}}^{{\ifx\param\empty\else\color{gray}\Pi_{\theuplace}\fi}}}}
\makeatother
\crefformat{uplace}{{\color{gray}\Pi_{#2#1#3}}}

\newcounter{unorm}
\makeatletter
\newcommand\unorm[2]{%
  \ifmeasuring@\else\refstepcounter{unorm}\cref@label{#1}\fi%
  \def\param{#1}%
  {\color{gray}\overbracket[0.4pt][1pt]{{\color{black}#2}}^{{\ifx\param\empty\else\color{gray}\mathrm{H}_{\theunorm}\fi}}}}
\makeatother
\crefformat{unorm}{{\color{gray}\mathrm{H}_{#2#1#3}}}

\setlength{\footskip}{13.0pt}

\acmBooktitle{untitled}
\title{EOLANG and \texorpdfstring{\(\varphi\)}{phi}-calculus}
\subtitle{%
  Ver:
  \texorpdfstring{
    \href{https://github.com/REPOSITORY/releases/tag/0.0.0}
      {\ff{0.0.0}}
  }{0.0.0}
}
\author{Yegor Bugayenko}
\orcid{0000-0001-6370-0678}
\email{yegor256@gmail.com}
\affiliation{
  \institution{Huawei}
  \city{Moscow}
  \country{Russia}
}
\ccsdesc[300]{Software and its engineering~Software notations and tools~Formal language definitions}
\keywords{Object-Oriented Programming, Object Calculus}

\begin{document}

\begin{abstract}
Object-oriented programming (OOP) is one of the most popular
paradigms used for building software systems\footnote{%
  \LaTeX{} sources of this paper are maintained in
  \href{https://github.com/REPOSITORY}{REPOSITORY} GitHub repository,
  the rendered version is \href{https://github.com/REPOSITORY/releases/tag/0.0.0}{\ff{0.0.0}}.}.
However, despite
its industrial and academic popularity, OOP is still missing
a formal apparatus similar to \(\lambda\)-calculus, which functional
programming is based on. There were a number of attempts to formalize
OOP, but none of them managed to cover all the features available in
modern OO programming languages, such as C++ or Java.
We have made yet another attempt and created \phic{}. We also
created EOLANG (also called \eolang{}), an experimental
programming language based on \phic{}.
\end{abstract}

\maketitle

\section{Introduction}
\label{sec:intro}
% SPDX-FileCopyrightText: Copyright (c) 2016-2025 Yegor Bugayenko
% SPDX-License-Identifier: MIT

\eolang{}\footnote{\url{https://www.eolang.org}}
was created in order to eliminate the problem of complexity of
OOP code, providing
\begin{inparaenum}[1)]
  \item a formal object calculus and
  \item a programming language with a reduced set of features.
\end{inparaenum}
The proposed \phic{} represents an object model through
data and objects, while operations with them are possible
through formation, application, decoration, dispatch, and dataization. The calculus
introduces a formal apparatus for manipulations with objects.

\eolang{}, the proposed programming language, fully implements
all elements of the calculus and enables implementation of
an object model on any computational platform.
Being an OO programming language, \eolang{} enables four key principles of OOP:
abstraction, inheritance, polymorphism, and encapsulation.


The rest of the paper is dedicated to the discussion of the syntax of the language that we created based on the calculus, and the calculus itself.
To make it easier to understand, we start the discussion with the syntax of the language, while the calculus is derived from it.
Then, we discuss the key features of \eolang{} and the differences between it and other programming languages.
At the end of the paper, we overview the work done by others in the area of formalization of OOP.

\section{Syntax}
\label{sec:syntax}
% SPDX-FileCopyrightText: Copyright (c) 2016-2025 Objectionary.com
% SPDX-License-Identifier: MIT

\section{Syntax}\label{sec:syntax}

The syntax of a program is defined by BNF in \cref{fig:ebnf} (the starting symbol is \EbnfNonTerminal{Program}).

\begin{figure*}
\begin{mdframed}
\raggedright
\begin{ebnf}[8em]
<Program> := "\(\Phi\)" "\(\mapsto\)" <Expression> \\
<Expression> := <Formation> | <Application> | <Dispatch> | "\(\dead\)" \\
<Formation> := "\(\llbracket\)" <Binding> "\(\rrbracket\)" \\
<Application> := <Expression> "\(\lparen\)" <A-Pair> "\(\rparen\)" \\
<A-Pair> := <\(\tau\)-Pair> | <\(\alpha\)-Pair> \\
<Dispatch> := <Subject> "." <Attribute> \\
<Subject> := <Expression> | <Locator> \\
<Locator> := "\(\Phi\)" | "\(\xi\)" \\
<Binding> := <Pair> <Bindings> | \(\epsilon\) \\
<Bindings> := "," <Pair> <Bindings> | \(\epsilon\) \\
<Pair> := <\(\varnothing\)-Pair> | <\(\tau\)-Pair> | <\(\Delta\)-Pair> | <\(\lambda\)-Pair> \\
<\(\varnothing\)-Pair> := <Attribute> "\(\mapsto\)" "\(\varnothing\)" \\
<\(\tau\)-Pair> := <Attribute> "\(\mapsto\)" <Subject> \\
<\(\alpha\)-Pair> := <Alpha> "\(\mapsto\)" <Subject> \\
<\(\Delta\)-Pair> := "\(\Delta\)" "\(\phiDotted\)" <Data> \\
<\(\lambda\)-Pair> := "\(\lambda\)" "\(\phiDotted\)" <Function> \\
\end{ebnf}
\end{mdframed}
\capt{Syntax as a context-free grammar, in BNF.}
\label{fig:ebnf}
\end{figure*}

Besides the literals mentioned in the grammar in blue color, the
alphabet includes three non-terminals that rewrite to terminals as follows:
\begin{itemize}
  \item \EbnfNonTerminal{Attribute}: either \begin{inparaenum}[1)]
      \item Greek letter \(\varphi\),
      \item Greek letter \(\rho\),
      or
      \item a string of lowercase English letters possibly with dashes inside, e.g. ``\ff{price}'' or ``\ff{a-car}'';
  \end{inparaenum}
  \item \EbnfNonTerminal{Data}: a sequence of bytes in hexadecimal format, e.g. ``\ff{EF-41-5C}'' is a sequence of three bytes, ``\ff{42-}'' is a one-byte sequence (with a trailing dash in order to avoid confusion with integers), and ``\ff{-{}-}'' (double dash) is an empty sequence of bytes;
  \item \EbnfNonTerminal{Function}: a string of English letters where the first letter is in uppercase, e.g. ``\ff{Sqrt}'';
  \item \EbnfNonTerminal{Alpha}: a Greek letter \(\alpha\) with a non-negative whole-number index, e.g. \(\alpha_2\).
\end{itemize}


\section{Calculus}
\label{sec:calculus}
% SPDX-FileCopyrightText: Copyright (c) 2024-2025 Yegor Bugayenko
% SPDX-License-Identifier: MIT

The proposed \phic{} is based on set theory~\citep{jech2013set} and lambda calculus,
representing objects as sets of pairs and their internals as \(\lambda\)-terms.
\Cref{fig:ebnf} presents formal syntax of \phic{}.

\begin{figure*}
\begin{mdframed}
\raggedright
\begin{ebnf}[8em]
<Program> := "\(\Phi\)" "\(\mapsto\)" <Object> \\
<Object> := <Formation> | <Application> | <Dispatch> | "\(\Phi\)" | "\(\xi\)" | "\(\perp\)" \\
<Formation> := "\(\llbracket\)" <Bindings> "\(\rrbracket\)" \\
<Application> := <Object> "\(\lparen\)" <Bindings> "\(\rparen\)" \\
<Dispatch> := <Object> "." <Attribute> \\
<Bindings> := <Bindings> "," <Binding> | \(\epsilon\) \\
<Binding> := <\(\tau\)-Binding> | <\(\Delta\)-Binding> | <\(\lambda\)-Binding> \\
<\(\tau\)-Binding> := <Attribute> "\(\mapsto\)" ( <Object> | "\(\varnothing\)" ) \\
<\(\Delta\)-Binding> := "\(\Delta\)" "\(\phiDotted\)" ( <Data> | "\(\varnothing\)" ) \\
<\(\lambda\)-Binding> := "\(\lambda\)" "\(\phiDotted\)" <Function> \\
\end{ebnf}
\end{mdframed}
\caption{Syntax as a context-free grammar, in EBNF}
\label{fig:ebnf}
\end{figure*}

Besides the literals mentioned in the grammar in blue color, the alphabet includes three non-terminals that rewrite to terminals as such:
\begin{itemize}
\item \EbnfNonTerminal{Attribute}: either \begin{inparaenum}[1)]
    \item one of two Greek letters \(\varphi\) and \(\rho\),
    \item a Greek letter \(\alpha\) with an optional index,
    or
    \item a string of English letters starting with a lowercase letter, e.g. ``\ff{price}'' or ``\ff{color}'';
\end{inparaenum}
\item \EbnfNonTerminal{Data}: a sequence of bytes in hexadecimal format, e.g. ``\texttt{EF-41-5C}'' is a sequence of three bytes, ``\texttt{42-}'' is a one-byte sequence (with a tailing dash in order to avoid confusion with integers), and ``\texttt{-{}-}'' (double dash) is an empty sequence of bytes;
\item \EbnfNonTerminal{Function}: a string of English letters where the first letter is uppercase, e.g. ``\ff{Sqrt}''.
\end{itemize}

\subsection{Bindings}

\begin{definition}[Binding]
\textbf{Binding}, ranging by \(B\), is a possibly empty map of key-value pairs, where all keys are unique, denoted as $k_1 -> v_1, k_2 -> v_2, \dots, k_n -> v_n$.
\end{definition}

The following operations can be performed on a binding:
\begin{itemize}
\item The statement \(k \in B\) evaluates to true if the key \(k\) is present within the map;
% \item The operation \(B-k\) results in a new binding that excludes the key-value pair with the key \(k\) (if such a pair exists);
\item The expression \(B[k]\) returns the value associated with the key \(k\).
\end{itemize}

\subsection{Attributes}

\begin{definition}[Attribute]
\textbf{Attribute}, ranged over \(\mathcal{T}\) by \(\tau_i\), is an identifier that is either
\begin{inparaenum}[1)]
    \item a text string that starts with a lower-case English letter,
    \item \(\alpha\) with an index,
    or
    \item one of the following two Greek letters:
    \(\varphi\),
    and
    \(\rho\).
\end{inparaenum}
\end{definition}

\subsection{Asset}

\begin{definition}[Asset]
\textbf{Asset} is either \(\Delta\) or \(\lambda\) Greek letter.
\end{definition}

\subsection{Data}

\begin{definition}[Data]
\textbf{Data} is a possibly empty sequence of 8-bit bytes, ranged over \(\mathcal{D}\) by \(\delta_i\).
\end{definition}

\subsection{Objects}

\begin{definition}[Object]
\textbf{Object} is either an terminator, denoted as \(\perp\), or a binding where keys are attributes or assets and the following conditions hold:
\begin{inparaenum}[1)]
\item if a key is \(\Delta\), then the value is a data or \(\varnothing\),
\item if a key is \(\lambda\), then the value is a function,
\item otherwise, the value is either an object or \(\varnothing\).
\end{inparaenum}
Objects range over \(\mathcal{B}\) by \(b_i\).
\end{definition}

The following expression is an example of an object:
\begin{phiquation}
\label{eq:object-example}
b_1 = [[ D> 42, 0-> b_2(|foo| -> b_3).|bar|, 1-> [[ L> Sqrt ]], 5-> \perp ]]
\end{phiquation}

The arrow ``\(\mapsto\)'' denotes an attachment of an object (right-hand side) to the attributes (left-hand side). The arrow ``\(\phiDotted\)'' denotes an attachment of a data or a function to either \(\Delta\) or \(\lambda\) assets respectively.

\begin{definition}[Domain]
\textbf{Domain} of an object is a set of all attributes of the object.
\end{definition}

The domain of the object of \cref{eq:object-example} is the following set:
\(\{ \alpha_0, \alpha_1, \alpha_5 \}\). The asset \(\Delta\) does not belong to the domain.

\begin{definition}[Abstract Object]
An object is \textbf{abstract} if at least one of its attribute is attached to \(\varnothing\), otherwise the object is \textbf{closed}.
\end{definition}

The object of \cref{eq:object-example} is abstract, because its attribute \ff{price} is attached to \(\varnothing\). The object of \cref{eq:simple-application} was abstract before the application, but the object that it was reduced to by the application, is a closed object since its attribute \(\alpha_1\) is attached to \(b_2\).

Even though \(\varnothing\) may be attached to an attribute of an object, it is not an object by itself. Instead, \(\varnothing\) is a ``placeholder'' for an object, which is attached to an attribute until an object is attached to it.

\subsection{Formation}

\begin{definition}[Formation]
Object \textbf{formation}, denoted as $[[B]]$, is a construction of a new object.
\end{definition}

We introduce the term ``object formation'' rather than using a more convenient
term ``construction'' because the latter generally implies a presence of a
class from which an object is being constructed or instantiated. Instead,
object formation is closer to the creation of a prototype, which may either be
used ``as is'' or copied.

\Cref{eq:price-color} is an example of object formation, where the binding of
the abstract object being formed consists of two pairs: $price -> ?$ and
$color -> [[D>"red"]]$. The object attached to the \ff{color} attribute is a
formation of a closed object.

\subsection{Application}

\begin{definition}[Application]
Object \textbf{application}, denoted as $b(B)$, is a copy of an existing abstract object \(b\), with new values attached to some of its attributes and/or assets.
\end{definition}

\Cref{eq:simple-application} demonstrates object application, where the binding $( 1-> b_2 )$ is applied to the formation of an abstract object $[[ 1-> ?]]$. The application creates a new object $[[ 1-> b_2 ]]$, while the existing abstract object remains intact.

\subsection{Forma}

\begin{definition}[Forma]
A \textbf{forma} of an object \(b\), denoted as \(\mathbb{F}(b)\), is the abstract object that was copied in order to create \(b\). A forma of a formation is the formation itself.
\end{definition}

In \cref{eq:simple-application}, the forma of $[[ 1-> b_2 ]]$ is the abstract object $[[ 1->? ]]$, while the forma of $[[ 1->? ]]$ is $[[ 1->? ]]$.

\subsection{Programs}

\begin{definition}[Program]
\textbf{Program} is an object attached to \(\Phi\) attribute of \emph{Universe}.
\end{definition}

The Universe resembles an object with the only attribute \(\Phi\), but it is not an object since it may not be attached to any attribute of any other object.

\subsection{Scope}

\begin{definition}[Scope]
Object \textbf{scope} is either the formation where the object is attached to an attribute, or the scope of the application where the object is used in a binding.
\end{definition}

In simpler terms, scope is the formation that is the ``closest'' to the object, moving to the left. In the following object formation, the scope of \(b_1\) is the formation where the \ff{source} attribute stays, while the scope of \(b_2\) is the formation where the \ff{ref} attribute stays:
\begin{phiquation}
    [[ ref -> [[ source -> Q.book( author -> b_1 ) ]], cite -> b_2 ]]
\end{phiquation}

\subsection{Dispatch}

\begin{definition}[Dispatch]
Object \textbf{dispatch} (also known as ``dot notation''), denoted as $x.\tau_1$, is a retrieval of an object attached to attribute \(\tau_1\) of \(x\), where \(x\) (the ``subject'') may be an object, \(\Phi\), or \(\xi\); and then attaching its \(\rho\) attribute to the subject, if it is not attached yet.
\end{definition}

The subject \(\Phi\) means the program.

The subject \(\xi\) means the scope of the dispatch.

\Cref{eq:dot-notation} demonstrates how dot notation works in a combination with object application. First, the scope \(\xi\) denotes the object formation, where \(x\) stays. Then, the \(.\ff{p}\) retrieves the object attached to \(\ff{p}\) attribute---this object is the formation where \(|t|\) stays. Then, the application of the $|t|->b$ binding makes a copy of the formation and attaches \(b\) to \(|t|\). Then, \(.|y|\) retrieves the object attached to \(|y|\) attribute of the copy: it is \(\xi.|t|\), where \(\xi\) means the nearest object formation, which is the copy itself. Finally, the dispatch retrieves what is the attached to \(b\) of the copy, which is \(b\).

\subsection{Terminator}

\begin{definition}[Terminator]
The \textbf{terminator} object, denoted as \(\perp\), equals to itself when being a subject of a dispatch or an application:
\begin{equation*}
\forall \tau : \perp.\tau \leadsto \perp \qquad \forall B : \perp(B) \leadsto \perp.
\end{equation*}
\end{definition}

\subsection{Immutability}

\begin{definition}[Immutability]
Every object is \textbf{immutable}, meaning that an application of its already attached attribute or asset equals to \(\perp\):
\begin{equation*}
\forall \tau, b_1, b_2 : \llbracket \tau \mapsto b_1 \rrbracket ( \tau \mapsto b_2 ) \leadsto \perp
\end{equation*}
\end{definition}

\subsection{Atoms}

\begin{definition}[Atom]
\textbf{Atom} is an object formation with a total function \(\langle \mathcal{B}, \mathcal{S} \rangle \to \langle \mathcal{B}, \mathcal{S} \rangle\) attached to its \(\lambda\)-asset, that maps objects to objects, possibly modifying the \emph{state} of evaluation \(s\), which ranges over \(\mathcal{S}\).
\end{definition}

\Cref{eq:Sqrt} demonstrates an atom with a function that calculates a square root of a number, which it retrieves from the \(\Delta\)-asset of \(b.\alpha_0\), with the help of dataization function. The implementation of functions is outside of the scope of \phic{}: they may be implemented, for example, in \(\lambda\)-calculus or some programming language, for example, Java or C++.

\subsection{Decoration}

\begin{definition}[Decoration]
Object \textbf{decoration} is a mechanism of extending an object (``decoratee'') by attaching it to the attribute \(\varphi\) of another object (``decorator''), which makes attributes of the decoratee retrievable from the decorator, unless the decorator has its own attributes with the same names.
\end{definition}

\subsection{Parent}

\begin{definition}[Parent]
Attaching object \(b_1\) to the \(\rho\) attribute of object \(b_2\) means setting the \textbf{parent} of \(b_2\) to \(b_1\).
\end{definition}


\section{Key Features}
\label{sec:features}
\input{sections/features}

\section{Four Principles of OOP}
\label{sec:four}
\input{sections/four}

\section{Related Work}
\label{sec:related}
% SPDX-FileCopyrightText: Copyright (c) 2016-2025 Objectionary.com
% SPDX-License-Identifier: MIT

In this section we analyze and categorize prior art related to our work.
Neither object-oriented formalism nor pure object-oriented languages are new research topics.
However, we identified certain gaps in existing studies that make us believe that our work has novelty.

\subsection{Object Calculi}

Attempts were made to formalize OOP and introduce object calculus, similar to lambda calculus~\citep{barendregt2012} used in functional programming.
For example, \citet{abadi1995imperative} suggested an imperative calculus of objects, which was extended by~\citet{bono1998imperative} to support classes, by~\citet{gordon1998concurrent} to support concurrency and synchronization, and by~\citet{jeffrey1999distributed} to support distributed programming.

Earlier, \citet{honda1991object} combined OOP and \(\pi\)-calculus in order to introduce object calculus for asynchronous communication, which was further referenced by~\citet{jones1993pi} in their work on object-based design notation.

A few attempts were made to reduce existing OOP languages and formalize what is left.
Featherweight Java is the most notable example proposed by~\citet{igarashi2001featherweight}, which omits almost all features of the full language (including interfaces and even assignment) to obtain a small calculus.
Later it was extended by~\citet{jagannathan2005transactional} to support nested and multi-threaded transactions.
Featherweight Java is used in formal languages such as Obsidian~\citep{coblenz2019} and SJF~\citep{usov2020}.

Another example is Larch/C++~\citep{cheon1994quick}, which is a formal algebraic interface specification language tailored to C++.
It allows interfaces of C++ classes and functions to be documented in a way that is unambiguous and concise.

Several attempts to formalize OOP were made by extensions of the most popular formal notations and methods, such as Object-Z~\citep{duke1991object} and VDM++~\citep{durr1992vdm}.
In Object-Z, state and operation schemes are encapsulated into classes.
The formal model is based upon the idea of a class history~\citep{duke1990towards}.
However, all these OO extensions do not have comprehensive refinement rules that can be used to transform specifications into implemented code in an actual OO programming language, as was noted by~\citet{paige1999object}.

\citet{bancilhon1985calculus} suggested an object calculus as an extension to relational calculus.
\citet{jankowska2003anotheroop} further developed these ideas and related them to a Boolean algebra.
\citet{leekwakryu1996transform} developed an algorithm to transform an object calculus into an object algebra.

However, all these theoretical attempts to formalize OO languages were not able to fully describe their features, as was noted by~\citet{nierstrasz1991towards}: ``The development of concurrent object-based programming languages has suffered from the lack of any generally accepted formal foundations for defining their semantics.''
In addition, when describing the attempts of formalization, \citet{eden2002visual} summarized: ``Not one of the notations is defined formally, nor provided with \nospell{denotational} semantics, nor founded on axiomatic semantics.''
Moreover, despite these efforts, \citet{ciaffaglione2003reasoning,ciaffaglione2003typetheories,ciaffaglione2007theory_of_contexts} noted in their series of works that relatively little formal work has been carried out on object-based languages, and it remains true to this day.

\subsection{Pure Object-Oriented Languages}

Since 1966, when Simula~\citep{dahl1966simula} was created and the term ``object-oriented programming'' was later coined by
Alan Kay~\citep{kay97keynote}, the exact characteristics defining object-orientation have been ambiguous~\citep{stefik1985object,madsen1988object,armstrong2006quarks}.
Initially, several key features were proposed by various authors~\citep{nygaard1986basic,stroustrup1987object,meyer1988object,korson1990understanding,wirfs1990designing,coad1991object,booch1994object}, including objects, classes, polymorphism, inheritance, abstraction, encapsulation, and dynamic binding.
Over time, object-orientation has been expanded to include additional elements such as, for example, types, traits, concurrency, annotations, records, and aspects.

Meanwhile, some programming languages claim to be ``pure'' object-oriented.
According to \citet{chambers1991making}, this designation implies that ``all computation, including low-level operations such as variable access, arithmetic, and array indexing, is performed by sending messages to objects.''
A broader definition by \citet{west2004object} suggests that being ``pure'' means ``everything is an object,'' which implies the exclusion of non-object entities such as procedures and operators~\citep{joque2016invention}.
\Cref{tab:languages} lists several object-oriented languages that describe themselves as ``pure'' in their documentation, yet include additional constructs beyond mere objects.

\begin{table}
\caption{A non-exhaustive list of object-oriented languages that claim to be ``pure.''}
\label{tab:languages}
% SPDX-FileCopyrightText: Copyright (c) 2016-2025 Objectionary.com
% SPDX-License-Identifier: MIT

\newcommand\yes{+}
\newcommand*\rot{\rotatebox{90}}
\begin{tabularx}{\linewidth}{llcccccccccX}
\toprule
Language
  & DoB
  & \rot{Classes}
  & \rot{Types}
  & \rot{Modules}
  & \rot{Operators}
  & \rot{Statements}
  & \rot{Procedures}
  & \rot{Arrays}
  & \rot{NULL}
  & \rot{Mixins}
  & Others \\
\midrule
Smalltalk~\citep{goldbergrobson1983smalltalk}
  & 1972
  & % classes
  & % types
  & % modules
  & % operators
  & % statements
  & % procedures
  & % arrays
  & % NULL
  & % mixins
  &
  \\
Trellis/Owl~\citep{schaffert1985trellis}
  & 1985
  & % classes
  & % types
  & % modules
  & % operators
  & % statements
  & % procedures
  & % arrays
  & % NULL
  & % mixins
  &
  \\
Emerald~\citep{black1986object}
  & 1985
  & % classes
  & % types
  & % modules
  & % operators
  & % statements
  & % procedures
  & % arrays
  & % NULL
  & % mixins
  &
  \\
Eiffel~\citep{meyer1986genericity}
  & 1986
  & \yes % classes
  & \yes % types
  & % modules
  & % operators
  & % statements
  & \yes % procedures
  & % arrays
  & % NULL
  & % mixins
  & contracts
  \\
Self~\citep{ungar1987self}
  & 1987
  & % classes
  & % types
  & % modules
  & % operators
  & % statements
  & % procedures
  & % arrays
  & % NULL
  & % mixins
  &
  \\
BETA~\citep{madsen1993object}
  & 1993
  % & classes, patterns\\
  & % classes
  & % types
  & % modules
  & % operators
  & % statements
  & % procedures
  & % arrays
  & % NULL
  & % mixins
  &
  \\
Ruby~\citep{flanagan2008ruby}
  & 1995
  & \yes % classes
  & % types
  & \yes % modules
  & % operators
  & \yes % statements
  & \yes % procedures
  & % arrays
  & % NULL
  & \yes % mixins
  &
  \\
Scala~\citep{odersky2004overview}
  & 2004
  & % classes
  & % types
  & % modules
  & % operators
  & % statements
  & % procedures
  & % arrays
  & % NULL
  & % mixins
  \\
Dart~\citep{walrath2012dart}
  & 2011
  & % classes
  & % types
  & % modules
  & % operators
  & % statements
  & % procedures
  & % arrays
  & % NULL
  & % mixins
  \\
Wyvern~\citep{nistor2013wyvern}
  & 2014
  & % classes
  & % types
  & % modules
  & % operators
  & % statements
  & % procedures
  & % arrays
  & % NULL
  & % mixins
  \\
JEff~\citep{inostroza2018jeff}
  & 2018
  & % classes
  & % types
  & % modules
  & % operators
  & % statements
  & % procedures
  & % arrays
  & % NULL
  & % mixins
  \\
\bottomrule
\end{tabularx}

\end{table}

Even though ``pure'' object-orientation may not be a virtue by itself, we claim that \eolang{} is the only object-oriented programming language that has nothing but objects.


\section{Acknowledgments}
% SPDX-FileCopyrightText: Copyright (c) 2016-2025 Objectionary.com
% SPDX-License-Identifier: MIT

Many thanks to (in alphabetic order of last names)
  \nospell{Aliaksei Bial\'{i}auski},
  \nospell{Fabricio Cabral},
  \nospell{Kirill Chernyavskiy},
  \nospell{Piotr Chmielowski},
  \nospell{Danilo Danko},
  \nospell{Konstantin Gukov},
  \nospell{Andrew Graur},
  \nospell{Ali-Sultan Ki\-giz\-ba\-ev},
  \nospell{Nikolai Ku\-da\-sov},
  \nospell{Alexander Legalov},
  \nospell{Mikhail Lipanin},
  \nospell{Tymur \(\lambda\)ysenko},
  \nospell{Alexandr Naumchev},
  \nospell{Alonso A. Ortega},
  \nospell{John Page},
  \nospell{Alex Panov},
  \nospell{Alexander Pushkarev},
  \nospell{Marcos Douglas B. Santos},
  \nospell{Alex Semenyuk},
  \nospell{Violetta Sim},
  \nospell{Sergei Skliar},
  \nospell{Stian Soiland-Reyes},
  \nospell{Viacheslav Tradunskyi},
  \nospell{Maxim Trunnikov},
  \nospell{Ilya Trub},
  \nospell{César Soto Valero},
  \nospell{Alena Vasileva},
  \nospell{David West},
  and
  \nospell{Vladimir Zakharov}
for their contribution to the development of \eolang{} and \phic{}.


{\raggedright
\bibliographystyle{ACM-Reference-Format}
\bibliography{bibliography/main}}
\vfill\eject

\appendix
% SPDX-FileCopyrightText: Copyright (c) 2016-2025 Objectionary.com
% SPDX-License-Identifier: MIT

\newpage
\section{Examples of Normalization}
\label{app:normalization-examples}

\newcounter{exmp}
\newcommand\phiExpression[1]{%
  \stepcounter{exmp}
  e_{\theexmp} = }

The following examples demonstrate how the reduction rules of \cref{fig:reduction} may normalize
some expressions, involving the contextualization function (\cref{sec:contextualization}):

\iexec[maybe]{./examples-to-tex.sh examples/basic > _tex/examples.tex}
\input{_tex/examples.tex}

The following expressions may not be reduced any further; they are in normal form:

\iexec[maybe]{./examples-to-tex.sh examples/nf > _tex/examples-nf.tex}
\input{_tex/examples-nf.tex}

Normalization of these expressions leads to endless recursion:

\iexec[maybe]{./endless-to-tex.sh examples/endless > _tex/examples-endless.tex}
\input{_tex/examples-endless.tex}

% SPDX-FileCopyrightText: Copyright (c) 2016-2025 Objectionary.com
% SPDX-License-Identifier: MIT

\newpage
\section{Examples of Contextualization}
\label{app:contextualization-examples}

The following examples demonstrate how the contextualization function (\cref{sec:contextualization}) works:

\begin{phiquation*}
\ctx{\xi.|x|.|k|}{b} \trans_{\ref{C:dot}}
  \trans \ctx{\xi.|x|}{b}.|k| \trans_{\ref{C:dot}}
  \trans \ctx{\xi}{b}.|x|.|k| \trans_{\ref{C:xi}}
  \trans b.|x|.|k|.
\end{phiquation*}

\begin{phiquation*}
\ctx{\xi.|t| ( |x| -> \xi, |k| -> \xi.|f| )}{b} \trans_{\ref{C:app}}
   \trans \ctx{\xi.|t|}{b} ( |x| -> \ctx{\xi}{b}, |k| -> \ctx{\xi.|f|}{b} ) \trans_{\ref{C:dot}}
   \trans \ctx{\xi.|t|}{b} ( |x| -> \ctx{\xi}{b}, |k| -> \ctx{\xi}{b}.|f| ) \trans_{\ref{C:xi}}
   \trans \ctx{\xi.|t|}{b} ( |x| -> \ctx{\xi}{b}, |k| -> b.|f| ) \trans_{\ref{C:xi}}
   \trans \ctx{\xi.|t|}{b} ( |x| -> b, |k| -> b.|f| ) \trans_{\ref{C:dot}}
   \trans \ctx{\xi}{b}.|t| ( |x| -> b, |k| -> b.|f| ) \trans_{\ref{C:xi}}
   \trans b.|t| ( |x| -> b, |k| -> b.|f| ).
\end{phiquation*}

% SPDX-FileCopyrightText: Copyright (c) 2016-2025 Objectionary.com
% SPDX-License-Identifier: MIT

\newpage
\section{Examples of Dataization}
\label{app:dataization-examples}

Consider a program that converts degrees from Celsius to Fahrenheit using
the formula:
\begin{equation*}
^{\circ}F ={} ^{\circ}C \times 1.8 + 32,
\end{equation*}
Assuming that \(^{\circ}C\) already equals \ff{25} and is attached to the \ff{c} attribute instead of being provided by a user:
\iexec[maybe]{phino rewrite --output=latex --sweet --nonumber --label=eq:celsius --canonize ./examples/celsius.phi > _tex/celsius-formula.tex}
\input{_tex/celsius-formula.tex}

The functions are defined as follows:
\begin{phiquation*}
|F1|(b,s) \to \langle Q.number( D> \mathbb{D} {(} b.^ {)} \boldsymbol{\times} \mathbb{D} {(} b.\alpha_0 {)} ), s \rangle
|F2|(b,s) \to \langle Q.number( D> \mathbb{D} {(} b.^ {)} \boldsymbol{+} \mathbb{D} {(} b.\alpha_0 {)} ), s \rangle
\end{phiquation*}

The following sequence of transformations constitutes the evaluation of the program:

\iexec[maybe]{phino dataize --output=latex --sweet --nonumber --compress --meet-prefix=dataization --sequence --flat --quiet --hide=Q.org --locator=Q.ex ./examples/celsius.phi > _tex/celsius-dataization.tex}
\input{_tex/celsius-dataization.tex}


\end{document}
