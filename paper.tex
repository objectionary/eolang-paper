% SPDX-FileCopyrightText: Copyright (c) 2016-2025 Objectionary.com
% SPDX-License-Identifier: MIT

\documentclass[sigplan,nonacm]{acmart}
\settopmatter{printfolios=false,printccs=false,printacmref=false}
\usepackage[utf8]{inputenc}
\usepackage[T2A,T1]{fontenc}
\usepackage{natbib} % for \citep and \citet
\usepackage{stmaryrd} % for arrows, like \mapstochar
\let\Bbbk\relax\usepackage{amssymb} % for special symbols
\usepackage{amsmath}
\usepackage[russian,english]{babel}
  \renewcommand\ttdefault{cmtt}
\usepackage{csquotes}
\usepackage{mdframed} % for drawing frames around figures
\usepackage{naive-ebnf} % for drawing EBNF
\usepackage[novert]{ffcode} % for fixed-fonts
\usepackage{phigure} % local, in this directory
\usepackage[capitalize]{cleveref} % for \cref
\usepackage{CJKutf8} % for chinese font
\usepackage{paralist} % for inlined lists
\usepackage{cancel} % to enable \cancel command
\usepackage{anyfontsize} % To get rid of font not found warnings
\usepackage{eolang} % for EO sources and formulas
% \usepackage{bibcop} % for checking quality of the .bib file
\usepackage{tabularx} % for special tables
\usepackage{to-be-determined} % for \tbd command
\usepackage{href-ul} % for nicely underscored links
\usepackage{tcolorbox} % for algorithm
  \tcbuselibrary{skins}
\usepackage{algpseudocode} % for algorithms
\usepackage{multicol} % for two cols in BNF
\usepackage{pgffor} % to enable \foreach
\usepackage{mathtools} % for matrix* environment

\usepackage{silence}
  \WarningFilter{acmart}{\vspace should only be used to provide space above/below}

\tolerance=1500
\raggedbottom
\setlength\headheight{21pt}

\newcommand\nospell[1]{#1}
\newcommand\br{\\[-4pt]}
\newcommand\figcap[1]{\caption{#1}\Description{#1}}
\newcommand\lref[1]{the line no.~\ref{ln:#1}}
\newcommand\lrefs[2]{the lines~\ref{ln:#1}--\ref{ln:#2}}

\newenvironment{twocols}{}{}

\newcounter{rule}
\renewcommand\therule{R\arabic{rule}}
\newcommand\rr{\triangleright{}}
\newcommand\rrule[1]{{\scshape\sffamily\ref{rule:#1}}}
\newcommand{\jrule}[1]{%
  \refstepcounter{rule}\label{rule:#1}%
  \text{\textbf{\rrule{#1}}}}
\newcommand*{\ohat}[2]{%
  \overbracket[0.4pt][2pt]{\textcolor{black}{#2}}^{\color{gray}#1}}
\newenvironment{algo}
  {\newcommand\kw[1]{{\bfseries\sffamily ##1}}
  \newcommand\tab{{\hspace*{1em}}}
  \noindent}
  {}

\setlength{\footskip}{13.0pt}

\acmBooktitle{untitled}
\title{EOLANG and \texorpdfstring{\(\varphi\)}{phi}-calculus}
\subtitle{%
  Ver:
  \texorpdfstring{
    \href{https://github.com/REPOSITORY/releases/tag/0.0.0}
      {\ff{0.0.0}}
  }{0.0.0}
}
\author{Yegor Bugayenko}
\orcid{0000-0001-6370-0678}
\email{yegor256@gmail.com}
\affiliation{
  \institution{Huawei}
  \city{Moscow}
  \country{Russia}
}
\ccsdesc[300]{Software and its engineering~Software notations and tools~Formal language definitions}
\keywords{Object-Oriented Programming, Object Calculus}

\begin{document}

\begin{abstract}
Object-oriented programming (OOP) is one of the most popular
paradigms used for building software systems\footnote{%
  \LaTeX{} sources of this paper are maintained in
  \href{https://github.com/REPOSITORY}{REPOSITORY} GitHub repository,
  the rendered version is \href{https://github.com/REPOSITORY/releases/tag/0.0.0}{\ff{0.0.0}}.}.
However, despite
its industrial and academic popularity, OOP is still missing
a formal apparatus similar to \(\lambda\)-calculus, which functional
programming is based on. There were a number of attempts to formalize
OOP, but none of them managed to cover all the features available in
modern OO programming languages, such as C++ or Java.
We have made yet another attempt and created \phic{}. We also
created EOLANG (also called \eolang{}), an experimental
programming language based on \phic{}.
\end{abstract}

\maketitle

\section{Introduction}
\label{sec:intro}
\eolang{}\footnote{\url{https://www.eolang.org}}
was created in order to eliminate the problem of complexity of
OOP code, providing
\begin{inparaenum}[1)]
  \item a formal object calculus and
  \item a programming language with a reduced set of features.
\end{inparaenum}
The proposed \phic{} represents an object model through
data and objects, while operations with them are possible
through formation, application, decoration, dispatch, and dataization. The calculus
introduces a formal apparatus for manipulations with objects.

\eolang{}, the proposed programming language, fully implements
all elements of the calculus and enables implementation of
an object model on any computational platform.
Being an OO programming language, \eolang{} enables four key principles of OOP:
abstraction, inheritance, polymorphism, and encapsulation.


The rest of the paper is dedicated to the discussion of the syntax of the language that we created based on the calculus, the calculus itself.
In order to make it easier to understand, we start the discussion with the syntax of the language, while the calculus is derived from it.
Then, we discuss the key features of \eolang{} and the differences between it and other programming languages.
At the end of the paper we overview the work done by others in the area of formalization of OOP.

\section{Syntax}
\label{sec:syntax}
% SPDX-FileCopyrightText: Copyright (c) 2016-2025 Objectionary.com
% SPDX-License-Identifier: MIT

The entire syntax of the \eolang{} language in BNF is available on the first page of the
\texttt{objectionary/eo} GitHub repository\footnote{\url{https://github.com/objectionary/eo}}.
Here, we only explain the basics of the language, not its entire syntax.

Similar to Python~\citep{lutz2013learning}, indentation in \eolang{} is part of the syntax:
the scope of a code block is determined by its horizontal position
in relation to other blocks, which is also known as ``off-side rule''~\citep{landin1966next}.

There are no keywords in \eolang{} but only a few special symbols
denoting grammar constructs, such as \ff{>} for attribute naming,
\ff{.} for dot notation, and \ff{[]} for specifying parameters of formations.

Attributes, which are the only identifiers that exist in \eolang{}, may have
any Unicode symbols in their names, as long as they start with a small English letter
and do not contain spaces, line breaks, or the symbols that are part of the syntax:
\ff{test-File} and
\begin{CJK}{UTF8}{gbsn}
\ff{i文件}
\end{CJK}
 are valid identifiers.
Identifiers are case-sensitive: \ff{acar} and \ff{aCar} are two different identifiers.
Java notation is used for numbers and strings.

\subsection{Objects}

This is a \emph{formation} of a new object \ff{book} that has a single \ff{isbn} attribute:

\begin{ffcode}
# A book in the library.
[isbn] > book (*@\label{ln:book}@*)
\end{ffcode}

To make another object with a specific ISBN, the \ff{book}
has to be \emph{copied}, with the \emph{data} as an argument:

\begin{ffcode}
book "978-1519166913" > b1
\end{ffcode}

Here, \ff{b1} is a new object created.
Its only attribute is accessible as \ff{b1.isbn}.

\subsection{Indentation}

This is an example of an \emph{abstract} object \ff{vector}, where
spaces are replaced with the ``\textvisiblespace'' symbol to demonstrate
the importance of their presence in specific quantities
(for example, there must be exactly one space after the closing square bracket on the
second line and the \ff{>} symbol, while two spaces will break the syntax):

{\lstset{showspaces=true}\begin{ffcode}
# This is a vector in 2D space.(*@\label{ln:comment}@*)
[dx dy] > vector(*@\label{ln:vector}@*)
  sqrt. > length(*@\label{ln:length}@*)
    plus.
      dx.times dx
      dy.times dy(*@\label{ln:length-end}@*)(*@\label{ln:vector-end}@*)
\end{ffcode}
}

The code at \lref{comment} is a \emph{comment}.
Two \emph{void attributes}, \ff{dx} and \ff{dy},
are listed in square brackets at \lref{vector}.
The \emph{name} of the object goes after the \ff{>} symbol.
The code at \lref{length} defines
an \emph{attached attribute} \ff{length}. Wherever an object
needs to get a name, the \ff{>} symbol can be added after the object.

The declaration of the attribute \ff{length} at \lrefs{length}{length-end}
can be written in one line, using \emph{dot notation}:

\begin{ffcode}
((dx.times dx).plus (dy.times dy)).sqrt > length
\end{ffcode}

An \emph{inverse} dot notation is used to simplify
the syntax. The identifier that goes after the dot is written
first, the dot follows, and the next line contains the part
that is supposed to come before the dot. It is also possible to rewrite
this expression in multiple lines without the usage of
inverse notation, but it will look subjectively less readable:

\begin{ffcode}
dx.times dx (*@\label{ln:dx-pow}@*)
.plus
  dy.times dy (*@\label{ln:dx-pow-2}@*)
.sqrt > length (*@\label{ln:dx-pow-3}@*)
\end{ffcode}

Here, \lref{dx-pow} is the application of the object \ff{dx.times} with
a new argument \ff{dx}. Then, the next line is the object \ff{plus} taken
from the object created at the first line, using dot notation. Then,
\lref{dx-pow-2} is the argument passed to the object \ff{plus}.
The code at \lref{dx-pow-3} takes the object \ff{sqrt} from the object constructed
at the previous line, and gives it the name \ff{length}.

Indentation is used for two purposes: either to define attributes
of an abstract object or to specify arguments for object application, also
known as making a \emph{copy}.

A definition of an abstract object starts with a list of void attributes
in square brackets on one line, followed by a list of attached attributes
each in its own line. For example, this is an abstract \emph{anonymous} object
(it does not have a name)
with one void attribute \ff{x} and two attached attributes \ff{succ} and \ff{prev}:

\begin{ffcode}
# A counter.
[x]
  x.plus 1 > succ
  x.minus 1 > prev
\end{ffcode}

The arguments of \ff{plus} and \ff{minus} are provided in \emph{horizontal}
mode, without the use of indentation. It is possible to rewrite this code
in a \emph{vertical} mode, where indentation will be required:

\begin{ffcode}
# A counter.
[x] (*@\label{ln:succ}@*)
  x.plus > succ
    1
  x.minus > prev
    1 (*@\label{ln:succ-end}@*)
\end{ffcode}

This abstract object can also be written in a horizontal mode,
because it is anonymous:

\begin{ffcode}
[x] (x.plus 1 > succ) (x.minus 1 > prev)
\end{ffcode}

\subsection{Data}

There is only one abstract object that can encapsulate data: \ff{bytes}.
Copies of the \ff{bytes} object are created by the compiler when it meets
a special syntax for data, for example:

\begin{ffcode}
3.14 > radius
\end{ffcode}

This code is compiled into the following:

\begin{ffcode}
number > radius
  bytes
    40-09-1E-B8-51-EB-85-1F
\end{ffcode}

Similar to the floating-point number compilation into a copy
of \ff{bytes}, there are a few other syntax shortcuts, listed in~\cref{tab:types}.

\begin{table}
\begin{tabularx}{\columnwidth}{l|X|r}
\toprule
Data & Example & Size \\
\midrule
\ff{bytes} & \ff{1F-E5-77-A6} & 4 \\
\ff{string} & \ff{"Hello, \foreignlanguage{russian}{друг}!"} & 16 \\
  & \ff{"\textbackslash{}u5BB6"} or \begin{CJK}{UTF8}{gbsn}\ff{"家"}\end{CJK} & 2 \\
\ff{number} & \ff{1024}, \ff{0x1A7E}, or \ff{-42.133e14} & 8 \\
\bottomrule
\end{tabularx}
\figcap{The syntax of all data with examples. The ``Size'' column
denotes the number of bytes in the \ff{as-bytes} attribute.
UTF-8 is the encoding used in \ff{string} object.}
\label{tab:types}
\end{table}

\subsection{Tuples}

The \ff{tuple} is yet another syntax sugar, for arrays:

\begin{ffcode}
* "Lucy" "Jeff" 3.14 (*@\label{ln:tuple-1}@*)
* > a (*@\label{ln:tuple-2a}@*)
  (* "a")
  TRUE (*@\label{ln:tuple-2b}@*)
* > b (*@\label{ln:tuple-3}@*)
\end{ffcode}

The code at \lref{tuple-1} makes a tuple of three elements: two strings
and one float. The code at \lrefs{tuple-2a}{tuple-2b} makes a tuple named \ff{a} with another
tuple as its first element and \ff{TRUE} as the second item.
The code at \lref{tuple-3} is an empty tuple with the name \ff{b}.

\subsection{Scope Brackets}

Brackets can be used to group object arguments in horizontal mode:

\begin{ffcode}
sum (div 45 5) 10  (*@\label{ln:sum}@*)
\end{ffcode}

The \ff{(div 45 5)} is a copy of the abstract object \ff{div}
with two arguments \ff{45} and \ff{5}. This object itself is
the first argument of the copy of the object \ff{sum}. Its second
argument is \ff{10}. Without brackets the syntax would read differently:

\begin{ffcode}
sum div 45 5 10
\end{ffcode}

This expression denotes a copy of \ff{sum} with four arguments.

\subsection{Inner Objects}

An object may have other abstract objects as its attributes, for example:

\begin{ffcode}
# A point on a 2D canvas.
[x y] > point
  [to] > distance
    length. > len (*@\label{ln:vector-length}@*)
      vector
        to.x.minus (^.x)
        to.y.minus (^.y)
\end{ffcode}

The object \ff{point} has two void attributes \ff{x} and \ff{y}
and the attribute \ff{distance}, which is attached to an abstract
object with one void attribute \ff{to} and one attached attribute \ff{len}.
The \emph{inner} abstract object \ff{distance} may only be copied
with a reference to its \emph{parent} object \ff{point}, via
a special attribute denoted by the \ff{\^{}} symbol:

\begin{ffcode}
distance. (*@\label{ln:point-copy}@*)
  point
    5:x
    -3:y
  point:to
    13:x
    3.9:y
\end{ffcode}

The parent object is \ff{(point 5 -3)}, while the object \ff{(point 13 3.9)}
is the argument for the void attribute \ff{to} of the object \ff{distance}.
Suffixes \ff{:x}, \ff{:y}, and \ff{:to} are optional and may be used
to denote the exact name of the void attribute to be attached to the
provided argument.

\subsection{Decorators}

An object may extend another object by \emph{decorating} it:

\begin{ffcode}
# A circle with a center and radius.
[center radius] > circle (*@\label{ln:circle}@*)
  center > @ (*@\label{ln:circle-phi}@*)
  [p] > is-inside
    lte. > @
      ^.@.distance $.p (*@\label{ln:circle-phi-2}@*)
      ^.radius (*@\label{ln:circle-end}@*)
\end{ffcode}

The object \ff{circle} has a special attribute \ff{@}
at \lref{circle-phi}, which denotes
the \emph{decoratee}: an object to be extended,
also referred to as ``component'' by~\citet{gamma1994design}.

The \emph{decorator} \ff{circle}
has the same attributes as its decoratee \ff{center}, but also
its own attribute \ff{is-inside}. The attribute \ff{@} may be used
the same way as other attributes, including in dot notation, as it is done
at \lref{circle-phi-2}. However, this line
may be re-written in a more compact way, omitting the explicit
reference to the \ff{@} attribute, because all attributes
of the \ff{center} are present in the \ff{circle};
and omitting the reference to \ff{\$} because the default scope of visibility of
\ff{p} is the object \ff{is-inside}:

\begin{ffcode}
^.distance p
\end{ffcode}

The inner object \ff{is-inside} also has the \ff{@} attribute: it
decorates the object \ff{lte} (stands for ``less than equal'').
The expression at \lref{circle-phi-2} means:
take the parent object of \ff{is-inside},
take the attribute \ff{@} from it, then take the inner object \ff{distance}
from there, and then make a copy of it with the attribute \ff{p}
taken from the current object (denoted by the \ff{\$} symbol).

The object \ff{circle} may be used like this, to understand whether
the \((0,0)\) point is inside the circle at \((-3,9)\) with the radius \(40\):

\begin{ffcode}
circle (point -3 9) 40 > c  (*@\label{ln:circle-c}@*)
c.is-inside (point 0 0) > i
\end{ffcode}

Here, \ff{i} will be a copy of \ff{bool} behaving like \ff{TRUE}
because \ff{lte} decorates \ff{bool}.

It is also possible to make decoratee void, similar to other void
attributes, specifying it in the list of void attributes in
square brackets.

\subsection{Anonymous Formations}

A formation may be used as an argument of another object when
making a copy of it, for example:

\begin{ffcode}
(dir "/tmp").walk
  *
    [f]
      f.is-dir > @
\end{ffcode}

Here the object \ff{walk} is copied with a single argument:
the one-item tuple, which is a formation with a single void attribute \ff{f}. The \ff{walk}
will use this formation, which does not have a name,
to filter out files while traversing the directory tree. It will
make a copy of the formation and then treat it as a boolean
value to make a decision about each file.

The syntax may be simplified and the formation may be inlined
(the tuple is also inlined):

\begin{ffcode}
(dir "/tmp").walk
  * ([f] (f.is-dir > @))
\end{ffcode}

An anonymous formation may have multiple attributes:

\begin{ffcode}
[x] (x.plus 1 > succ) (x.minus 1 > prev)
\end{ffcode}

This object has two attributes \ff{succ} and \ff{prev}, and does not
have a name.

The parent of each copy of the abstract object will be set by
the object \ff{walk} and will point to the \ff{walk} object itself.

\subsection{Constants}

\eolang{} is a declarative language with lazy evaluations. This means
that this code would read the input stream two times:

\begin{ffcode}
# Just say hello.
[] > hello
  QQ.io.stdout > say
    QQ.txt.sprintf
      "The length of %s is %d"
      QQ.io.stdin.next-line > x!
      x.length
\end{ffcode}

The \ff{sprintf} object will go to the \ff{x} two times. The first time,
to use it as a substitute for \ff{\%s}, and the second time for
\ff{\%d}. There will be two round-trips to the standard input stream, which
is obviously not correct. The exclamation mark at the \ff{x!} solves the
problem, making the object by the name \ff{x} a \emph{constant}. This means
that on the first trip to \ff{x}, it turns into bytes.

Here, \ff{x} is an attribute of the object \ff{hello}, even though
it is not defined as explicitly as \ff{say}. Wherever a new
name appears after the \ff{>} symbol, it is a declaration of a new
attribute in the nearest object abstraction.

\subsection{Metas}

A program may have an optional list of \emph{meta} statements,
which are passed to the compiler as is. The meaning of them depends
on the compiler and may vary
between target platforms. This program instructs the compiler
to put all objects from the file into the package \ff{org.example}
and helps it resolve the name \ff{stdout}, which is external
to the file:

\begin{ffcode}
+package org.example
+alias org.eolang.io.stdout

# A simple app.
[args] > app
  stdout > @
    "Hello, world!\n"
\end{ffcode}

\subsection{Atoms}

Some objects in \eolang{} programs may need to be platform-specific
and cannot be composed from other existing objects---they are called
\emph{atoms}.
For example, the object \ff{app} uses the object \ff{stdout},
which is an atom. Its implementation would be provided by the
runtime. This is how the object may be defined:

\begin{ffcode}
+rt jvm org.eolang:eo-runtime:0.7.0
+rt ruby eolang:0.1.0

[text] > stdout ? (*@\label{ln:stdout}@*)
\end{ffcode}

The \ff{?} suffix informs the compiler that this object must
not be compiled from \eolang{} to the target language. The object
with this suffix already exists in the target language and most
probably could be found in the library specified by the \ff{rt}
meta. The exact library to import has to be selected by the compiler.
In the example above, there are two libraries specified: for JVM and
for Ruby.

Atoms in \eolang{} are similar to ``native'' methods in Java
and ``\nospell{extern}'' methods in C\#: this mechanism is also
known as foreign function interface (FFI).


\section{Calculus}
\label{sec:calculus}
% SPDX-FileCopyrightText: Copyright (c) 2016-2025 Objectionary.com
% SPDX-License-Identifier: MIT

In this section we introduce \phic{}, a formalism that we use later to optimize object-oriented code.
We give an informal introduction to the calculus, helping the reader to grasp the idea without diving into details.
In \cref{app:syntax}, we formulate the syntax using EBNF and introduce individual concepts of the calculus, explaining their semantics and illustrating them with examples.

An \emph{object} is a collection of \emph{attributes}, which are uniquely named pairs, for example:
\begin{phiquation}
\label{eq:price-color}
[[ price -> ?, color -> [[ D> FF-C0-CB ]] ]].
\end{phiquation}
This is a \emph{formation} of an object with two attributes |price| and |color|.
The object is \emph{abstract} because the |price| attribute is \emph{void}, i.e. there is no object \emph{attached} to it.
The |color| attribute is attached to another object formation with one \(\Delta\)-\emph{asset}, which is attached to \emph{data} (three bytes).
Objects do not have \emph{names}, while only the attributes, which objects are attached to, do.

A \emph{full application} of an abstract object to a pair of attribute and object \emph{results} in a new \emph{closed} object, for example:
\begin{phiquation}
\label{eq:simple-application}
[[ |x| -> ? ]]( |y| -> b_2 ) \trans [[ |x| -> b_2 ]].
\end{phiquation}
A \emph{partial} application results in a new abstract object, for example:
\begin{phiquation*}
[[ |x| -> ?, |y| -> ? ]]( |x| -> b ) \trans [[ |x| -> b, |y| -> ? ]].
\end{phiquation*}

An object may be \emph{dispatched} from another object with the help of \emph{dot notation} where the right side \emph{accesses} the left side (underlined), for example:
\begin{phiquation}
\label{eq:dot-notation}
[[ |x| -> \underline{\xi.|p|(|t| -> b)}.|y|, |p| -> [[ |t| -> ?, |y| -> \xi.|t| ]] ]].|x| \trans b.
\end{phiquation}
The leftmost symbol \(\xi\) denotes the \emph{scope} of the object,
which is the object formation itself; the following \(.|p|\) part
retrieves the object attached to the attribute \(|p|\);
then, the application $(|t|->b)$ makes a copy of the object;
finally, the \(.|y|\) part retrieves object \(b\).

Each object has a \emph{parent} attribute \(\rho\), which is attached to the
formation from where the object was dispatched.

Attributes are \emph{immutable}, i.e. an application of a binding to an object, where the attribute is already attached, results in a \emph{terminator} denoted as \(\dead\) (runtime error), for example:
\begin{phiquation*}
[[ |x| -> b_1 ]]( |x| -> b_2 ) \trans \dead{}.
\end{phiquation*}

An object may be textually reduced, for example:
\begin{phiquation*}
[[ |x| -> ?, |y| -> ? ]]( |x| -> b_1 )( |y| -> b_2 ) \trans
  \trans [[ |x| -> b_1, |y| -> ? ]] ( |y| -> b_2 ) \trans
  \trans [[ |x| -> b_1, |y| -> b_2 ]].
\end{phiquation*}
The object on the left is reduced to the object on the right in two reduction steps.
Every object has a \emph{normal form}, which is a form that
has no more possible applicable reductions.

An object is called a \emph{decorator} if it has the \(\varphi\) attribute
with an object attached to it, known as a \emph{decoratee}. An attribute
dispatched from a decorator reduces to the same attribute dispatched from the decoratee,
if the attribute is not present in the decorator, for example:
\begin{phiquation*}
[[ @ -> [[ |x| -> b ]] ]].|x| \trans b.
\end{phiquation*}

An object may have data attached only to its \(\Delta\)-asset, for example:
\begin{phiquation*}
[[ |x| -> [[ D> 00-2A ]] ]].
\end{phiquation*}

An object may have a \emph{function} attached to its \(\lambda\)-asset.
Such an object is referred to as \emph{atom}.
An atom may be \emph{morphed} to another object by evaluating
its function with \begin{inparaenum}[1)]
    \item the object and
    \item the state of evaluation
\end{inparaenum}
as arguments, for example:
\begin{phiquation}
\label{eq:Sqrt}
\frac \
  { \vdash Sqrt (b, s) \to \langle [[ D> \sqrt{\mathbb{D}(b.|x|)} ]], s \rangle } \
  {[[ L> Sqrt, |x| -> 256 ]] \to 16}.
\end{phiquation}
There is also a \emph{dataization} function \(\mathbb{D}(b, s)\) that normalizes its first argument
and then either returns the data attached to the \(\Delta\)-asset of
the normal form or morphs the atom into an object and dataizes it (recursively),
for example:
\begin{phiquation*}
\mathbb{D}([[ |x| -> 42 ]].|x| \trans 42, s) \to 42.
\end{phiquation*}

A program is an object that is attached to the \(\Phi\) attribute
of the \emph{Universe}, for example:
\begin{phiquation*}
Q -> [[ D> CA-FE ]].
\end{phiquation*}
The dataization of \(\Phi\) is the evaluation of the program.

\subsection{Syntax}\label{sec:syntax}

The syntax of a program is defined by EBNF in \cref{fig:ebnf} (the starting symbol is \EbnfNonTerminal{Program}).

\begin{figure*}
\begin{mdframed}
\raggedright
\begin{ebnf}[8em]
<Program> := "\(\Phi\)" "\(\mapsto\)" <Expression> \\
<Expression> := <Formation> | <Application> | <Dispatch> | "\(\dead\)" \\
<Formation> := "\(\llbracket\)" <Binding> "\(\rrbracket\)" \\
<Application> := <Expression> "\(\lparen\)" <A-Pair> "\(\rparen\)" \\
<A-Pair> := <\(\tau\)-Pair> | <\(\alpha\)-Pair> \\
<Dispatch> := <Subject> "." <Attribute> \\
<Subject> := <Expression> | <Locator> \\
<Locator> := "\(\Phi\)" | "\(\xi\)" \\
<Binding> := <Pair> <Bindings> | \(\epsilon\) \\
<Bindings> := "," <Pair> <Bindings> | \(\epsilon\) \\
<Pair> := <\(\varnothing\)-Pair> | <\(\tau\)-Pair> | <\(\Delta\)-Pair> | <\(\lambda\)-Pair> \\
<\(\varnothing\)-Pair> := <Attribute> "\(\mapsto\)" "\(\varnothing\)" \\
<\(\tau\)-Pair> := <Attribute> "\(\mapsto\)" <Subject> \\
<\(\alpha\)-Pair> := <Alpha> "\(\mapsto\)" <Subject> \\
<\(\Delta\)-Pair> := "\(\Delta\)" "\(\phiDotted\)" <Data> \\
<\(\lambda\)-Pair> := "\(\lambda\)" "\(\phiDotted\)" <Function> \\
\end{ebnf}
\end{mdframed}
\caption{Syntax as a context-free grammar, in EBNF.}
\label{fig:ebnf}
\end{figure*}

Besides the literals mentioned in the grammar in blue color, the
alphabet includes three non-terminals that rewrite to terminals as follows:
\begin{itemize}
  \item \EbnfNonTerminal{Attribute}: either \begin{inparaenum}[1)]
      \item Greek letter \(\varphi\),
      \item Greek letter \(\rho\),
      or
      \item a string of lowercase English letters possibly with dashes inside, e.g. ``|price|'' or ``|a-car|'';
  \end{inparaenum}
  \item \EbnfNonTerminal{Data}: a sequence of bytes in hexadecimal format, e.g. ``\texttt{EF-41-5C}'' is a sequence of three bytes, ``\texttt{42-}'' is a one-byte sequence (with a trailing dash in order to avoid confusion with integers), and ``\texttt{-{}-}'' (double dash) is an empty sequence of bytes;
  \item \EbnfNonTerminal{Function}: a string of English letters where the first letter is in uppercase, e.g. ``|Sqrt|'';
  \item \EbnfNonTerminal{Alpha}: a Greek letter \(\alpha\) with a non-negative whole-number index, e.g. \(\alpha_2\).
\end{itemize}

\subsection{Expression}\label{sec:expression}

\begin{definition}[Expression]
\textbf{Expression}, ranged over \(\mathcal{E}\) by \(e_i\),
is a grammatical construct that may result in an object after
certain transformations.
\end{definition}

\subsection{Attribute}\label{sec:attribute}

\begin{definition}[Attribute]
\textbf{Attribute}, ranged over \(\mathcal{T}\) by \(\tau_i\),
is an identifier to which either ``\stx{\varnothing}'' or an expression
may be attached.
\end{definition}

\begin{definition}[Void vs. Attached]
An attribute is \textbf{void} if it is attached to ``\stx{\varnothing},''
otherwise it is an \textbf{attached} attribute.
\end{definition}

\subsection{Data}\label{sec:data}

\begin{definition}[Data]
\textbf{Data}, ranged over \(\mathcal{D}\) by \(\delta_i\), is a possibly
empty sequence of 8-bit bytes.
\end{definition}

\subsection{Binding}\label{sec:binding}

\begin{definition}[Binding]
\textbf{Binding}, ranged over \(\mathcal{G}\) by \(B\), is a possibly empty set of key-value pairs,
denoted as \( k_1 \to v_1, k_2 \to v_2, \dots, k_n \to v_n \), where all keys are unique.
\end{definition}

The predicate \(k \in B\) holds if \(k\) is present in any pair of \(B\).

\subsection{Function}\label{sec:function}

\begin{definition}[Function]
\textbf{Function} is a total mapping
\(\langle \mathcal{G}, \mathcal{S} \rangle \to \langle \mathcal{G}, \mathcal{S} \rangle\)
that maps binding to binding, possibly modifying the \emph{state} of evaluation \(s_i\),
which itself ranges over \(\mathcal{S}\).
\end{definition}

\subsection{Asset}\label{sec:asset}

\begin{definition}[Asset]
\textbf{Asset} is an identifier to which either data
(denoted as ``\stx{\Delta}''-asset) or function
(denoted as ``\stx{\lambda}''-asset) is attached.
\end{definition}

\subsection{Object}\label{sec:object}

\begin{definition}[Object]
\textbf{Object}, ranged over \(\mathcal{B}\) by \(b_i\), is a binding
where keys are either attributes or assets and values are either
``\stx{\varnothing},'' expressions, data, or functions.
\end{definition}

The following is an example of an object with four pairs, where the first one
is an asset attached to data, while the other three are attributes attached to
expressions:
\begin{phiquation}
\label{eq:object-example}
[[ D> 00-2A, |a| -> b_2(0-> b_3).|bar|, |b| -> [[ L> Sqrt ]], foo -> \dead ]]
\end{phiquation}

The arrow ``\stx{\mapsto}'' denotes an attachment of an expression (right-hand side)
to an attribute (left-hand side). The arrow ``\stx{\phiDotted}'' denotes
an attachment of data or function to an asset.

\subsection{Domain}\label{sec:domain}

\begin{definition}[Domain]
\textbf{Domain} of object \(b\), denoted as \(\bar{b}\), is a set
that includes all attributes of \(b\).
\end{definition}

The domain of the object in \cref{eq:object-example} is \(\{ |a|, |b|, |foo| \}\).
Assets ``\stx{\Delta}'' and ``\stx{\lambda}'' do not belong to object domain.

\subsection{Formation}\label{sec:formation}

\begin{definition}[Formation]
Object \textbf{formation}, denoted as ``\(\stx{\llbracket} B \stx{\rrbracket}\)'',
is a construction of a new object.
\end{definition}

We introduce the term ``object formation'' rather than using a more traditional
``construction'' term because the latter generally implies a presence of a
class from which an object is being constructed or instantiated. Instead,
object formation is closer to the creation of a prototype, which may either be
used ``as is'' or copied.

\subsection{Program}\label{sec:program}

\begin{definition}[Program]
\textbf{Program} is a pair $ Q -> [[ B ]] $ that exists only
in the \textit{Universe} binding.
\end{definition}

The Universe resembles an object with only the attribute \(\Phi\), but it
is not an object since it cannot be attached to any attribute of any other object.

\subsection{Abstract Object}\label{sec:abstract}

\begin{definition}[Abstract Object]
An object is \textbf{abstract} if at least one of its attributes is void,
otherwise the object is \textbf{closed}.
\end{definition}

\Cref{eq:price-color} is an example of an object formation, where the binding of
the abstract object being formed consists of two pairs: $price -> ?$ and
$color -> [[ D> FF-C0-CB ]]$. The object attached to the |color| attribute is a
formation of a closed object.

\subsection{Ordinal}\label{sec:ordinal}

\newcommand\ordinal[2]{#1 \circ #2}
\begin{definition}[Ordinal]
An attribute's \textbf{ordinal}, denoted as \(\ordinal{\tau}{b}\),
is a non-negative whole number that is equal to the position of \(\tau\)
in \(\mathbb{F}(b)\), starting from zero, not counting assets.
\end{definition}

\begin{example}
\Cref{tab:ordinals} shows a few examples of attributes and their ordinals.

\begin{table*}
\caption{A few examples of attributes' ordinals in their objects.}
\label{tab:ordinals}
\begin{tabular}{lr}
\toprule
\(b\) & \(\ordinal{|x|}{b}\) \\
\midrule
$[[ |x| -> \xi.|k| ]]$
  & 0 \\
$[[ |foo| -> \xi.|k|, |x| -> \Phi.|t| ]]$
  & 1 \\
$[[ L> Fn, |x| -> \xi.|k| ]]$
  & 0 \\
$[[ D> CA-FE, |foo| -> \xi, |x| -> \xi.|foo| ]]$
  & 2 \\
\bottomrule
\end{tabular}
\end{table*}
\end{example}

\subsection{Application}\label{sec:application}

\begin{definition}[Application]
Object \textbf{application}, denoted as \( b \stx{(} \tau \;\stx{\mapsto}\; e \stx{)} \), is a copy of an existing abstract object \(b\) (the ``subject''), with \(e\) attached to its \(\tau\) attribute.
\end{definition}

\Cref{eq:simple-application} demonstrates object application, where $ |a| -> b_2 $ is
applied to the formation of an abstract object $[[ |a| -> ? ]]$. The application creates
a new object $[[ |a| -> b_2 ]]$, while the existing abstract object remains intact.

The object in \cref{eq:price-color} is abstract, because its attribute |price| is void.
The object in \cref{eq:simple-application} was abstract before the application, but the object created by the application is closed since its attribute |a| is not void (attached to \(b_2\)).

Even though \stx{\varnothing} may be attached to an attribute of an object,
it is not an object by itself. Instead, \(\varnothing\) is a ``placeholder''
for an object, which remains attached to an attribute until an object is attached to it.
Even though this mechanism resembles NULL references, there is a significant
difference: in \phic{}, void attributes may be attached to objects only once,
while any further reattachments are prohibited.

In $b(\alpha_i -> e)$ application, \(e\) must be attached to the attribute \(\tau\) of \(b\) for which \(\ordinal{\tau}{\mathbb{F}(b)}\) equals \(i\).

\subsection{Forma}\label{sec:forma}

\begin{definition}[Forma]
A \textbf{forma} of an object \(b\), denoted as \(\mathbb{F}(b)\), is an abstract
object that was copied in order to create \(b\).
A forma of a formation is the formation itself.
\end{definition}

In \cref{eq:simple-application}, the forma of $[[ |a| -> b_2 ]]$ is the abstract
object $[[ |a| -> ? ]]$, while the forma of $[[ |a| -> ? ]]$ is itself.

\subsection{Dispatch}\label{sec:dispatch}

\begin{definition}[Dispatch]
Object \textbf{dispatch} (also known as ``dot notation''), denoted as
\( b \stx{.} \tau \) where \(b\) is the ``subject'', means retrieval of the object
attached to the \(\tau\) attribute of \(b\).
\end{definition}

\subsection{Scope}\label{sec:scope}

\begin{definition}[Scope]
The \textbf{scope} of expression \(e\), denoted as \(e^\varsigma\) is either
\begin{inparaenum}[a)]
\item the formation where the expression is attached to an attribute,
\item or the scope of the expression where \(e\) is used.
\end{inparaenum}
\end{definition}

In simpler terms, the scope is the formation that is the ``closest'' to the pair,
moving to the left in the expression. In~\cref{eq:simple-scope}, the scope of |author| is the
formation where the |source| attribute stays, while the scope of |cite|
is the formation where the |ref| attribute stays.
\begin{phiquation}
\label{eq:simple-scope}
  [[ ref -> \uplace{}{ [[ source -> Q.book( author -> b_1 ) ]] }, cite -> b_2 ]]
\end{phiquation}

\Cref{fig:scopes} illustrates the concept of scope of a pair.

\begin{figure*}
\begin{mdframed}
\begin{phiquation*}
[[ |a| -> \uplace{}{ [[ |y| -> b_2.|t|( |f| -> [[ |z| -> b_3 ]] ( |x| -> e ) ) ]] } ]]

[[ |f| -> \uplace{}{ [[ |k| -> b_1( |x| -> e ) ]] } ]]

[[ |k| -> [[ |y| -> b( |f| -> \uplace{}{ [[ |x| -> e ]] } ) ]] ]]

[[ |k| -> \uplace{}{ [[ |f| -> [[ |x| -> ? ]] ( |x| -> b_2 ) ( |x| -> b_3 ) ( |x| -> e ) ]] } ]]
\end{phiquation*}
\end{mdframed}
\label{fig:scopes}
\caption{Illustrative examples of scopes: the bars over the terms highlight \(e^\varsigma\), the scope of \(e\).}
\end{figure*}

\subsection{Locators}\label{sec:locators}

In an expression, the locator ``\stx{\Phi}'' means the program,
while the locator ``\stx{\xi}'' means the scope of the expression.

\subsection{Head and Tail}

\begin{definition}
Since any expression may recursively be defined as either \begin{inparaenum}[1)]
    \item \(\dead\),
    \item formation,
    \item application,
    or
    \item dispatch,
\end{inparaenum}
it consists of a \textbf{head} and a possibly empty \textbf{tail}, denoted together as \(h\bullet{}t\).
\end{definition}

\begin{example}
\Cref{tab:head-and-tail} shows a few examples that demonstrate the separation
between a head and a possibly empty tail of an expression.

\begin{table*}
\caption{A few examples of objects that demonstrate the separation between a head and a possibly empty tail of an expression.}
\label{tab:head-and-tail}
\begin{tabular}{lll}
\toprule
Expression & Head & Tail \\
\midrule
$b_1( foo -> b_2 )$
  & $b_1( foo -> b_2 )$
  & --- \\
$b_1.\alpha_1.\alpha_2( 0-> b_2 )$
  & $b_1$
  & $\alpha_1.\alpha_2( 0-> b_2 )$ \\
$[[ |a| -> b_2 ]].|a|.|test|.|b|$
  & $[[ |a| -> b_2 ]]$
  & $\alpha_0.|test|.\alpha_2$ \\
$b_1( |foo| -> b_2)( 0-> 42 ).|print|( 1-> 7 )$
  & $b_1( |foo| -> b_2)( 0-> 42 )$
  & $|print|( 1-> 7 )$ \\
\bottomrule
\end{tabular}
\end{table*}
\end{example}

\subsection{Terminator}\label{sec:terminator}

\begin{definition}[Terminator]
The \textbf{terminator}, denoted as \stx{\dead}, is an object that equals itself when it is
the subject of a dispatch or an application:
\begin{equation*}
\forall \tau : \dead.\tau \trans \dead \qquad \forall \tau, e : \dead( \tau \mapsto e) \trans \dead.
\end{equation*}
\end{definition}

\subsection{Immutability}\label{sec:immutability}

\begin{definition}[Immutability]
Every object is \textbf{immutable}, meaning that an application of
its already attached attribute equals to \(\dead\):
\begin{equation*}
\forall \tau, e_1, e_2 : \llbracket \tau \mapsto e_1 \rrbracket ( \tau \mapsto e_2 ) \trans \dead
\end{equation*}
\end{definition}

\subsection{Atoms}\label{sec:atoms}

\begin{definition}[Atom]
\textbf{Atom} is an object with a function attached to its \(\lambda\)-asset.
\end{definition}

\Cref{eq:Sqrt} demonstrates an atom with a function that calculates
the square root of a number, which it retrieves from the \(\Delta\)-asset
of \(b.\alpha_0\) with the help of the morphing function (\cref{sec:morphing}).
The implementation of functions is outside the scope of \phic{}: they may be implemented,
for example, in \(\lambda\)-calculus or a programming language
such as Java or C++.

\subsection{Decoration}\label{sec:decoration}

\begin{definition}[Decoration]
Object \textbf{decoration} is a mechanism of extending an object (``decoratee'')
by attaching it to the ``\stx{\varphi}''-attribute of another object (``decorator''),
which makes attributes of the decoratee retrievable from the decorator,
unless the decorator has its own attributes with the same names.
\end{definition}

\subsection{Syntactic Sugar}

\Cref{tab:sugar} shows all possible syntax sugar.

\begin{table*}
\caption{Syntax sugar.}
\label{tab:sugar}
\newcommand\sugar[2]{$ #1 $ & $ #2 $ \\}
\newcommand\subs[1]{& \textcolor{gray}{(#1)} \\}
\newcommand\tto{\;\stx{\mapsto}\;}
\begin{tabular}{ll}
\toprule
Syntax sugar & Its more verbose equivalent \\
\midrule
\sugar
  {\stx{ QQ }}
  {\stx{ Q } \stx{.} |org| \stx{.} |eolang|}
\sugar
  {e \stx{(} \tau_1 \tto e_1, \tau_2 \tto e_2, \dots \stx{)}}
  {e \stx{(} \tau_1 \tto e_1 \stx{)}\stx{(} \tau_2 \tto e_2 \stx{)} \dots}
\sugar
  {e \stx{(} e_0 \stx{,}\; e_1 \stx{,}\; \dots \stx{)}}
  {e \stx{(} \alpha_0 \tto e_0 \stx{,}\; \alpha_1 \tto e_1 \stx{,}\; \dots \stx{)}}
\sugar
  {\tau_1 \stx{(} \tau_2 \stx{,}\; \tau_3 \stx{,}\; \dots \stx{)} \tto \stx{[[} B \stx{]]}}
  {\tau_1 \tto \stx{[[} \tau_2 \tto \stx{?} \stx{,}\; \tau_3 \tto \stx{?} \stx{,}\; \dots \stx{,}\; B \stx{]]}}
\sugar
  {\tau_1 \tto \tau_2}
  {\tau_1 \tto \stx{\xi}\stx{.}\tau_2}
\sugar
  {\stx{[[} B \stx{]]}}
  {\stx{[[} B \stx{,}\; \rho \tto \stx{?} \stx{]]} \quad\text{if}\; \rho \notin B}
\sugar
  {\tau \;\stx{\mapsto}\; \texttt{\begin{CJK}{UTF8}{gbsn}"你好"\end{CJK}}}
  {\tau \tto \stx{ QQ } \stx{.} |string| \stx{(} \stx{ QQ } \stx{.} |bytes| \stx{(} \stx{[[} \stx{ D> }\; |E4-BD-A0-E5-A5-BD| \stx{]]} \stx{)} \stx{)}}
  \subs{UTF-8 string}
\sugar
  {\tau \tto 42}
  {\tau \tto \stx{ QQ } \stx{.} |number| \stx{(} \stx{ QQ } \stx{.} |bytes| \stx{(} \stx{[[} \stx{ D> }\; |40-45-00-00-00-00-00-00| \stx{]]} \stx{)} \stx{)}}
  \subs{eight bytes per integer}
\sugar
  {\tau \tto 3.14}
  {\tau \tto \stx{ QQ } \stx{.} |number| \stx{(} \stx{ QQ } \stx{.} |bytes| \stx{(} \stx{[[} \stx{ D> }\; |40-09-1E-B8-51-EB-85-1F| \stx{]]} \stx{)} \stx{)}}
  \subs{eight bytes per number with a floating point}
\sugar
  {\stx{\Big\{} e \stx{\Big\}}}
  {\stx{ Q } \tto \stx{[[} e \stx{]]}}
\bottomrule
\end{tabular}
\end{table*}

\newpage
\subsection{Contextualization}\label{sec:contextualization}
\newcommand\ctx[2]{\lceil #1 \;\textcolor{gray}{\shortmid}\; #2 \rfloor}

\begin{definition}[Contextualization]
A contextualization function \(\mathbb{C} : \mathcal{E} \times \mathcal{B} \to \mathcal{E}\) denoted as \( \ctx{e}{b} \), which replaces locators with objects, is defined by induction:
\begin{enumerate}[label=\(\mathbb{C}_\arabic*:\),ref=\ensuremath{\mathbb{C}.\arabic*}]
  \item\label{C:Phi} $ \ctx{\Phi}{b} \trans e $ \quad if $ \Phi -> e $,
  \item\label{C:xi} $ \ctx{\xi}{b} \trans b $,
  \item\label{C:forma} $ \ctx{[[ B ]]}{b} \trans [[ B ]] $,
  \item\label{C:dead} $ \ctx{\dead}{b} \trans \dead $,
  \item\label{C:dot} $ \ctx{e.\tau}{b} \trans \ctx{e}{b}.\tau $,
  \item\label{C:app} $ \ctx{e_1( \tau -> e_2 )}{b} \trans \ctx{e_1}{b}( \tau -> \ctx{e_2}{b} ) $.
\end{enumerate}
\end{definition}

\Cref{app:contextualization-examples} demonstrates how contextualization function works, by examples.

\subsection{Normalization}\label{sec:normalization}

An expression that may be rewritten by the \emph{rules} (or \emph{reductions})
listed in \cref{fig:reduction} is a \emph{reducible} expression.
The notation \(e_1 \trans e_2\), optionally followed by a condition,
denotes a reduction of \(e_1\) to \(e_2\), if the condition holds.

These rules may be applied in any order.

A specific reduction may be denoted, for example, as \(\trans_{\nameref{r:dot}}\),
or just \(\trans\) if a reduction is meant with no specificity.
The notation \(e_1 \strans e_2\) denotes a reflexive transitive
closure of all reductions, so that there is a possibly empty finite
sequence of reductions between \(e_1\) and \(e_2\).

An expression that has no more possible applications of reductions
is \emph{irreducible} or a \emph{normal form}, denoted as \(\nf{}_i\)
ranging over \(\mathcal{N} \in \mathcal{E}\). Thus, \(\nf\) is a normal
form of \(e\) if \(e \strans \nf\) and there is no expression \(e_1\)
such that \(\nf \trans e_1\).

An expression may not have a normal form.

\begin{figure*}
\newcommand\trrule[5][]{%
  \newrule[#1]{#2}:
  &
  { $ #3 $ }
  \(\trans\)
  { $ #4 $ }
  \quad #5
  \\%
}
\begin{mdframed}
\renewcommand{\arraystretch}{1.2}
\begin{tabular}{rl}
\trrule{copy}
  { [[ B_1, \tau -> ?, B_2 ]] ( \tau -> e ) }
  { [[ B_1, \tau -> n, B_2 ]] }
  { if \( \ctx{e}{e\textsuperscript{\(\varsigma\)}} \strans n \) }
\trrule[\alpha]{alpha}
  { [[ B_1, \tau -> ?, B_2 ]] ( \alpha_i -> e ) }
  { [[ B_1, \tau -> ?, B_2 ]] ( \tau -> e) }
  { if $ \ordinal{\tau}{[[B]]} = i $ }
\trrule{dot}
  { [[ B_1, \tau -> \nf, B_2 ]].\tau }
  { \ctx{\nf}{n\textsuperscript{\(\varsigma\)}} ( \rho -> n\textsuperscript{\(\varsigma\)} ) }
  { }
\trrule[\varphi]{phi}
  { [[ B ]].\tau }
  { [[ B ]].\varphi.\tau }
  { if \( \tau \notin B\) and \( \varphi \in B\)  }
\trrule{stay}
  { [[ B_1, \rho -> e_1, B_2 ]]( \rho -> e_2 ) }
  { [[ B_1, \rho -> e_1, B_2 ]] }
  { }
\trrule{over}
  { [[ B_1, \tau -> e_1, B_2 ]]( \tau -> e_2) }
  { \dead }
  { if \( \tau \not= \rho \) }
\trrule{stop}
  { [[ B ]].\tau }
  { \dead }
  { if \( [ \tau, \varphi, \lambda ] \cap B = \emptyset \) }
\trrule{null}
  { [[ B_1, \tau -> ?, B_2 ]].\tau }
  { \dead }
  { }
\trrule{miss}
  { [[ B ]] ( \tau -> e ) }
  { \dead }
  { if \( \tau \notin B \) and \( \tau \notin [ \alpha_0, \alpha_1, \dots ] \) }
\trrule{dd}
  { \dead.\tau }
  { \dead }
  { }
\trrule{dc}
  { \dead ( \tau -> e ) }
  { \dead }
  { }
\end{tabular}
\end{mdframed}
\caption{Reduction rules.}
\label{fig:reduction}
\end{figure*}

\Cref{app:normalization-examples} demonstrates how normalization works, by examples.

\subsection{Primitives}\label{sec:primitives}

\begin{definition}[Primitive]
\textbf{Primitive} denoted by \(p\) and ranging over \(\mathcal{P} \in \mathcal{N}\)
is either \(\dead\) or object formation without \(\lambda\) asset.
\end{definition}

\begin{theorem}\label{th:norm-head}
A normal form is either a primitive or a dispatch where the subject is an atom.
\end{theorem}

\begin{proof}
\tbd{The proof is yet to be written.}
\end{proof}

\subsection{Confluence}\label{sec:confluence}

The reduction strategy in \cref{fig:reduction} is \emph{confluent},
i.e., possesses the Church–Rosser property~\citep{church1936some}. This means
that if there are reduction sequences from any expression to two different expressions,
then there exist reduction sequences from those two expressions to some common expression.
Moreover, the reduction strategy possesses the unique normal form property:

\begin{theorem}[Confluence]
Any expression may have only one normal form.
\end{theorem}

\begin{proof}
\tbd{The proof is going to be written soon.}
\end{proof}

\subsection{Equivalence}\label{sec:equivalence}

Two expressions are said to be \emph{syntactically equivalent} or just
\emph{equivalent} (denoted by \(\equiv\)) if their normal forms are
syntactically identical.

\Cref{fig:equivalence} shows some properties that hold.

\begin{figure*}
\begin{mdframed}
\begin{phiquation*}
e(\tau_1 -> e_1)(\tau_2 -> e_2) \equiv e(\tau_2 -> e_2)(\tau_1 -> e_1) \
\quad\text{(by \nameref{r:copy})}
  \text{Commutativity of Application}
\end{phiquation*}
\end{mdframed}
\caption{Equivalence properties that hold.}
\label{fig:equivalence}
\end{figure*}

\subsection{Morphing}\label{sec:morphing}

The \emph{morphing} function \(\mathbb{M} : \langle \mathcal{E}, \mathcal{S} \rangle \to \langle \mathcal{P}, \mathcal{S} \rangle\)
maps expressions to primitives, possibly modifying the \emph{state} of evaluation.
The inference rules at \cref{fig:morphing} inductively describe the algorithm of morphing.

\begin{figure*}
\begin{mdframed}
\begin{phiquation*}
\newrule{prim} \
\frac \
{  } \
{ \langle e, s \rangle \stepto \langle p, s \rangle }  \
\;\text{if}\; e \in \mathcal{P} \
\quad\quad \
\newrule{nmz} \
\frac \
{ \langle \nf, s_1 \rangle \stepto \langle p, s_2 \rangle } \
{ \langle e, s_1 \rangle \stepto \langle p, s_2 \rangle }  \
\;\text{if}\; e \strans \nf \;\text{and}\; e \not\equiv n

\newrule[\lambda]{lambda} \
\frac \
{ \langle e \bullet t, s_1 \rangle \stepto \langle p, s_2 \rangle} \
{ \langle [[ B_1, L> f, B_2 ]] \bullet{} t, s_1 \rangle \stepto \langle p, s_2 \rangle }  \
\;\text{if}\; f( [[ B_1, B_2 ]], s_1 ) \to \langle e, s_2 \rangle

\newrule[\Phi]{Phi} \
\frac \
{ \langle e \bullet{} t, s_1 \rangle \stepto \langle p, s_2 \rangle} \
{ \langle \Phi.\tau \bullet{} t, s_1 \rangle \stepto \langle p, s_2 \rangle }  \
\;\text{if}\; \Phi \mapsto \llbracket B_1, \tau \mapsto e, B_2 \rrbracket
\end{phiquation*}
\end{mdframed}
\caption{Morphing rules.}
\label{fig:morphing}
\end{figure*}

The notation \(\langle e, s_1 \rangle \stepto \langle p, s_2 \rangle\)
means that \(\mathbb{M}(e, s_1)\) evaluates to \(p\), thus \emph{morphing} \(e\)
with a side-effect of changing the state of evaluation \(s_1\) to \(s_2\).

In \nameref{r:lambda}, the function \(f\) is called by value.

\begin{lemma}
The morphing function is not total because not every expression has a normal form.
\end{lemma}

\subsection{Dataization}\label{sec:dataization}

The \emph{dataization} function
\(\mathbb{D} : \langle \mathcal{E}, \mathcal{S} \rangle \rightharpoonup \langle \mathcal{D}, \mathcal{S} \rangle\)
maps expressions to data, possibly modifying the \emph{state} of evaluation.
The inference rules in \cref{fig:dataization} inductively describe the algorithm of dataization.

\begin{figure*}
\begin{mdframed}
\begin{phiquation*}
\newrule[\Delta]{delta} \
\frac \
{ } \
{ \langle [[ B_1, D> \delta, B_2 ]], s \rangle \stepto \langle \delta, s \rangle } \
\quad\quad \
\newrule{norm} \
\frac{ \langle p, s_1 \rangle \stepto \langle \delta, s_2 \rangle }{ \langle e, s_1 \rangle  \stepto \langle \delta, s_2 \rangle } \
\;\text{if}\; \mathbb{M}(e, s_1) \to \langle p, s_2 \rangle

\newrule{box} \
\frac \
{ \langle \ctx{e}{e\textsuperscript{\(\varsigma\)}}, s_1 \rangle \stepto \langle \delta, s_2 \rangle } \
{ \langle [[ B_1, @ -> e, B_2 ]], s_1 \rangle  \stepto \langle \delta, s_2 \rangle } \
\;\text{if}\; [\Delta, \lambda] \cap ( B_1 \cup B_2 ) = \emptyset
\end{phiquation*}
\end{mdframed}
\caption{Dataization rules.}
\label{fig:dataization}
\end{figure*}

The rules \nameref{r:box} and \nameref{r:phi} coexist because
\(\Delta \notin \mathcal{T}\), thus making the expression $ [[ @ -> 42 ]].\Delta $ invalid.

The notation \(\langle e, s_1 \rangle \stepto \langle \delta, s_2 \rangle\)
means that \(\mathbb{D}(e, s_1)\) evaluates to \(\delta\),
thus \emph{dataizing} \(e\) with a side-effect of changing the state of evaluation \(s_1\) to \(s_2\).
The notation \(\mathbb{D}(e)\) is a shortened form of \(\mathbb{D}(e, \varnothing)_{(1)}\),
which is a \emph{pure} dataization --- inputting an empty state of evaluation and ignoring
the output state of evaluation. Here and later,
the notation \(x_{(i)}\) denotes the \(i\)-th element of the tuple \(x\).

\begin{lemma}
The dataization function is partial, because not every primitive contains \(\Delta\)-asset.
\end{lemma}

\Cref{app:dataization-examples} demonstrates how dataization function works, by examples.

\subsection{Parent}

\begin{definition}[Parent]
Attaching expression \(e\) to the \(\rho\) attribute of object \(b\)
means setting the \textbf{parent} of \(b\) to \(e\).
\end{definition}

The presence of ``parent'' in each object is essential for the coordination of inner
objects after dynamic dispatch. Consider the following formation of an abstract object
with an inner object:
\begin{phiquation*}
\Big\{ |x| -> [[ |a| -> ?, next -> [[ @ -> \xi.^.\alpha_0.plus( 1 ), ^ -> Q.|x| ]] ]], |k| -> \xi.|x|( |42| ) \Big\}.
\end{phiquation*}
Here, if the parent of \(\Phi.|x|.|next|\) were attached in the formation, the result of \(\mathbb{D}(\Phi.|k|.|next|)\)
would not be equal to |43|. Instead, it would be equal to \(\dead\), because \(\Phi.|k|.|next|.\rho\)
would still be attached to \(\Phi.|x|\) after the dispatch of \(\Phi.|k|.|next|\).
The parent attribute may be compared with the |this| pointer in Java or C++, which
does not point anywhere until a method of a class is called. Then, when the method
is called, the |this| pointer refers to the object that owns the method.

\subsection{Congruence}

Two expressions \(e_1\) and \(e_2\) are said to be \emph{behaviorally equivalent}
or \emph{congruent} (denoted by \(\cong\)) if for any state \(s\): \(\mathbb{D}(e_1, s) = \mathbb{D}(e_2, s)\).

The following properties hold:
\begin{phiquation*}
\nf \cong [[ @ -> \nf ]] \quad \text{(by \nameref{r:phi})}
  \text{Transparency of Decoration}

[[ \tau_1 -> ?, B]] \cong [[ B ]] \quad \text{(by \nameref{r:stop})}
  \text{Redundancy of Void Attributes}

[[ B ]] \cong [[ B ]].@ \quad\text{if}\; \stx{@} \in B \;\text{and}\; \stx{\Delta} \not\in B \quad \text{(by \nameref{r:phi} and \nameref{r:dot})}
  \text{Implicitness of Decoration}
\end{phiquation*}

Two congruent expressions may be non-equivalent, for example:
\begin{phiquation*}
\Big\{ \tau_1 -> [[ foo -> ?, D> 01-02 ]], \tau_2 -> [[ bar -> ?, D> 01-02 ]] \Big\}
Q.\tau_1 \cong Q.\tau_2 \;\not\to\; Q.\tau_1 \equiv Q.\tau_2.
\end{phiquation*}

% \subsection{Dataless Objects}

% \begin{definition}[\(\Delta\)-depth]
% \(\Delta\)-depth of an object describes how deep data is in the object when recursively traversing values attached to the object attributes.
% More specifically:
% \begin{enumerate}
%   \item the \(\Delta\)-depth of an object with \(\Delta\)-asset is \(1\);
%   \item the \(\Delta\)-depth of an empty object is \(\infty\);
%   \item otherwise, the \(\Delta\)-depth of an object is:
%     \begin{inparaenum}
%       \item \(1 + M\), where \(M\) is the minimal \(\Delta\)-depth among objects attached to its attributes;
%       \item \(\infty\) if no objects are attached to its attributes;
%     \end{inparaenum}.
% \end{enumerate}
% \end{definition}

% \begin{definition}[Dataless Object]
% An object is \emph{dataless} if its \(\Delta\)-depth is greater than 2.
% \end{definition}

% \begin{example}
% \Cref{tab:depths} demonstrates objects with their \(\Delta\)-depths.
% \begin{table*}
% \caption{A few examples of objects and their \(\Delta\)-depths.}
% \label{tab:depths}
% \begin{tabular}{lll}
% \toprule
% Formation                                               & \(\Delta\)-Depth & Dataless \\
% \midrule
% $[[  ]]$                                                & \(\infty\)       & Yes      \\
% $[[ |a| -> ? ]]$                                           & \(\infty\)       & Yes      \\
% $[[ D> |42-43| ]]$                                      & 1              & No       \\
% $[[ bar -> [[ D> ? ]] ]]$                               & \(\infty\)       & Yes      \\
% $[[ foo -> [[ D> 01-02-03 ]] ]]$                        & 2              & No       \\
% $[[ D> |42-|, L> Fn ]]$                                 & 1              & No       \\
% $[[ L> Fn ]]$                                           & \(\infty\)       & Yes      \\
% $[[ |a| -> ?, x -> [[ y -> ?, z -> [[ D> |42-01| ]] ]] ]]$ & 3              & Yes      \\
% \bottomrule
% \end{tabular}
% \end{table*}
% \end{example}


\section{Key Features}
\label{sec:features}
% SPDX-FileCopyrightText: Copyright (c) 2024-2025 Yegor Bugayenko
% SPDX-License-Identifier: MIT

\newcounter{feature}
\renewcommand\thefeature{\thesection.\arabic{feature}}
\newcommand\feature[2]{{\refstepcounter{feature}\label{feature:#1}\textbf{%
  %\thefeature.
  #2%
}.\;}}

There are a few features that distinguish \eolang{} and \phic{}
from other existing OO languages and object theories, while some of
them are similar to what other languages have to offer. The Section is not
intended to present the features formally, which was done earlier in
\cref{sec:calculus,sec:syntax}, but to compare \eolang{} with other
programming languages and informally identify similarities.

\feature{no-classes}{No Classes}
%
\eolang{} is similar to other delegation-based languages like Self~\citep{ungar1987self},
where objects are not created by a class as in class-based languages like C++ or Java,
but from another object, inheriting properties from the original.
However, while in such languages, according to~\citet{fisher1995delegation},
``an object may be created,
and then have new methods added or existing methods redefined,''
in \eolang{} such object alteration is not allowed.

\feature{no-types}{No Types}
%
Even though there are no types in \eolang{}, compatibility
between objects may be inferred in compile-time and validated strictly, which other
\nospell{typeless} languages such as Python,
Julia~\citep{bezanson2012julia},
Lua~\citep{ierusalimschy2016lua},
or Erlang~\citep{erlang2020manual} can't guarantee.
Also, there is no type casting or reflection on types in \eolang{}.

\feature{no-inheritance}{No Inheritance}
%
It is impossible to inherit attributes from another object in \eolang{}.
The only two possible ways to re-use functionality are either via object
composition or decorators.
There are OO languages without implementation inheritance, for example Go~\citep{donovankernighan2015go},
but only Kotlin~\citep{jemerov2017kotlin} has decorators
as a language feature. In all other languages, the Decorator pattern~\citep{gamma1994design}
has to be implemented manually~\citep{bettinibono2008delegation}.

\feature{no-methods}{No Methods}
%
An object in \eolang{} is a composition of other objects and atoms:
there are no methods or functions similar to Java or C++ ones.
Execution control is given to a program when atoms' attributes are referred to.
Atoms are implemented by \eolang{} runtime similar to Java native objects.
To our knowledge, there are no other OO languages without methods.

\feature{no-ctors}{No Constructors}
%
Unlike Java or C++, \eolang{} doesn't allow programmers to alter
the process of object construction or suggest alternative
paths of object instantiation via additional constructions.
Instead, all arguments are attached to attributes ``as is'' and can't be modified.

\feature{no-static}{No Static Entities}
%
Unlike Java and C\#,
\eolang{} objects may consist only of other objects, represented
by attributes, while class methods, also known as static methods, as well as
static literals, and static blocks---don't exist in \eolang{}.
Considering modern programming languages, Go has no static methods either,
but only objects and ``\nospell{structs}''~\citep{schmager2010gohotdraw}.

\feature{no-primitives}{No Primitive Data Types}
%
There are no primitive data types in \eolang{}, which
exist in Java and C++, for example.
As in Ruby, Smalltalk~\citep*{goldbergrobson1983smalltalk}, Squeak, Self, and Pharo,
integers, floating point numbers, boolean
values, and strings are objects in \eolang{}:
``everything is an object'' is the key design principle, which,
according to~\citet[p.66]{west2004object}, is an ``obviously necessary prerequisite
to object thinking.''

\feature{no-operators}{No Operators}
%
There are no operators like \ff{+} or \ff{/} in \eolang{}. Instead,
numeric objects have built-in atoms that represent math operations. The same
is true for all other manipulations with objects: they are provided
only by their encapsulated objects, not by external language constructs, as in
Java or C\#. Here \eolang{} is similar to Ruby, Smalltalk and Eiffel,
where operators are syntax sugar, while implementation is encapsulated in
the objects.

\feature{no-null}{No NULL References}
%
Unlike C++ and Java, there is no concept of NULL in \eolang{}, which
was called a ``billion dollar mistake'' by~\citet{hoare2009null} and
is one of the key threats for design consistency~\citep{eo1}.
Haskell, Rust, OCaml, Standard ML, and Swift also don't have NULL references.

\feature{no-empty}{No Empty Objects}
%
Unlike Java, C++ and all other OO languages,
empty objects with no attributes are forbidden in \eolang{} in order
to guarantee the presence of object composition and
enable separation of concerns~\citep{dijkstra1982role}:
larger objects must always encapsulate smaller ones.

\feature{no-private}{No Private Attributes}
%
Similar to Python~\citep{lutz2013learning} and Smalltalk~\citep{hunt1997smalltalk},
\eolang{} makes all object attributes publicly visible.
There are no protected ones, because there is no implementation
inheritance, which is considered harmful~\citep{hunt2000}.
There are no private attributes either, because information
hiding can anyway easily be violated via getters, and usually is, making the code longer
and less readable, as explained by~\citet{holub2004more}.

\feature{no-global}{No Global Scope}
%
All objects in \eolang{} are attached to some attributes. Objects constructed
in the global scope of visibility are attached to attributes of the
\(\Phi\) object of the highest level of abstraction.
Newspeak and Eiffel are two programming languages that does not have global scope as well.

\feature{no-mutability}{No Mutability}
%
Similar to Erlang~\citep{armstrong2010erlang},
there are only immutable objects in \eolang{}, meaning that their attributes may
not be changed after the object is constructed or copied.
Java, C\#, and C++, have modifiers like
\ff{final}, \ff{readonly}, or \ff{const} to make attributes immutable, which
don't mean constants though. While the latter will
always expose the same functionality, the former may represent mutable
entities, being known as read-only references~\citep{birka2004practical}.
For example, an attribute \ff{r} may have an object \ff{random.pseudo}
attached to it, which is a random number generator. \eolang{} won't allow
assigning another object to the attribute \ff{r}. However, every time
the attribute is dataized, its value will be different.
%
There are number of OOP languages that also prioritize immutability of objects.
In Rust~\citep{matsakis2014rust}, for example, all variables are immutable by
default, but can be made mutable via the \ff{mut} modifier. Similarly,
D~\citep{bright2020origins} has qualifier \ff{immutable}, which expresses
transitive immutability of data.

\feature{no-functions}{No Functions}
%
There are no lambda objects or functions in \eolang{}, which exist in Java~8+, for example.
However, objects in \eolang{} have ``bodies,'' which make it possible to interpret
objects as functions.
Strictly speaking, if objects in \eolang{} would only have bodies and no other attributes,
they would be functions. It is legit to
say that \eolang{} extends lambda calculus, but in a different way
comparing to previous attempts made by~\citet{mitchell1993lambda} and \citet{di1998lambda}:
methods and attributes in \eolang{} are not new concepts, but lower-level
objects.

\feature{no-mixins}{No mixins}
%
There are no ``traits'' or ``mixins'' in~\eolang{}, which exist in Ruby and PHP to enable
code reuse from other objects without inheritance and composition.


\section{Four Principles of OOP}
\label{sec:four}
In order to answer the question, whether
the proposed object calculus is sufficient
to express any object model, in this Section we demonstrate
how four fundamental principles of OOP are realized by \phic{}:
encapsulation, abstraction, inheritance, and polymorphism.

\subsection{Abstraction}

Abstraction, which is called ``modularity'' by~\citet{grady2007object}, is,
according to~\citet[p.203]{west2004object},
``the act of separating characteristics into the relevant and
irrelevant to facilitate focusing on the relevant without
distraction or undue complexity.'' While \citet{stroustrup1997} suggests
C++ classes as instruments of abstraction, the ultimate goal
of abstraction is decomposition, according to~\citet[p.73]{west2004object}:
``composition is accomplished by applying abstraction---the
`knife' used to carve our domain into discrete objects.''

In \phic{} objects are the elements the problem domain
is decomposed into.
This goes along the claim of~\citet[p.24]{west2004object}:
``objects, as abstractions of entities in the real world,
represent a particularly dense and cohesive clustering of information.''

% Mention that dynamic dispatch, as the main property of abstraction,
% exists in EO.

\subsection{Inheritance}

Inheritance, according to~\citet{grady2007object}, is
``a relationship among classes wherein one class shares the structure
and/or behavior defined in one (single inheritance)
or more (multiple inheritance) other classes,'' where
``a subclass typically augments or
restricts the existing structure and behavior of its superclasses.''
The purpose of inheritance, according to~\citet{meyer1997object}, is
``to control the resulting potential complexity'' of the design
by enabling code reuse.

Consider a classic case of behaviour extension, suggested by~\citet[p.38]{stroustrup1997}
to illustrate inheritance. C++ class \ff{Shape} represents a graphic object
on the canvas (a simplified version of the original code):

\begin{ffcode}
class Shape {
  Point center;
public:
  void move(Point to) { center = to; draw(); }
  virtual void draw() = 0;
};
\end{ffcode}

The method \ff{draw()} is ``virtual,'' meaning that it is not implemented in the class
\ff{Shape} but may be implemented in sub-classes, for example in
the class \ff{Circle}:

\begin{ffcode}
class Circle : public Shape {
  int radius;
public:
  void draw() { /* To draw a circle */ }
};
\end{ffcode}

The class \ff{Circle} inherits the behavior of the class \ff{Shape} and
extends it with its own feature in the method \ff{draw()}. Now, when the
method \ff{Circle.move()} is called, its implementation from the class \ff{Shape}
will call the virtual method \ff{Shape.draw()}, and the call will be dispatched
to the overridden method \ff{Circle.draw()} through the ``virtual table'' in the class \ff{Shape}.
The creator of the class \ff{Shape} is now aware of sub-classes
which may be created long after, for example \ff{Triangle}, \ff{Rectangle}, and so on.

Even though implementation inheritance and method overriding seem to
be powerful mechanisms, they have been criticized.
According to~\citet{holub2003extends}, the main problem with
implementation inheritance is that it introduces unnecessary coupling
in the form of the ``fragile base class problem,''
as was also formally demonstrated by~\citet{mikhajlov1998study}.

The fragile base class problem is one of the reasons why
there is no implementation inheritance in \phic{}.
Nevertheless, object hierarchies to enable code reuse
in \phic{} may be created using decorators.
This mechanism is also known as ``delegation'' and, according
to~\citet[p.98]{grady2007object}, is ``an alternate approach to inheritance, in which
objects delegate their behavior to related objects.''
As noted by~\citet[p.139]{west2004object}, delegation is ``a way to extend or restrict
the behavior of objects by composition rather than by inheritance.''
\citet{seiter1998evolution} said that ``inheritance breaks encapsulation'' and suggested that
delegation, which they called ``dynamic inheritance,'' is a better way
to add behavior to an object, but not to override existing behavior.

The absence of inheritance mechanism in \phic{} doesn't make it
any weaker, since object hierarchies are available. \citet{grady2007object}
while naming four fundamental elements of object model mentioned
``abstraction, encapsulation, modularity, and hierarchy'' (instead of inheritance, like
some other authors).

\subsection{Polymorphism}

According to~\citet[p.467]{meyer1997object}, polymorphism means
``the ability to take several forms,'' specifically a variable
``at run time having the ability to become attached to objects
of different types, all controlled by the static declaration.''
\citet[p.67]{grady2007object} explains polymorphism as
an ability of a single name (such as a variable declaration)
``to denote objects of many different classes that are related by some common superclass,''
and calls it ``the most powerful feature of object-oriented programming languages.''

Consider an example C++ class, which is used by~\citet[p.310]{stroustrup1997}
to demonstrate polymorphism (the original code was simplified):

\begin{ffcode}
class Employee {
  string name;
public:
  Employee(const string& name);
  virtual void print() { cout << name; };
};
\end{ffcode}

Then, a sub-class of \ff{Employee} is created, overriding
the method \ff{print()} with its own implementation:

\begin{ffcode}
class Manager : public Employee {
  int level;
public:
  Employee(int lvl) :
    Employee(name), level(lvl);
  void print() {
    Employee::print();
    cout << lvl;
  };
};
\end{ffcode}

Now, it is possible to define a function, which accepts a set
of instances of class \ff{Employee} and prints them one by one,
calling their method \ff{print()}.:

\begin{ffcode}
void print_list(set<Employee*> &emps) {
  for (set<Employee*>::const_iterator p =
    emps.begin(); p != emps.end(); ++p) {
    (*p)->print();
  }
}
\end{ffcode}

The information of whether elements of the set \ff{emps} are instances of \ff{Employee}
or \ff{Manager} is not available for the \ff{print\char`\_list} function in compile-time.
As explained by \citet[p.103]{grady2007object},
``polymorphism and late binding go hand in hand;
in the presence of polymorphism, the binding of a method
to a name is not determined until execution.''

Even though there are no explicitly defined types in \phic{},
the comformance between objects is derived and ``strongly'' checked
in compile time. In the example above, it would not be possible to
compile the code that adds elements to the set \ff{emps}, if any
of them lacks the attribute \ff{print}. Since in \eolang{},
there is no reflection on types or any other mechanisms
of alternative object instantiation, it is always known where
objects are constructed or copied and what is the structure of them.
Having this information in compile-time it is possible to guarantee
strong compliance of all objects and their users. To our knowledge,
this feature is not available in any other OOP languages.

\subsection{Encapsulation}

Encapsulation is considered the most important principle of OOP
and, according to~\citet[p.51]{grady2007object},
``is most often achieved through information hiding,
which is the process of hiding all the secrets of an object
that do not contribute to its essential characteristics;
typically, the structure of an object is hidden, as well
as the implementation of its methods.'' Encapsulation in C++ and
Java is achieved through access modifiers like \ff{public} or
\ff{protected}, while in some other languages, like JavaScript or Python,
there are no mechanisms of enforcing information hiding.

However, even though~\citet[p.51]{grady2007object} believe that
``encapsulation provides explicit barriers among different
abstractions and thus leads to a clear separation of concerns,''
in reality the barriers are not so explicit: they can be easily
violated.
\citet[p.141]{west2004object} noted that
``in most ways, encapsulation is a discipline more than a real barrier;
seldom is the integrity of an object protected in any absolute
sense, and this is especially true of software objects,
so it is up to the user of an object to respect that object's encapsulation.''
There are even programming ``best practices,'' which encourage
programmers to compromise encapsulation: getters and setters are
the most notable example, as was demonstrated by~\citet{holub2004more}.

The inability to make the encapsulation barrier explicit
is one of the main reasons why there is no information hiding in \phic{}.
Instead, all attributes of all objects in \phic{}
are visible to any other object.

In \eolang{} the primary goal of encapsulation is achieved differently.
The goal is to reduce coupling between objects:
the less they know about each other the thinner the
the connection between them, which is one of the virtues of
software design, according to~\citet{yourdon1979structured}.

In \eolang{} the tightness of coupling between objects should be controlled
during the build, similar to how the threshold of test code coverage is
usually controlled. At compile-time the compiler collects the information
about the relationships between objects and calculates the coupling depth of each connection.
For example, the object \ff{garage} is referring to the object
\ff{car.engine.size}. This means that the depth of this connection between objects
\ff{garage} and \ff{car} is two, because the object \ff{garage} is using
two dots to access the object \ff{size}. Then, all collected depths from
all object connections are analyzes and the build is rejected if the numbers
are higher than a predefined threshold. How exactly the numbers are analyzed
and what are the possible values of the threshold is a subject for future
researches.





\section{Related Work}
\label{sec:related}
% SPDX-FileCopyrightText: Copyright (c) 2024-2025 Yegor Bugayenko
% SPDX-License-Identifier: MIT

Attempts were made to formalize OOP and introduce object calculus,
similar to lambda calculus~\citep{barendregt2012} used in functional programming.
For example, \citet{abadi1995imperative} suggested an imperative calculus of objects,
which was extended by~\citet{bono1998imperative} to support classes,
by~\citet{gordon1998concurrent} to support concurrency and synchronisation,
and by~\citet{jeffrey1999distributed} to support distributed programming.

Earlier, \citet{honda1991object} combined OOP and \(\pi\)-calculus in order to
introduce object calculus for asynchronous communication, which was further
referenced by~\citet{jones1993pi} in their work on object-based design notation.

A few attempts were made to reduce existing OOP languages
and formalize what is left. Featherweight Java is the most notable example
proposed by~\citet{igarashi2001featherweight}, which is
omitting almost all features of the full language (including interfaces and
even assignment) to obtain a small calculus.
Later it was extended by~\citet{jagannathan2005transactional} to support
nested and multi-threaded transactions. Featherweight Java is used in formal languages
such Obsidian~\citep{coblenz2019} and SJF~\citep{usov2020}.

Another example is Larch/C++~\citep{cheon1994quick}, which is a formal
algebraic interface
specification language tailored to C++. It allows interfaces of C++ classes and
functions to be documented in a way that is unambiguous and concise.

Several attempts to formalize OOP were made by extensions of the most popular
formal notations and methods, such as Object-Z~\citep{duke1991object} and
VDM++~\citep{durr1992vdm}. In
Object-Z, state and operation schemes are encapsulated into classes. The formal
model is based upon the idea of a class history~\citep{duke1990towards}.
Although, all these OO extensions do not have comprehensive
refinement rules that can be used to transform specifications into implemented
code in an actual OO programming language, as was noted by~\citet{paige1999object}.

\citet{bancilhon1985calculus} suggested an object calculus as an extension
to relational calculus. \citet{jankowska2003anotheroop} further developed
these ideas and related them to a Boolean algebra.
\citet{leekwakryu1996transform} developed an algorithm
to transform an object calculus into an object algebra.

However, all these theoretical attempts to formalize OO languages
were not able to fully describe their features, as was noted
by~\citet{nierstrasz1991towards}:
``The development of concurrent object-based programming languages
has suffered from the lack of any generally accepted formal
foundations for defining their semantics.'' In addition, when describing the
attempts of formalization, \citet{eden2002visual} summarized: ``Not one of the
notations is defined formally, nor provided with
\nospell{denotational} semantics,
nor founded on axiomatic semantics.''
%
Moreover, despite these efforts,
\citet{ciaffaglione2003reasoning,ciaffaglione2003typetheories,ciaffaglione2007theory_of_contexts}
noted in their series of works that a relatively little formal work has
been carried out on object-based languages and it remains true to this day.


\section{Acknowledgments}
% SPDX-FileCopyrightText: Copyright (c) 2016-2025 Objectionary.com
% SPDX-License-Identifier: MIT

Many thanks to (in alphabetic order of last names)
  \nospell{Fabricio Cabral},
  \nospell{Kirill Chernyavskiy},
  \nospell{Piotr Chmielowski},
  \nospell{Danilo Danko},
  \nospell{Konstantin Gukov},
  \nospell{Andrew Graur},
  \nospell{Ali-Sultan Ki\-giz\-ba\-ev},
  \nospell{Nikolai Ku\-da\-sov},
  \nospell{Alexander Legalov},
  \nospell{Tymur \(\lambda\)ysenko},
  \nospell{Alexandr Naumchev},
  \nospell{Alonso A. Ortega},
  \nospell{John Page},
  \nospell{Alex Panov},
  \nospell{Alexander Pushkarev},
  \nospell{Marcos Douglas B. Santos},
  \nospell{Alex Semenyuk},
  \nospell{Violetta Sim},
  \nospell{Sergei Skliar},
  \nospell{Stian Soiland-Reyes},
  \nospell{Viacheslav Tradunskyi},
  \nospell{Maxim Trunnikov},
  \nospell{Ilya Trub},
  \nospell{César Soto Valero},
  \nospell{David West},
  and
  \nospell{Vladimir Zakharov}
for their contribution to the development of \eolang{} and \phic{}.


{\raggedright
\bibliographystyle{ACM-Reference-Format}
\bibliography{bibliography/main}}

\clearpage

\end{document}
